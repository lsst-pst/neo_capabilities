
\section{Analysis of Moving Object Processing Software Performance}


MOPS: Given the expected false-positive rates demonstrated by
\cite{goldstein15} and in Colin's  section, LSST MOPS linking will
be possible. This has already been shown by the PanSTARRS project 
with simulations performed for the PS4 system\footnote{PanSTARRS 
PS1 experience does not contradict this conclusion. In addition to 
hardware issues, PS1 was only 1/4 of the assumed system (see 
\citep{denneau13} for more details). }, which is in this
respect equivalent to LSST. The robustness to unexpected false
positives is further tested with simulations performed by LSST,
as described below. 
   




XXX see below for basic scalings 

Zeljko and Mario. There is a report on MOPS (LDM-156)...   XXX refer to its
results here... 

Quoting \cite{denneau13}: ``MOPS achieves $>$99:5\% efficiency in
producing orbits from a synthetic
but realistic population of asteroids whose measurements were
simulated for a Pan-STARRS4-class telescope. \dots MOPS has been
adapted successfully to the prototype Pan-STARRS1 telescope despite
differences in expected false detection rates, fill-factor loss, and
relatively sparse observing cadence compared to a hypothetical
Pan-STARRS4 telescope and survey.'' 

But we did our own analysis, too...

The LSST project has developed an enhanced prototype implementation of MOPS.
We ran simulations with LSST system and cadence, and a significantly
wider range of false positive candidate rates. 

Known limitations/caveats:
\begin{itemize}
\item Due to computational constraints at the time when the simulation
  was performed (2011), a $v < 0.5$ deg/day velocity limit was
  imposed.
\item For similar reasons, the filters were imposed on track fitting
  were not optimized, artificially reducing the yield
\item As we understand the algorithmic scalings, these will not change the
final results; nevertheless they are being actively mitigated by new
simulations which are in progress.
\end{itemize}

Simulation results: Asteroids are discoverable in the presence of significant noise.
Get more details from Lynne \& Axelrod writeup. 

\newpage

\subsection{The Basic Strategy} 

First form tracklets, and then candidate tracks. Don't worry about false positives, 
be it detections, tracklets or tracks, because the initial orbit determination (IOD) 
will efficiently and reliably filter out false tracks (due to high-accuracy astrometry 
and well-understood simple Keplerian model predictions). The essential question
is whether the resulting numbers of false tracklets and tracks can be handled with 
available computing resources.

Assuming 0.005 sec per IOD (based on an analysis by Pan-STARRS MOPS team,
reference XXX; possibly shorter now), 1000 cores, and 12 hours of computation time 
per day, one can filter about 10$^{10}$ candidate tracks per day. How many tracks 
do we expect? 

MOPS tests show that with of the order a million tracklets per night (similar to 
the expected rate, as demonstrated below), a 30-day search window results in 
about the same number of tracks as input tracklets. Assuming 2 million tracklets
per night, a 30-day search window would result in about 60 million candidate
tracks. This expectation is more than two orders of magnitude smaller than the 
assumed system capacity above. As discussed below, the number of false tracks
could be decreased by another order of magnitude by decreasing the mean revisit
time to 10 minutes. 


\subsection{Expected False Tracklet Rates} 


Given a detection in the difference image, we search for a matching detection in another
difference image to form a tracklet. For orientation, the highest sky density of asteroids 
down to LSST faint flux limit ($r \sim 24.5$) is of the order $\rho_{ast} \sim 100$ deg$^{-2}$
(perhaps 2-3 times larger, depending on model assumptions). The number of false positives 
due to (gaussian) background fluctuations, assuming typical LSST seeing (0.8 arcsec) and
SNR$>$5, is about $\rho_{bkgd} = 60$ deg$^{-2}$. However, analysis of DECam images reduced 
using prototype LSST software shows a higher rate of detections in difference images, that 
cannot be readily associated with moving objects. This analysis implies a conservative upper
limit for the false positive rate of about $\rho_{FP} =  400$ detections deg$^{-2}$. This value 
is an upper limit because the analyzed fields are close to the Ecliptic, with a significant but
not well known contribution from real asteroids (due to very faint flux levels, $r \sim 24$). 
Hence, it is possible that the false positive rate is actually as much as four times lower. 

We will assume that the sky density of detections in difference images is given by 
$\rho_{det} = \rho_{ast} + \rho_{FP}$. When searching for a matching detection in another
difference image, there are two distinct types of behavior. Correct matches of detections
of the same asteroids into tracklets follow the behavior expected for correlated samples:
as long as the object's angular displacement between the two epochs is sufficiently larger 
than the seeing disk, while at the same time smaller than the search radius, the number
of matches is simply 
\begin{equation}
             N_{tracklet}^{true} = \rho_{ast}  \, A_{FOV},
\end{equation}
where $A_{FOV}$ is the field-of-view area (for LSST, $A_{FOV}=9.6$ deg$^2$). With 
$\rho_{ast} = 100$ deg$^{-2}$, $N_{tracklet}^{true} \sim 1,000$ per a pair of visits, and with
500 visit pairs per typical observing night, $N_{tracklet}^{true} \sim 500,000$ per night
(the implied number of asteroid detections per night is about a million, but note that 
the sky density of asteroids falls rapidly with the distance from the Ecliptic). We emphasize 
that this number of true tracklets does not directly depend on the search radius, nor the 
time elapsed between the two visits, as long as they have their plausible values (about an 
arcminute, and a few tens of minutes, as discussed further below). 

There are three other types of tracklets that follow behavior for uncorrelated (random) 
samples: associations of different asteroids, associations of asteroids and false detections, 
and tracklets made of two false detections. Assuming the same $\rho_{det}$ in both 
difference images, for each of $N_{det} = \rho_{det} \, A_{FOV}$ detections in one image,
we search for a matching detection in another image. The search radius is given by 
$\delta_{max} = v_{max} \, \Delta t$. Here $v_{max}$ is the  cutoff velocity and $\Delta t$ 
is the time elapsed between the two images. For LSST baseline cadence, $\Delta t$ is in 
the range 20-60 minutes. The search area, $A_S = \pi \delta_{max}^2$, is then 
\begin{equation}
     A_S = 0.0055 \left( v_{max}  \over {\rm deg \, day}^{-1} \right) \, \left(\Delta t \over {\rm hour} \right) {\rm deg}^2.
\end{equation}
Adopting $v_{max} = 1$ deg day$^{-1}$, which ensures a high completeness level even for fast-moving 
NEOs\footnote{Simulations imply that 95\% of NEO detections have $v<1$ deg day$^{-1}$; the completeness
for main-belt asteroids is essentially 100\%. In addition, objects moving faster than 1 deg day$^{-1}$ will
be resolved in LSST images and can be treated separately.}, and $\Delta t = 30$ minutes (which together 
imply $\delta_{max} = 1.3$ arcmin), gives a search area of $A_S = 0.0014$ deg$^2$. 

As long as $\rho_{det}$ is much smaller than $\rho_A = 1/A_S = 733$ deg$^{-2}$, the expected number of 
matches within the search radius is less than unity (for a discussion of second-order effects, see Appendix B 
in \citealt{IVLZ2005}). In this regime, the probability of forming a tracklet is 
\begin{equation}
                 p_{tracklet}^{false} =   { \rho_{det}  \over \rho_A}, 
\end{equation}
and the total expected number of {\it false} tracklets is 
\begin{equation}
           N_{tracklet}^{false} = N_{det} \, p_{tracklet}^{false} =  \rho^2_{FP}  \, A_S \, A_{FOV} \,
                                \left(1 + 2 \eta + \eta^2\right),
\end{equation}
where $\eta = \rho_{ast}  / \rho_{FP}$. With $\rho_{ast} = 100$ deg$^{-2}$ and  $\rho_{FP} = 400$ deg$^{-2}$,
$N_{tracklet}^{false} \sim 3,400$ per pair of visits, and $N_{tracklet}^{false} \sim 1.7$ million per observing night. 
Note that there are false tracklets even when $\rho_{FP} = 0$ because of incorrect associations of different asteroids:
this term contributes 0.1 million tracklets per observing night for $\rho_{ast} = 100$ deg$^{-2}$. The total
number of tracklets is thus to the first order ($\eta \approx 0$)
\begin{equation}
   N_{tracklet} =  N_{tracklet}^{true} + N_{tracklet}^{false} = \rho_{ast}  \, A_{FOV} + \rho^2_{FP}  \, A_S \, A_{FOV}. 
\end{equation}
Therefore, there will be up to about $2\times10^6$ tracklets per observing night, for the chosen parameter 
values. Given that these choices are rather conservative, this estimate is essentially an upper limit;
approximately, {\it we expect of the order a million tracklets per observing night}. 


In addition to $N_{tracklet}^{false}$ scaling with the square of $\rho_{FP}$, $N_{tracklet}^{false}$ scales with the squares of
both $v_{max}$ and  $\Delta t$ (via the dependence on $A_S$). Therefore, if $\Delta t$ would be made
as small as 10 minutes by modifying observing strategy, the resulting $N_{tracklet}^{false}$ would be about an 
order of magnitude smaller (and $N_{tracklet}$ about three times smaller).  Hence, the shortening of $\Delta t$ is 
a good mitigation strategy against high false positive detection rates in difference images\footnote{An
extreme example of this mitigation strategy would be to obtain two consecutive 30-second visits separated 
by 34 seconds (additional 2 seconds due to shutter motion and another 2 seconds due to readout). 
The acceptable false positive density would be increased by about three orders of magnitude, with the
lower limit for the detectable motion of the order 0.1 deg day$^{-1}$.}.


\subsubsection{Can MOPS handle a million tracklets per night?} 

How many tracks can we form with $\sim$10$^6$ tracklets per night, and a 30-day search window? 
Based on MOPS test runs, we should expect well below 10$^8$ candidate tracks. LSST's 1000-core
system will be capable of easily handling about 10$^{10}$ IODs per day, so we have a margin
of about two orders of magnitude. Hence, {\it even if the false positive rate in difference images is 
ten times higher than expected, it can still be handled without a change of baseline cadence.}
And yet another factor of a few increase in the false positive rate can be mitigated by a simple
shortening of the revisit time by about a factor of 3. At the same time, an order of magnitude 
larger false positive rates for LSST than measured using DECam images and prototype LSST software
are rather implausible. If the LSST camera, or other system component, would somehow
cause such high false positive rates, the whole LSST mission would indeed be a failure. 

