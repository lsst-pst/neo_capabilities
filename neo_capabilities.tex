\documentclass[12pt,preprint]{aastex}
\usepackage{lsst}
\usepackage{xspace}
\usepackage[english]{babel}
\usepackage[utf8x]{inputenc}
\usepackage{amsmath}
\usepackage{graphicx}
\usepackage{longtable}
\usepackage{hyperref}
\usepackage{comment}

\newcommand{\Alert}{\code{Alert}\xspace}
\newcommand{\Alerts}{\code{Alerts}\xspace}
\newcommand{\DIASource}{\code{DIASource}\xspace}
\newcommand{\DIASources}{\code{DIASources}\xspace}
\newcommand{\DIAObject}{\code{DIAObject}\xspace}
\newcommand{\DIAObjects}{\code{DIAObjects}\xspace}
\newcommand{\DB}{{Level 1 database}\xspace}
\newcommand{\DR}{{Level 2 database}\xspace}
\newcommand{\Object}{\code{Object}\xspace}
\newcommand{\Objects}{\code{Objects}\xspace}
\newcommand{\Source}{\code{Source}\xspace}
\newcommand{\Sources}{\code{Sources}\xspace}
\newcommand{\ForcedSource}{\code{ForcedSource}\xspace}
\newcommand{\ForcedSources}{\code{ForcedSources}\xspace}
\newcommand{\CoaddSource}{\code{CoaddSource}\xspace}
\newcommand{\CoaddSources}{\code{CoaddSources}\xspace}
\newcommand{\SSObject}{\code{SSObject}\xspace}
\newcommand{\SSObjects}{\code{SSObjects}\xspace}
\newcommand{\VOEvent}{\code{VOEvent}\xspace}
\newcommand{\VOEvents}{\code{VOEvents}\xspace}
\newcommand{\transSNR}{5\xspace}

\begin{document}
\title{Large Synoptic Survey Telescope as a Near-Earth Object Discovery Machine}

\author{R. Lynne Jones$^1$, Colin Slater$^1$, Joachim Moeyens$^1$, 
Lori Allen$^2$, Mario Juri\'{c}$^1$,  \and \v{Z}eljko Ivezi\'{c} $^1$}

\affil{
$^1$University of Washington, \\
$^2$National Optical Astronomy Observatory}

\begin{abstract}
We discuss the ability of LSST to contribute to Near-Earth Objects (NEO) discoveries and
Congressional Brown mandate to NASA. The two main issues addressed here are robustness 
of the LSST strategy for discovering NEOs using nightly pairs of observations, and the 
expected cumulative completeness for potentially hazardous asteroids (PHAs) with 
visual absolute magnitudes $H<22$.  We argue that the observing and data processing 
strategies chosen by LSST are robust, and would yield a completeness of about 70\% with 
the current LSST baseline survey. We describe a number of modifications of the LSST baseline 
survey which could potentially yield the completeness of 90\% for $H<22$ PHAs. 
\end{abstract}

\keywords{}


\section{Introduction}

XXX This is so totally work in progress. ZI push-ed it just as a backup... XXX

Main parts:
\begin{itemize}
\item NEO impacts as a concern; the Brown mandate, NASA panels, introduce LSST 
\item LSST defaults and claimed performance in older publications
\item informal concerns by the community, GMS paper
\item study by the JPL team, technical support from LSST (assuming baseline cadence)
\item exploration of baseline cadence modifications designed to boost NEO completeness 
\item Outline of this paper
\end{itemize} 



The small-body populations in the Solar System, such as asteroids, trans-Neptunian objects (TNOs) 
and comets, are remnants of its early assembly. Collisions in the main asteroid belt between Mars and 
Jupiter still occur, and occasionally eject objects on orbits that may place them on a collision course 
with Earth. About 20\% of this near-Earth Object (NEO) population, the so-called potentially hazardous 
asteroids (PHAs), are in orbits that pass sufficiently close to Earth's orbit, to within 0.05 AU, that 
perturbations with time scales of a century can lead to intersections and the possibility of collision. 
In December 2005, the U.S. Congress directed\footnote{For details see http://neo.jpl.nasa.gov/neo/report2007.html} 
NASA to implement a NEO survey that would catalog 90\% of NEOs with diameters larger than 140 meters 
by 2020 (the George E. Brown, Jr. mandate). For a compendium of information about NEOs and PHAs 
and an up-to-date summary of discovery progress, see NASA's NEO webpage\footnote{http://neo.jpl.nasa.gov/neo/}. 

The completeness level set by Congressional mandate can be fulfilled with a 10-meter-class ground-based
telescope equipped with a multi-gigapixel camera, and a sophisticated and robust data processing system. 
The Large Synoptic Survey Telescope (LSST), currently being constructed, is such a system (for a concise
system description, science drivers and other information, see \citep{LSSToverview}). Early simulations of 
LSST performance presented by \cite{IvezicNEO2007} showed that the 10-year baseline cadence would 
result in 75\% completeness for PHAs greater than 140 m (more precisely, for PHAs with $H<22$; see 
\S~XX for further discussion). They also suggested that with additional optimization of the observing cadence, 
LSST could achieve 90\% completeness. Such optimization was discussed by \cite{LSSToverview} who
reported that, to reach 90\% completeness, about 15\% of observing time would have to be dedicated to NEOs
and the survey would have to run for 12 years.  
%% From the overview paper: 
%% - the LSST baseline cadence provides orbits for 82% of PHAs larger than 140 meters after 10 years of operations
%% - 84% completeness with minor changes to the cadence (5% of time for NEO-optimized observations)
%% - 90% completeness with major changes to the cadence (15% of time for NEO-optimized observations and 12 years)
The latest LSST simulation results, presented in \cite{JJI2016}, yielded a completeness of $\sim$72\% for
PHAs with $H<22$ using the current 10-year baseline survey. The minor difference compared to older
studies is attributable to the differences in simulated NEO populations and other modeling details. 


\cite{JPLstudy} described a new study, related to this work. Their preliminary results indicate a completeness
of $\sim$65\% for NEOs with $H<22$. The difference compared to \cite{JJI2016} result (72\%) is due 
to XXX (true?): PHA vs. NEO, possible slope of the size distribution. 


We're going to evaluate moving object detection capabilities with
LSST, comparing performance of our current prototype pipelines with 
the required performance during operations.  The two main goals are
to demonstrate that i) MOPS can cope with false detection in image differences,
and that ii) the NEO detection performance of the LSST baseline cadence can be
further boosted by adequate modifications 


From JPL paper -- which gives a concise summary of the main simulation aspects. 

The LSST baseline survey cadence relies primarily on single night pairs of detections, 
with roughly 30 minutes between the elements of a detection pair. These pairs form 
what are known in MOPS parlance as tracklets, and sets of tracklets are linked across 
multiple nights to form tracks, which can then be sent to the final step, which is orbit 
determination. The strategy of using pairs is an aggressive and potentially fragile
approach, but theoretically represents the most productive NEO search with the minimum 
impact on other LSST science drivers. An alternative to visit each field three times per 
night to form tracklets from triplets of detections may prove more robust, but likely 
with a penalty of reduced performance. One of our study objectives is to understand the
tradeoffs between these two approaches.

The two major questions to be addressed by our study can be informally stated as 
``Will MOPS work?'' and ``If MOPS works what fraction of NEOs will LSST discover?''. 

Main problems:
\begin{enumerate}
\item Linking large number of detections in the presence of false positives (false detections due to problems 
with image differencing software). 
\item Adequacy of data, including image depth, sky coverage and cadence, to reach the required 
completeness level. 
\end{enumerate} 

Therefore, the main analysis components to check are: 
\begin{enumerate}
\item The performance of image differencing, with emphasis on the rate and properties of 
   false detections 
\item Linking large number of detections in the presence of false positives 
\item Observing cadence simulations coupled with NEO population models to forecast 
        discovery rates 
\end{enumerate} 



\cite[hereafter GMS]{GMS2016} reported different NEO completeness levels than
published by the LSST team in 2007 and 2014 . There are three main 
reasons why the GMS results differ:
\begin{enumerate}
\item GMS used a different realization of the LSST baseline survey
\item GMS used different synthetic NEO populations to evaluate completeness
\item GMS {\it redefined} the completeness limit from the commonly
  used $H<22$ criterion to an albedo-dependent value of $H$ limit (which
  attempts to directly model the $D>140$ m size cut)
\end{enumerate}

Regarding the last point, GMS found that the completeness drops by 5\%
when $H<22$ criterion is replaced by $D>140$ m criterion. Regarding 
the first point, GMS results can be more meaningfully compared to an LSST
study by \cite{JJI2016}, who used the same simulated cadence. After accounting for 
the $H<22$ vs. $D>140$ m methodological difference of 5\%, GMS obtained a 
completeness of 67\% using 3 pairs in 12 nights (for simulated cadence {\it enigma\_1189}), while Jones et al. 
study obtained $\sim$73\% using 3 pairs in 15 nights (for simulated cadence {\it minion\_1016}, which is 
statistically very similar to {\it enigma\_1189}). This difference 
of $\sim$5\% is attributable to the differences in simulated NEO
populations and other modeling details. In summary, GMS find the NEO
completeness in the range $\sim$60\% to $\sim$70\% for the LSST
baseline cadence. The variation is due to different NEO populations,
different NEO detection criteria, and other specifics. When accounting 
for different choices of simulation parameters, their results are
consistent with the results published by the LSST team. 

But: how much higher can the completeness be pushed with cadence
modifications optimized for NEOs? 

In Observing Strategy white paper: fig. 3.4 gives 73.4\% for PHAs with 
$H<22$ and {\it minion\_1016}, using 3 pairs in 15 nights. 


\begin{enumerate} 
\item Control and quantify the rate of (false positive) detections
\item Software (and computational capacity) capable of inter-night linking of detections given the expected rates
\item Quantify the discovery yields (and their robustness) under those assumptions
\end{enumerate} 


From Mario's talk to NASA:

LSST will detect variability (motion and flux variability) by
differencing each incoming image against a deep template.
Sources will be detected at an S/N=5 threshold (see Appendix A). 

We expect on average about 1,000 per sq. deg. astrophysical, real,
detections, including up to about 500 asteroids per sq. deg on the 
Ecliptic (for scale, the LSST field of view is about 10 sq. deg., with 
about 20 4kx4k CCDs per sq. deg.)

We also expect a false-positive detections due to random
fluctuations in the background at a level of about 200 per
sq. deg. (all at the faint end).  XXX check Colin's report 

However, historically surveys have reported factors of 10 to 500 times
more (depending on the survey; see \citep{denneau13};
\citep{goldstein15} ). 
Those additional false positive
detections are due to systematic effects: 
\begin{itemize} 
\item Camera and telescope artifacts
\item Imperfect image subtractions
\item Cosmic rays
\end{itemize} 

For a ``menagerie'' of artifacts (with amusing names such as 
{\it chocolate chip cookies, frisbee, piano, arrowhead, UFO}), from
Pan-STARRS, see Fig.~17 in \cite{denneau13}. 


``Many of the false detections are easily explained as internal
reflections, ghosts, or other well-understood image artifacts,...''


Learning from PS1 Experience: PanSTARRS was a first generation
experiment. Over the past decade, subsequent surveys (including LSST) 
have learned tremendously from the PS1 experience. There are surveys 
running today which have largely solved the key problems that PS1 has encountered.
These are recent developments driven largely by extragalactic and
transient science cases. They are not yet well known beyond those
communities and reporting on those developments is additional
motivation for this paper. 
 
Major improvements to hardware include CCDs with significantly fewer 
artifacts (e.g. DECam, see below; LSST) and optical systems designed to
minimize ghosting and internal reflections (e.g. LSST). 

Improvements to the software include advanced image differencing
pipelines (e.g., PTFIDE for the Palomar Transient Factory and the
Zwicky Transient Facility) and various machine learning classifiers
for filtering false positives (see below). 

DECam: \cite{goldstein15} 

The Dark Energy Survey (DES) is an optical/near-infrared survey that
aims to probe the dynamics of the expansion of the universe and the
growth of large scale structure by imaging 5,000 sq. deg. of the
southern sky. It is technologically very similar to LSST with
\begin{itemize}
\item 520 Mpix camera, 62 mosaicked chips (deep depleted devices)
\item 3 sq.deg. field of view, same filter bands as LSST
\item Single-exposure depths comparable to LSST
\item Includes a supernova search program which employs image
differencing methods analogous to LSST’s  and detects objects at the 
same effective signal-to-noise ratio as LSST (S/N=5)
\end{itemize} 

The false positives in DECam data are morphologically much simpler
(compare Fig.~1 in \citep{goldstein15} to Fig.~17 in \citep{denneau13})
than those in Pan-STARRS, and thus are much more amenable to automated 
screening using machine learning methods. Using a Random Forest 
classifier, \cite{goldstein15} cleaned their sample from having a 
raw false detection rate of 13:1 to a filtered rate of 1:3. This performance
is already within the acceptable range for LSST performance goals. 


{\bf Need to refer to section by Colin.} 

LSST will use two methods to detect moving objects
\begin{enumerate}
\item Detecting trailed motion on the sky:  objects trailed by more
  than 2 PSF widths (corresponding to motion faster than about 1
  deg/day) will be easily detectable as trailed.  Two trailed
  detections within 30--60 minutes in a single night will be
  sufficient to identify an object as an NEO candidate,
\item Inter-night linking of pairs: this technique will recover
  objects moving too slow enough to be measurably elongated in 
  a single exposure. 
\end{enumerate} 

MOPS: Given the expected false-positive rates demonstrated by
\cite{goldstein15} and in Colin's  section, LSST MOPS linking will
be possible. This has already been shown by the PanSTARRS project 
with simulations performed for the PS4 system\footnote{PanSTARRS 
PS1 experience does not contradict this conclusion. In addition to 
hardware issues, PS1 was only 1/4 of the assumed system (see 
\citep{denneau13} for more details). }, which is in this
respect equivalent to LSST. The robustness to unexpected false
positives is further tested with simulations performed by LSST,
as described below. 
   



 


\section{LSST Strategy for Discovering Solar System Objects} 

Opening paragraph (lift text from overview and Lynne's IAU paper) 

Possible subsections here or below:

- Concepts for discovering moving objects

- Outline for simulations (see Chesley) 


\subsection{Brief Overview of LSST and LSST Survey Design} 
%\subsection{Brief Overview of LSST} 

LSST will be a large, wide-field ground-based optical telescope system
designed to obtain multiple images covering the sky that is visible
from Cerro Pach\'{o}n in Northern Chile. The current baseline design,
with an 8.4m (6.7m effective) primary mirror, a 9.6 deg$^2$ field of
view, and a 3.2 Gigapixel camera, will allow about 10,000 square
degrees of sky to be covered every night, with typical 5$\sigma$ depth 
for point sources of $r\sim24.5$ (AB). The system is designed to yield 
high image quality (the median delivered seeing in the $r$ band of 
about 0.8 arcsec) as well as superb astrometric  and photometric 
accuracy\footnote{For detailed specifications, please see the LSST
Science Requirements Document, http://ls.st/srd}. The total survey
area will include $\sim$30,000 deg$^2$ with $\delta<+34.5^\circ$, and 
will be imaged multiple times in six bands, $ugrizy$, covering the 
wavelength range 320--1050 nm. For a more detailed, but still concise,
summary of LSST, please see 
the LSST Overview paper\footnote{arXiv:0805.2366, http://ls.st/2m9}. 

The project is scheduled to  begin the regular survey operations at
the start of next decade. About 90\% of the observing time will be
devoted to a deep-wide-fast survey mode which will uniformly observe 
a 18,000 deg$^2$ region about 1000 times (summed over all six bands) 
during the anticipated 10 years of operations, and yield a coadded map 
to $r\sim27.5$. These data will result in catalogs including about
$40$ billion stars and galaxies, that will serve the majority of the
primary science programs. The remaining 10\% of the observing time
will be allocated to special projects such as a Very Deep and Fast
time domain survey\footnote{Informally known as ``Deep Drilling Fields".}.

The LSST will be operated in fully automated survey mode. The images
acquired by the LSST Camera will be processed by LSST Data Management
software \cite{juric15} to a) detect and characterize imaged
astrophysical sources and b) detect and characterize temporal changes
in the LSST-observed universe. The results of that processing will be
reduced images, catalogs of detected objects and the measurements of 
their properties, and prompt alerts to ``events'' -- changes in
astrophysical scenery discovered by differencing incoming images
against older, deeper, images of the sky in the same direction ({\em
emplates}, see \S \ref{sec:AppA} for more details). 
 


\subsection{LSST Observing Strategy} 

XXX get text from the overview paper and Science Book 

As deployed and funded (by the U.S National Science Foundation and
Department of Energy), LSST is primarily a science-driven mission. 
Existing cadence is optimized to maximize the overall science returns
(incl. Solar System science), rather than NEO/PHA discovery
completeness (though the two goals are highly interrelated).  As designed, the survey is not optimized for rapid
discovery and follow-up of all types of moving objects\footnote{
XXX What's the purpose of this footnote? Note that LSST will enable rapid identification and follow-up of
trailed objects (within 60 seconds of discovery). If deployed with a 
planetary-defense optimized cadence, the NEO yields could be
significantly improved, and approaching the 90\% completeness level
for $H<22$.} 
Early simulations indicate 90\% is achievable for NEO-optimized
cadence. However, other science goals would be affected (including
Solar System science!).  XXX Refer to Jones et al.  (2016) 

The current baseline cadence is optimized for science returns.
It is expected to yield approximately $\sim$70\% of the extant NEO population.


\subsection{Overview of LSST  Data Management and Image Processing} 

Refer to \cite{DM2016} and Appendix A. 
 

%\section{Image Differencing Performance}

\subsection{Intro and Test Data}

False positives have historically dominated over true transients in previous
surveys. \citep{goldstein15, kessler15}

Major impediment to linking detections of moving objects

Develop a conservative estimate of the FP rate in the LSST stack, using similar
sensors and the current software.

Data from Allen et al., offers a useful cadence. Not the same depth or seeing,
but sufficient to test the software capabilities. Stack has not been optimized
to run at production-levels yet, so this is the starting point of that effort,
not an upper limit on LSST's performance.

\subsection{Processing}

Some reference to LSST software heritage and version, circa Jan 2016.

Differencing two images, rather than a template vs image. Running force
photometry. Ingest into DB.

\subsection{Correlated Noise}

Histogram of stack-reported diaSource SNR, alongside force photometry SNR.

Because we are diffing two images, we know that the photometry of the diaSource
must be the same as the difference of the force photometry on the direct images.
Gives us independent check on SNR reporting.

Strong ramp up of sources at SNR 5-6, but order of magnitude more than should be
present due to Gaussian statistics.

Check with force photometry shows that many of these cannot actually have the
high SNRs reported, more like 4-sigma events. FP values consistent with Gaussian
statistics.

Issue is tracking correlated noise after PSF-matching the science image. Known
issue, ref Paul \& Gene somehow (must be more references too).

For our purposes, we will estimate SNR using force phot. Several choices of long
term solutions; implementations are being studied now.

\subsection{False-Positive Results}

Using corrected SNR cut, ~1000 diaSources per square degree (only positive,
ignoring negative). ~500 on best fields.

Many of these are poorly-subtracted stars. To estimate rates of candidate moving
objects, exclude all diaSources that had detections in both direct images (at
the same position). These are still FPs, but they are someone else's FPs.
Resulting levels are ~350 per sq deg. Irreducable noise level is 33/sq deg.

Results from trying to make tracklets, hopefully?? On visual inspection, 25\%
look like junk, 25\% looked like good detections, and the rest were ambiguous
because low SNR. SNR power law exponent is $\sim -2.5$.

Very few detections around bright stars; mostly well-masked by the code. Few
large scale detected artifacts, but not a very big sample of images.

How detailed of a recipe are we providing here; should it be the same level of
detail as we sent to JPL?

%Paper on ``Automated Transient Identification in the Dark Energy Survey''  is \cite{goldstein15}. 


\section{Analysis of Moving Object Processing System Performance \label{sec:mops}}


The linking of individual detections from difference images into plausible orbital tracks will be performed using
a special-purpose code referred to as the Moving Object Processing System (MOPS). There are several slightly modified
versions of MOPS in use by various projects; the original version was developed collaboratively by Pan-STARRS
and LSST, and is described in \cite{denneau13}. MOPS employs a two-step processing: first pairs of detections
from a given night are connected into {\it tracklets}, and then at least three tracklets are associated into a
candidate {\it track}. Realistic MOPS simulations show high linking efficiency ($>$99\%; \citealt{denneau13})
across all classes of Solar System objects. The core algorithmic components of MOPS are {\it findTracklets} and
{\it linkTracklets} kd-tree algorithms by \citet{kubica07}. {\it findTracklets} links \DIASources from a single
night to produce {\it tracklets}, and {\it linkTracklets} links tracklets from at least three nights to produce candidate
{\it tracks} (assuming quadratic motion in each coordinate; the LSST version also accounts for topocentric
corrections). Candidate tracks produced by MOPS are then filtered using initial orbital determination (IOD) step,
which is executed using a stand-alone code (e.g. OrbFit, \citealt{milani08}; OpenOrb, \citealt{OpenOrb2009}).

Given the empirically estimated false positive rates expected for LSST, discussed in the preceeding section,
in this section we show that MOPS performance is already adequate - MOPS requires significantly less
computing capacity than planned for other LSST data processing needs. In addition to reporting the results of
numerical experiments with MOPS, we also analyze them using analytic and semi-analytic results for the
rates of false tracklets and false tracks.




\subsection{A Summary of LSST tests of MOPS}

As a part of the Final Design Review preparations, the LSST team has developed an enhanced prototype
implementation of MOPS and analyzed its behavior. Here we summarize the main results of that work;
a detailed report is publicly available \citep{LDM-156}.

Simulated \DIASources were based on a Solar System model by \citet{Grav2011}.
The model includes about 11 million objects; about 9 million are main-belt asteroids. Observations span
30 days and were selected from a simulated baseline cadence (at that time, the baseline simulation was
OpSim3.61, which in this context is statistically the same as the current baseline cadence, {\it minion\_1016}).
The number of tracklets and tracks, the runtime, and the memory usage were studied as functions of
the false positive detection rate. The rate was varied from none to four times the asteroid detection rate
(100 deg$^{-2}$).  The highest rate corresponds to the expected false positive detection rate for LSST
($\rho_{FP} =  400$ deg$^{-2}$).

Tests were run with 16 threads on single 16 CPU node on Gordon cluster at San Diego Supercomputing
Center (in 2011). Due to computational constraints, a $v < 0.5$ deg/day velocity limit for pairing detections
into tracklets was imposed. For similar reasons, the filters that were imposed on track fitting were not
optimized, artificially reducing the yield. As we now understand the algorithmic scalings much better
(see Appendix~\ref{sec:appMOPS}), it is clear that these unoptimized filters have no major impact on the
simulation results and derived conclusions.

As expected, the addition of false detections increases the number of tracklets and tracks,
the runtime, and the memory usage. For the 4:1 false:true detection rate ratio, compared to case with
no false detections, the number of tracklets increases by about
a factor of 10, the number of tracks by about 50\%, and runtime increases by about a factor of 3.
For the 4:1 false:true detection rate ratio, the runtime with 16 CPUs is 33 hours, with maximum memory
usage of about 80 GB.




\begin{figure}[t!]
\centering
\vskip -0.3in
%\includegraphics[width=0.49\textwidth]{figures/tracklet}
%\includegraphics[width=0.49\textwidth]{figures/tracks}
\includegraphics[width=0.95\textwidth]{figures/track_stats}
\caption{A summary of MOPS tests for the dependence of the number of tracklets (left)
and tracks (right) on the false detection rate. As the rate of false detections
increases from none to four times the asteroid detection rate, the number of tracklets
increases by about an order of magnitude. At the same time, the number of candidate
tracks increases by only about 50\%.
\label{fig:MOPStests}}
\end{figure}




\subsection{Understanding MOPS Performance}

The rather slow increase of the number of tracks with false positive detection rate (a 50\% increase
although the number of tracklets increased by a factor of 10) is somewhat unexpected. We have
developed analytic and semi-analytic analysis to better understand the scaling of the number of
tracklets and tracks with false detection rate and other relevant parameters. Details of this
analysis are provided in Appendix~\ref{sec:appMOPS}. Here we briefly discuss the main results.

The increase of the number of tracklets with the false detection rate,
$\rho_{FP}$, shown in left panel in Figure~\ref{fig:MOPStests}, is well
described by eq.~\ref{eq:NttFalse}. In particular, the number of tracklets
approximately increases proportionally to $(C_1 + C_2\rho_{FP}^2)$, where $C_1$
and $C_2$ do not depend on $\rho_{FP}$. As both the full analytic result and the
simulations show, false tracklets quickly outnumber true tracklets even at low
false detection rates, resulting in the observed $\rho_{FP}^2$ behavior.

While the number of tracklets is dominated by the false detections, in the
baseline LSST cadence and the nominal noise assumptions under which the MOPS
simulations were run ($\rho_{FP} \leq 500\,\rm{deg}^{-2}$), the number of
tracks is not dominated by spurious detections---instead it is dominated by true
tracks and mislinkages between true objects. This is due to the fundamental
feature of MOPS: the 4-dimensional space of tracks (two coordinates and two
velocity vector components) is sparse at up to moderate levels of contamination,
and at the tested noise levels false tracklets are effectively filtered out. This
behavior accounts for the slow growth in tracks in the right panel of
Figure~\ref{fig:MOPStests}.

As we evaluate the impact of different survey parameters, we can assess the
number of tracks that would be generated (and thus require IOD processing) using
the analytic results developed in Appendix~\ref{sec:appMOPS}. For a given window
width and false detection density, the number of false tracks per search window
that would arise from false detections is given by
\begin{equation}
\label{eq:falsetracks2}
   N^{falsetracks} = 4.5 \times 10^6 \, \left( {N_w \over 30 \, {\rm day} } \right)^{8} \left( {\rho_{FP} \over 400 \, {\rm deg}^{-2} }\right)^{3.7}.
\end{equation}
This expression is valid around fiducial values and assumes $\rho_{ast}=100$ deg$^{-2}$.
The number of true tracks is of the order 10$^6$; therefore, with the baseline
window $N_W=15$ the contribution of false detections is small, while in the
enhanced NEO cadences with $N_W=30$ the contribution is only a factor of a few
times the number of true tracks.

\subsection{Required Computing Resources for MOPS and IOD Processing}

Given the modest computing resources used in MOPS tests described above, the runtime and memory
usage results bode well for LSST processing. Assuming a 1000-core machine dedicated to LSST moving
object processing (corresponding to about 1\% of the anticipated total LSST compute power at the
National Center for Supercomputing Applications), MOPS runtime for producing
candidate tracks should not exceed an hour, assuming sufficient parallelization can be achieved.

The IOD step can also be handled with anticipated resources and is trivially parallelizable. The number 
of available IOD computations for a compute system with $N_{core}$ cores and allocated runtime $T_{runtime}$ 
can be estimated as
\begin{equation}
  N_{IOD} = 3.6\times10^8 \left({ 0.1\,{\rm sec} \over T_{IOD}}\right) \,
                                         \left({ T_{runtime}  \over 10\,{\rm hr} }\right) \,
                                         \left({ N_{core}  \over 1000}\right).
\end{equation}
where $T_{IOD}$ is the time it takes to perform one IOD computation on a single core. Estimates
of the upper range of $T_{IOD}$ are of the order 50 ms (S. Chesley, priv. comm.), considerably below the fiducial
value of 100 ms adopted here. Tests of IOD with 
Find_Orb\footnote{\url{https://www.projectpluto.com/find_orb.htm}} using observations similar to
those from our short-arc MOPS tracks required about 0.3 ms to complete.
Given that the expected number of candidate tracks to filter using
IOD is well below $10^7$, it should be possible to accomplish the IOD step in well under an hour.
Alternatively, it is plausible that a 100-core machine might be sufficient for LSST moving object
processing (assuming no engineering safety margin).
 

\section{LSST Observing Cadence Optimization to Enhance PHA Completeness}

The effects of varying the LSST observing strategy on PHA completeness and other science can be evaluated in detail using a combination of the LSST Operations Simulator (OpSim) and the LSST Metrics Analysis Framework (MAF). 

The LSST Operations Simulation (OpSim) is a python software package that generates a realistic pointing history, with the time, filter, location, astronomical conditions, weather conditions, and predicted point-source 5-sigma limiting magnitude, for each LSST visit for ten years. This pointing history is generated using weather data (cloudiness and seeing) from the Cerro Pachon site and a high-fidelity model of the telescope itself (including slew and settle time and dome movement, for example), combined with a parameterized set of observing proposals that determine how the scheduling algorithm attempts to gather observations. By configuring OpSim with different parameters for the observing proposals, we can generate a series of simulated surveys which prioritize different science goals. 

The LSST Metrics Analysis Framework (MAF) is a user-oriented, python package for evaluating the pointing history from these simulated surveys in light of particular science goals or interests. The results of metrics coded in the MAF framework can be calculated for any given simulated survey and compared as proposal parameters are changed in OpSim. Metrics will be gathered from as wide a cross section of the astronomical community as possible, together with "figures of merit" that summarize and define 'success' for a given metric. This permits a thorough investigation of the trades between different observing strategies, in terms of the effect on science goals.

We can use MAF to evaluate the effect of various observing strategies on moving object completeness and characterization as well. Here we focus on discovery completeness of Potentially Hazardous Asteroids (PHAs) as both the observing strategy (related to changes in OpSim proposal parameters) and discovery criteria (related to LSST Data Management (DM) and the Moving Object Processing Software (MOPS) workloads) are varied. 

\subsection{Details of MAF analysis}

Using MAF to evaluate metrics for moving objects, such as PHAs, requires first defining the parameters of the input small body population by:

\begin{enumerate}
\item{Defining an orbital distribution for the input moving object population, specified by a set of orbital parameters for each object. Here we have chosen to use an orbital distribution defined by the large ($>1$ km) diameter known PHAs, as reported to the Minor Planet Center (MPC). This population is thought to be relatively complete and thus should be relatively unbiased. For comparison with other completeness estimates which use an Near Earth Object (NEO) population based on the \cite{Bottke2002} model instead of the subset of these objects which are classified as PHAs, we also evaluate completeness using a random sample of 2000 NEOs from the synthetic solar system model presented in \cite{Grav2011}.  A plot of the $a$, $e$, $i$ distribution for these PHAs and NEOs is shown in Figure~\ref{fig:PHA_orbits}.  }
\begin{figure}
\centering
\includegraphics[width=0.45\textwidth]{figures/pha20141031_orbits} 
\includegraphics[width=0.45\textwidth]{figures/neos_2k_orbits}
\caption{The eccentricity and inclination distributions, as a function of semi-major axis, of the PHAs (left) and NEOs (right) used in this analysis. The PHA population consists of the orbital distribution of $>1$~km PHAs recorded by the MPC as of 2014 (1511 objects). The NEO population is a random sampling of 2000 NEOs from the S3M \citep{Grav2011}, a synthetic solar system model based on the \cite{Bottke2002} NEO orbital distribution. \label{fig:PHA_orbits}}
\end{figure}

\item{Optionally, define a size for each object. For populations where the size distribution is strongly tied to the orbital distribution, this is necessary. However, most small body populations can be well described by independent orbital and size distributions; the PHA population larger than 140~m in diameter can be generally described in this manner. In these cases, a smaller set of orbits can be used to represent the overall larger population; during analysis, each object can be `cloned' from a fiducial $H$ magnitude associated with the orbit to a range of $H$ magnitudes covering the range interesting for analysis. This makes the metric analysis, and particularly the generation of the expected observations for each object, simpler and faster. As long as sufficient resolution of the orbital parameter space is maintained, the metric results over the range of $H$ magnitudes will be comparable to the results calculated with a larger population. Here we use the small population of $>1$~km diameter known PHAs and clone them to a range of $H$ magnitudes between $H$=11 and $H$=28. We have verified with a larger, simulated set of NEOs that reducing the population from 10,000 model NEOs to 2000 model NEOs does not change the calculated survey  completeness. }

\item{Optionally, define a spectrum or color for each object. This facilitates the conversion from $H$ magnitude (assumed to be in $V$ band) to the apparent magnitude in a given LSST observation, which may be in any of $u$, $g$, $r$, $i$, $z$, or $y$ filters. Here we have assumed that our entire PHA population consists of C-type asteroids, with resulting transformations to  LSST bandpasses as described in Table~\ref{tab:sed_colors}.  }
\begin{deluxetable}{ccccccc}
\centering
\tablecolumns{7}
\tablecaption{Color transformations from Harris $V$ band to LSST bandpasses, for C and S type asteroids. \label{tab:sed_colors}}
\tablewidth{0.7\textwidth}
\tablehead{ Type & $V-u$ & $V-g$ & $V-r$ & $V-i$ & $V-z$ & $V-y$  \\ }
\startdata
C  & -1.53 &  -0.28 &  0.18 &  0.29 &  0.30 & 0.30 \\
S & -1.82 &  -0.37 &  0.26 & 0.46 &  0.40 & 0.407  \\
\enddata
\end{deluxetable}

\end{enumerate}

Using the details of the input population, MAF then generates the expected observations of each object using the pointing history from a specific OpSim simulated survey. Ephemerides are generated using OpenOrb \citep{OpenOrb2009} for a closely spaced grid of times, and then interpolated to the exact times of each OpSim pointing. If the object is within the LSST field of view, its predicted position, velocity, and apparent $V$ magnitude (for the fiducial $H$ magnitude associated with the orbit) is recorded along with information about the simulated observation itself (such as the seeing, limiting magnitude, filter, and boresight RA/Dec). The full LSST camera footprint, including chipgaps, is used to determine if an object is within the field of view. The camera footprint is shown in Figure~\ref{fig:camera_footprint}. 

\begin{figure}
\centering
\includegraphics[width=0.65\textwidth]{figures/focalplane} 
\caption{Model of the LSST camera footprint, including chipgaps and CCD + raft layout. \label{fig:camera_footprint}}
\end{figure}

Trailing loss estimates are provided by MAF. Trailing losses occur whenever the motion of a moving object spreads its photons over a wider area than a simple stellar PSF. There are two aspects
of trailing loss to consider: simple SNR losses and detection losses.
The first is simply the degradation in SNR that occurs (relative to a
stationary PSF) because the trailed object includes a larger number of
background pixels in its footprint. This will affect photometry and
astrometry, but typically doesn't directly affect whether an object is
detected or not. The second effect (detection loss) is not related to
measurement errors but does typically affect whether an object passes
a detection threshhold. Detection losses occur because source
detection software is optimized for detecting point sources;
a stellar PSF-like filter is used when identifying sources that pass
above the defined threshhold, but this filter is non-optimal for
trailed objects. This can be mitigated with improved software ({\it                                                                               
e.g.} detecting to a lower SNR threshhold and attempting to detect
sources using a variety of trailed PSF filters). Both trailing losses can
be fit as:
\begin{eqnarray}
\Delta \, m & = &-1.25 \, log_{10} \left( 1 + \frac{a \, x^2} { 1 + b\,
    x} \right) \\
x & = & \frac{v \, T_{exp}} {24 \, \theta} 
\end{eqnarray}
where $v$ is the velocity (in degrees/day), $T_{exp}$ is the exposure
time (in seconds), and $\theta$ is the FWHM (in arcseconds). For
SNR trailing losses, we find $a = 0.67$ and $b = 1.16$; for
detection losses, we find $a=0.42$ and $b=0$. An illustration of the
magnitude of these trailing loss effects for 0.7'' seeing is given in
Figure~\ref{trailinglosses}. When considering whether a source would
be detected at a given SNR using typical source detection software,
the detection loss should be used.


and 'fading'

explain cloning and analysis (differential completeness), and how combined to generate cumulative completeness


\subsection{OpSim Simulated Surveys}

describe set of new opsim runs (minion1016, astrolsst011016, astrolsst011015, astrolsst011017)


- a basic description of the baseline cadence (you/I can lift that from 
  Ch. 2 in observing strategy white paper) 
- description of modifications for other 3 simulations 
- a set of basic metrics for all 4 sims that describe “non-NEO” programs: 
   fOArea, fONv, median coadded ugrizy depth for WFD, median airmass 
   for WFD (all bands together), parallax normed for WFD; use 10 yrs
   for baseline and 12 years for other 3 
   
 \subsection{Evaluation of PHA discovery completeness}
 
- existing cumulative completeness plots for each survey, with
       - 3 pairs in 30 days, SNR=4, no PSF detection loss
       - 3 pairs in 30 days, SNR=5, total trailing loss
       - 3 pairs in 15 days, SNR=5, total trailing loss
and summarize completeness changes 
\begin{enumerate}
\item baseline (minion1016) to more observations in NES
\item adding two years to astrolsst011016
\item extending MOPS window to 30 days, at SNR=5
\item adding longer exposures throughout the ecliptic (astrolsst011015/astrolsst1017)
\item making DM find objects with trailing loss, not detection losses
\item extending SNR limit to 4 instead of 5
\end{enumerate}

\subsection{Comparison with external completeness analysis}

The leading systematic effects in completeness estimates are: 
\begin{enumerate}
\item NEO vs. PHA difference (the completeness is about $\sim$5\% higher for PHAs than for NEOs) 
\item Different sample definitions: $H<22$ vs. $D>140$m (as shown by \citep{GMS2016}, completeness
           increases by $\sim$5\% when $H$-based criterion is used) 
\item Variations of ``discovery window'' (e.g., X visit pairs in N nights: changing N from 15 to 30 with X=3 increases
          completeness by 3\%; changing X from 3 to 4 with N=15 decreases completeness by 6\%). 
\end{enumerate}          
          

\subsection{Remaining uncertainties in completeness estimates}

Common to all, really.          

\begin{enumerate}
\item Sensitivity to details in sky coverage and cadence (e.g. nightly pairs of visits vs. quads of visits;
          requiring quads instead of pairs of visits decreases completeness by 30\% using baseline cadence; 
          about half of that loss can be recovered using cadence simulations that request four visits per night) 
\item Uncertainties when predicting effective image depth (system throughput, variation of the detection efficiency
          with the signal-to-noise ratio, treatment of trailing losses); for a survey that has a completeness above 60\%, 
          each additional one magnitude of depth for a given survey cadence increases the completeness by another 10\%.
\item Uncertainties when predicting apparent flux (albedo distribution, phase effects, photometric variability 
          due to non-spherical shapes, color distributions); assuming an uncertainty of 0.2 mag in the effective 
          limiting magnitude, the corresponding  systematic uncertainty in completeness is about 2\%.)
\item The slope of the asteroid size distribution (current measurement uncertainty of this parameter 
          corresponds to a systematic uncertainty in completeness of about 2\%.)
\item The impact of known objects (assuming that 43\% objects would be discovered by the start of
          LSST survey, \citep{GMS2016} boosted the final PHA completeness for LSST baseline survey by 11\%). 
\end{enumerate} 

Given these systematic effects, a comparison of different simulation results (both for the same system,
and those of different systems, especially systems operating at different wavelengths) has to be undertaken
with due care. It is unlikely that a meaningful quantitative comparison can be pushed beyond a level
of a few percent (and perhaps as much as 10\%). In practice, the completeness of a given operating survey
is best estimated using the object re-discovery rate. 




\section{Conclusions}

{\bf Summary:} Well behaved image differencing and detection are needed for the
asteroid detection strategy adopted by LSST. This has historically
been difficult to achieve. First generation surveys have encountered
significant problems including CCD artifacts, optical system
artifacts, and software issues. Significant progress has been made
since. Contemporary surveys comparable to LSST (specifically, DES) are
already achieving false positive rates below the few:1 ratio required
for LSST MOPS (see next section). LSST image differencing will {\bf
not} be a limiting factor in its ability to discover asteroids
(including NEOs). DES experience has already demonstrated algorithms 
sufficient to meet LSST MOPS requirements.

\begin{enumerate}
\item LSST will employ the 2+2+2 MOPS discovery strategy, and detect fast-moving asteroids via trailing.
\item Existing surveys demonstrate that false positive rates required
  by LSST (approaching $\sim$1:1) are  achievable (Dark Energy Survey;  \citep{goldstein2015}).
\item PanSTARRS PS4 simulations \citep{denneau13}) as well as LSST
  simulations (Myers et al. 2011) demonstrate the ability of MOPS to perform the linkages under those conditions.
\item LSST software and observing strategy are robust to perturbations
  around assumed efficiencies; even large differences in expectation
  cause only a few percent difference in efficiency. 
\item Our simulations indicate that, if the cadence is optimized for
          NEO searches, LSST likely has the capability and capacity to reach the Brown mandate.
\end{enumerate}

Add words about consistency with \cite{GMS2016}. 

 
\appendix
%\section{LSST Image Processing Steps and Data Products Relevant for Asteroids} \label{sec:AppA}
\section{LSST Image Processing Steps and Data Products Relevant for Asteroids} \label{sec:AppA}

The data products produced by the LSST Data Management system are described in
LSST Document LSE-163\footnote{See \url{http://ls.st/LSE-163}} (LSST Data Products
Definition Document). Here we briefly summarize parts of that
document\footnote{To ensure the continued scientific adequacy of LSST data
products, their designs and plans are periodically reviewed and updated and
thus LSE-163 is a living document -- please always consult the latest version.}
that are most relevant for discovering moving Solar System objects.

The LSST Data Management system will perform nightly analysis of difference images\footnote{A difference
image is an image produced by subtracting a science image from an appropriate
``average'' of the previously collected similar images of the same sky area, and using the
same filter.}, with the goal of detecting and characterizing astrophysical phenomena
revealed by their time-dependent nature. The detection of supernovae superimposed
on bright extended galaxies is an example of this analysis, and of course moving Solar
System objects are another example. The processing will be done on nightly/daily
basis and will result in the so-called Level 1 data products. Level 1 products will include
difference images, catalogs of sources detected in difference images (the so-called
\DIASources), static astrophysical objects\footnote{The LSST has adopted the nomenclature by
which single-epoch detections of astrophysical {\em objects} are called {\em sources}.
The reader is cautioned that this nomenclature is not universal: some surveys call
{\em detections} what LSST calls {\em sources}, and use the term {\em sources} for what
LSST calls {\em objects}.} these \DIASources are positionally associated to (the so-called \DIAObjects),
and moving Solar System objects (\SSObjects\footnote{\SSObjects used to be called
``Moving Objects'' in previous versions of the LSST Data Products baseline. The name is
potentially confusing as high-proper motion stars are moving objects as well. A more
accurate distinction is the one between objects {\em inside} and {\em outside} of the Solar
System.}). The catalogs will be entered into the Level 1 database and made available in near
real time. Notifications (``alerts'') about new \DIASources will be issued using
community-accepted standards\footnote{For example, VOEvent, see \url{http://ls.st/4tt}} within
60 seconds of observation.

The Moving Object Processing Software (\code{MOPS}) pipeline combines all unassociated \DIASources into
plausible \SSObjects and estimates their orbital parameters. The three main pipeline stages
include associating new \DIASources with known \SSObjects, discovering new \SSObjects,
and orbit refinement and management. This conceptual MOPS design is illustrated in
Figure~\ref{fig:Pipe8}. Further details about the MOPS pipeline design and implementation are available
from the LSST Science Pipelines Design Document\footnote{See \url{http://ls.st/LDM-151}}, LDM-151.
The next section briefly describes the main processing steps in nightly/daily Level 1 data processing.

\begin{figure}[!t]
    \centering
    %\vskip -2.3in
%    \hskip -0.2in
    \includegraphics[scale=0.60, angle=270]{MOPS-Level0}
    \vskip -0.1in
    \caption{Illustration of the conceptual algorithm design for the Moving Object Processing Software.
   \DIASources are data structures that describe detections of sources in difference images and
   \SSObjects are data structures that describe discovered Solar System objects (see Table~\ref{tab:SSObj}).
\label{fig:Pipe8}}
\end{figure}


\subsection{LSST Level 1 Data Processing}

Level 1 data products are a result of difference image analysis (DIA).
\DIASources are sources detected on difference images with the signal-to-noise ratio $S/N>transSNR$,
with $transSNR$=5.
They represent changes in flux with respect to a deep template. Physically, a \DIASource may be an observation of new astrophysical object that was not present at that position in the template image (for example, an asteroid), or an observation of flux change in an existing source (for example, a variable star). Their flux can be negative (eg., if a source present in the template image reduced its brightness, or moved away). Their shape can be complex (eg., trailed, for a source with proper motion approaching $\sim {\rm deg}/{\rm day}$, or ``dipole-like", if an object's observed position exhibits an offset -- true or apparent -- compared to its position on the template).
Some \DIASources will be caused by background fluctuations; with $transSNR = 5$,
the expected false positive rate is about three per CCD ($\sim 60$ per sq. deg.) for the median seeing,
or of the order 500,000 per typical night.
The expected number of false positives due to background fluctuations is a very strong function
of adopted $transSNR$: a change of $transSNR$ by 0.5
results in a variation of an order of magnitude, and a change of $transSNR$ by unity changes the number of false
positives by about two orders of magnitude.

Clusters of \DIASources detected on visits taken at different times are associated with either a \DIAObject or an \SSObject, to represent the underlying astrophysical phenomenon. The association can be made in two different ways: by assuming the underlying phenomenon is an object within the Solar System moving on an orbit around the Sun\footnote{LSST pipelines will not fit for motion around other Solar System bodies; eg., identifying new satellites of Jupiter is left to the community.}, or by assuming it to be distant enough to only exhibit small parallactic and proper motion\footnote{Where ``small'' is small enough to unambiguously positionally associate together individual apparitions of the object.}. The latter type of association is performed during difference image analysis right after the image has been acquired. The former is done at daytime by \code{MOPS}, unless the \DIASource is an apparition of an already known \SSObject. In that case, it will be flagged as such during difference image analysis. At the end of the difference image analysis of each visit, LSST will issue time domain event alerts for all
newly detected \DIASources\footnote{For observations on the Ecliptic near the opposition Solar System objects will dominate the \DIASource counts and (until they're recognized as such) overwhelm the explosive transient signal. It will therefore be advantageous to quickly identify the majority of Solar System objects early in the survey.}.

\subsubsection{Nightly Difference Image Processing \label{sec:ssProcessing}}

The following is a high-level description of steps which will occur during regular {\em nightly}
difference image analysis:
\begin{enumerate}
\item A visit is acquired and reduced to a single {\em visit image} (cosmic ray rejection, instrumental signature removal\footnote{Eg., subtraction of bias and dark frames, flat fielding, bad pixel/column interpolation, etc.}, etc.).
\item The visit image is differenced against the appropriate template and \DIASources are detected and
their properties measured.
\item The flux and shape\footnote{The ``shape'' in this context consists of weighted 2$^{\rm nd}$ moments
of the intensity distribution, as well as fits to a trailed source model and a dipole model.} of the DIASource are measured on the difference image. PSF photometry is performed on the visit image at the position of the \DIASource to obtain a measure of the total flux.
\item The Level 1 database is searched for a \DIAObject or an \SSObject with which to positionally associate the newly discovered \DIASource\footnote{The association algorithm will guarantee that a \DIASource is associated with not more than one existing \DIAObject or \SSObject. The algorithm will take into account the parallax and proper (or Keplerian) motions, as well as the errors in estimated positions of \DIAObject, \SSObject, and \DIASource, to find the maximally likely match. Multiple \DIASources in the same visit will not be matched to the same \DIAObject.}. If no match is found, a new \DIAObject is created and the observed \DIASource is associated to it.
\item If the \DIASource has been associated with an \SSObject (a known Solar System object), it will be flagged as such and an alert will be issued. Further processing will occur in daytime (see \S\ref{sec:ssProcessing} below).
\item Otherwise, the associated \DIAObject measurements will be updated with new data
collected during previous 12 months. All affected columns will be recomputed, including proper motions, centroids, light curves, etc.
\item The \DR\footnote{\DR is a database resulting from annual data release processing.} is searched for one or more \Objects positionally close to the \DIAObject, out to some maximum radius\footnote{Eg., a few arcseconds.}. The IDs of these nearest-neighbor \Objects are recorded in the \DIAObject record and provided in the issued
event alert.
\item An alert is issued that includes the \DIASource ID, the \SSObject ID or \DIAObject ID, and the associated science content (centroid, fluxes, low-order lightcurve moments, periods, etc.), including the full light curves.
\item For all \DIAObjects overlapping the field of view, including those that have an associated
new \DIASource from this visit, forced photometry will be performed on difference image (point source photometry only).
No alerts will be issued for these \DIASources.
\item Within 24 hours of discovery, {\em precovery} PSF forced photometry will be performed on any difference image overlapping the position of new \DIAObjects taken within the past 30 days, and added to the database. Alerts will not be issued with precovery photometry information.
\end{enumerate}

In addition to the processing described above, a smaller sample of sources detected on difference images {\em below} the nominal $transSNR = \transSNR$ threshold will be measured and stored, in order to enable monitoring of difference image analysis quality.

Also, the system will have the ability to measure and alert on a limited\footnote{It will be sized for no less than $\sim 10\%$ of average \DIASource per visit rate.} number of sources detected below the nominal threshold for which additional criteria are satisfied. For example, a $transSNR = 3$ source detection near a gravitational keyhole\footnote{
A gravitational keyhole is a region of space where Earth's gravity would modify the orbit of a passing asteroid
such that the asteroid would collide with the Earth in the future.}
may be highly significant in assessing the danger posed by a potentially hazardous asteroid.
The initial set of criteria will be defined by the start of LSST operations.

\subsubsection{Solar System Object Processing \label{sec:ssProcessing}}

The following will occur during regular Solar System object processing in daytime\footnote{Note that there {\em is no strict bound on when daytime Solar System processing must finish}, just that, averaged over some reasonable timescale (eg., a month), a night's worth of observing is processed within 24 hours. Nights rich in moving objects may take longer to process, while nights with less will finish more quickly. In other words, the system requirement is on {\em throughput}, not latency.}, after a night of observing (see Figure~\ref{fig:Pipe8}):
\begin{enumerate}
\item The orbits and physical properties of all \SSObjects re-observed on the previous night are recomputed. External orbit catalogs (or observations) are also used to improve orbit estimates. Updated data are entered to the \SSObjects table.
\item All \DIASources detected on the previous night, that have {\em not} been matched at a high confidence level to a known \Object,
\DIAObject, \SSObject, or an artifact, are analyzed for potential pairs, forming {\em tracklets}.
\item The collection of tracklets collected over the past 30 days is searched for subsets forming {\em tracks} consistent with being on the same Keplerian orbit around the Sun.
\item For those that are, an orbit is fitted and a new \SSObject table entry created. \DIASource records are updated to point to the new \SSObject record. \DIAObjects ``orphaned'' by this unlinking are deleted.\footnote{Some \DIAObjects may only be left with forced photometry measurements at their location (since all \DIAObjects are force-photometered on previous and subsequent visits);  these will be kept but flagged as such.}.
\item Precovery linking is attempted for all \SSObjects whose orbits were updated in this process. Where successful, \SSObjects (orbits) are recomputed as needed.
\end{enumerate}


\subsubsection{Level 1 Catalogs}
\label{sec:level1db}

The described alert processing design relies on the ``living'' \DB that contains the objects and sources detected on difference images. At the very least\footnote{It will also contain exposure and visit metadata, MOPS-specific tables, etc.}, this database will have tables of \DIASources, \DIAObjects, and \SSObjects, populated in the course of nightly and daily difference image and Solar System object processing\footnote{The latter is also colloquially known as {\em DayMOPS}.}. As these get updated and added to, their updated contents becomes visible (query-able) immediately\footnote{No later than the moment of issuance of any event alert that may refer to it.}.

Table~\ref{tab:SSObj}  presents the {\em conceptual schema} for the \SSObject table (it conveys {\em what} data
will be recorded in each table, rather than the details of {\em how}).
Columns whose type is an array will likely be expanded to one table column per element of the array
once this schema is translated to SQL\footnote{The SQL realization of this schema can be browsed at \url{http://ls.st/8g4}}. In addition, the table presented here is normalized (i.e., it contains no redundant
information with other tables in Level 1 database). For example, since the band of observation can be found
by joining a \DIASource table to the table with exposure metadata, there's no column named {\tt band} in the \DIASource table. In the as-built database, the views presented to the users will be appropriately denormalized
for ease of use.

\subsubsection{\SSObject Table}

\begin{center}
\label{tab:SSObj}
\begin{longtable}{p{3cm}p{2cm}p{2cm}p{5cm}}
%\caption[\SSObject Table]{\SSObject Table} \\

\hline \multicolumn{1}{c}{\bf Name} & \multicolumn{1}{c}{\bf Type} & \multicolumn{1}{c}{\bf Unit} & \multicolumn{1}{c}{\bf Description} \\ \hline
\endhead

\hline \multicolumn{4}{r}{{\em Continued on next page}} \\
\endfoot

\hline\hline
\endlastfoot

ssObjectId & uint64 & ~ & Unique identifier. \\

oe & double[7] & various & Osculating orbital elements at epoch ($q$, $e$, $i$, $\Omega$, $\omega$, $M_0$, epoch). \\

oeCov & double[21] & various & Covariance matrix for \texttt{oe}. \\

arc & float & days & Arc of observation. \\

orbFitLnL & float & ~ & Natural log of the likelihood of the orbital elements fit. \\

orbFitChi2 & float & ~ & $\chi^2$ statistic of the orbital elements fit. \\

orbFitNdata & int & ~ & The number of data points (observations) used to fit the orbital elements. \\

MOID & float[2] & AU & Minimum orbit intersection distances\footnote{\url{http://www2.lowell.edu/users/elgb/moid.html}} \\

moidLon & double[2] & degrees & MOID longitudes. \\

H & float[6] & mag & Mean absolute magnitude, per band (Muinonen et al. 2010 magnitude-phase system). \\

${\rm G_1}$ & float[6] & mag & $G_1$ slope parameter, per band (Muinonen et al. 2010 magnitude-phase system). \\

${\rm G_2}$ & float[6] & mag & $G_2$ slope parameter, per band (Muinonen et al. 2010 magnitude-phase system). \\

hErr & float[6] & mag & Uncertainty of H estimate.\\

g1Err & float[6] & mag & Uncertainty of $G_1$ estimate. \\

g2Err & float[6] & mag & Uncertainty of $G_2$ estimate. \\

flags & bit[64] & bit & Various useful flags. \\ \hline

\end{longtable}
\end{center}

The $G_1$ and $G_2$ parameters for the large majority of asteroids will not be well constrained until later in the survey. LSST may decide not to fit for it at all over the first few DRs and add it later in Operations, or provide two-parameter $G_{12}$ fits. Alternatively, they may be fitted using strong priors on slopes poorly constrained by the data. The design of the data management system is insensitive to this decision, making it possible to postpone it to Commissioning to ensure it follows the standard community practice at that time.
The LSST database will provide functions to compute the phase (Sun-Asteroid-Earth) angle $\alpha$ for every observation, as well as the reduced, $H(\alpha)$, and absolute, $H$, asteroid magnitudes in LSST bands.
 

\bibliography{neo_capabilities}
\end{document}

To Do:
- ask Lori Allen to be coauthor (and if there are others from her team) 
- ask Tim Axelrod and Jonathan Myers (others from old days?) 

