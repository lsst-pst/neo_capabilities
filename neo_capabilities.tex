\documentclass[12pt,preprint]{aastex}
\usepackage{lsst}
\usepackage{xspace}
\usepackage[english]{babel}
\usepackage[utf8x]{inputenc}
\usepackage{hyperref}
\usepackage{amsmath}
\usepackage{graphicx}
\usepackage{longtable}
\usepackage{comment}
\usepackage{booktabs}

\newcommand{\Alert}{\code{Alert}\xspace}
\newcommand{\Alerts}{\code{Alerts}\xspace}
\newcommand{\DIASource}{\code{DIASource}\xspace}
\newcommand{\DIASources}{\code{DIASources}\xspace}
\newcommand{\DIAObject}{\code{DIAObject}\xspace}
\newcommand{\DIAObjects}{\code{DIAObjects}\xspace}
\newcommand{\DB}{{Level 1 database}\xspace}
\newcommand{\DR}{{Level 2 database}\xspace}
\newcommand{\Object}{\code{Object}\xspace}
\newcommand{\Objects}{\code{Objects}\xspace}
\newcommand{\Source}{\code{Source}\xspace}
\newcommand{\Sources}{\code{Sources}\xspace}
\newcommand{\ForcedSource}{\code{ForcedSource}\xspace}
\newcommand{\ForcedSources}{\code{ForcedSources}\xspace}
\newcommand{\CoaddSource}{\code{CoaddSource}\xspace}
\newcommand{\CoaddSources}{\code{CoaddSources}\xspace}
\newcommand{\SSObject}{\code{SSObject}\xspace}
\newcommand{\SSObjects}{\code{SSObjects}\xspace}
\newcommand{\VOEvent}{\code{VOEvent}\xspace}
\newcommand{\VOEvents}{\code{VOEvents}\xspace}
\newcommand{\transSNR}{5\xspace}


\begin{document}
\title{The Large Synoptic Survey Telescope as a Near-Earth Object Discovery Machine (version 1.0)}

\author{R. Lynne Jones\altaffilmark{1},
Colin T. Slater\altaffilmark{1},
Joachim Moeyens\altaffilmark{1},
Lori Allen\altaffilmark{2},
Mario Juri\'{c}\altaffilmark{1},
\v{Z}eljko Ivezi\'{c}\altaffilmark{1},
Tim Axelrod\altaffilmark{3},
Jonathan Myers
}
\altaffiltext{1}{Department of Astronomy, University of Washington, Box 351580,
  57 Seattle, WA 98195, USA}
\altaffiltext{3}{National Optical Astronomy Observatory, 950 North Cherry
Avenue, Tucson, AZ 85719, USA}
\altaffiltext{2}{University of Arizona, Steward Observatory, 933 North Cherry
Avenue, Tucson, AZ 85721, USA}

\begin{abstract}
We investigate the capabilities of LSST to discover Near-Earth Objects (NEOs), fast-moving objects
which are challenging to track and link, and discuss the expected yield of NEO discoveries
resulting from variations on the survey strategy. The typical LSST cadence uses pairs of observations
over a series of nights to discover NEOs; we test this data processing strategy by empirically determining 
an expected false positive detection rate and validating that this rate is tractable with the Moving Object
Processing Software (MOPS).  Using a high-fidelity simulated survey pointing history, we evaluate the 
performance of the LSST baseline survey strategy for Potentially Hazardous Asteroids (PHAs) and 
find an expected completeness of 68\% for PHAs with $H<22$ from LSST alone. We also generate variations
on the baseline survey strategy to evaluate potential methods to increase the PHA completeness. For example,
extending the LSST survey by two additional years and including previously known objects increases the 
completeness for PHAs to XX. 
\end{abstract}

\keywords{Near-Earth objects --- Image processing -- Asteroids}


\section{Introduction}

XXX This is so totally work in progress. ZI push-ed it just as a backup... XXX

Main parts:
\begin{itemize}
\item NEO impacts as a concern; the Brown mandate, NASA panels, introduce LSST 
\item LSST defaults and claimed performance in older publications
\item informal concerns by the community, GMS paper
\item study by the JPL team, technical support from LSST (assuming baseline cadence)
\item exploration of baseline cadence modifications designed to boost NEO completeness 
\item Outline of this paper
\end{itemize} 



The small-body populations in the Solar System, such as asteroids, trans-Neptunian objects (TNOs) 
and comets, are remnants of its early assembly. Collisions in the main asteroid belt between Mars and 
Jupiter still occur, and occasionally eject objects on orbits that may place them on a collision course 
with Earth. About 20\% of this near-Earth Object (NEO) population, the so-called potentially hazardous 
asteroids (PHAs), are in orbits that pass sufficiently close to Earth's orbit, to within 0.05 AU, that 
perturbations with time scales of a century can lead to intersections and the possibility of collision. 
In December 2005, the U.S. Congress directed\footnote{For details see http://neo.jpl.nasa.gov/neo/report2007.html} 
NASA to implement a NEO survey that would catalog 90\% of NEOs with diameters larger than 140 meters 
by 2020 (the George E. Brown, Jr. mandate). For a compendium of information about NEOs and PHAs 
and an up-to-date summary of discovery progress, see NASA's NEO webpage\footnote{http://neo.jpl.nasa.gov/neo/}. 

The completeness level set by Congressional mandate can be fulfilled with a 10-meter-class ground-based
telescope equipped with a multi-gigapixel camera, and a sophisticated and robust data processing system. 
The Large Synoptic Survey Telescope (LSST), currently being constructed, is such a system (for a concise
system description, science drivers and other information, see \citep{LSSToverview}). Early simulations of 
LSST performance presented by \cite{IvezicNEO2007} showed that the 10-year baseline cadence would 
result in 75\% completeness for PHAs greater than 140 m (more precisely, for PHAs with $H<22$; see 
\S~XX for further discussion). They also suggested that with additional optimization of the observing cadence, 
LSST could achieve 90\% completeness. Such optimization was discussed by \cite{LSSToverview} who
reported that, to reach 90\% completeness, about 15\% of observing time would have to be dedicated to NEOs
and the survey would have to run for 12 years.  
%% From the overview paper: 
%% - the LSST baseline cadence provides orbits for 82% of PHAs larger than 140 meters after 10 years of operations
%% - 84% completeness with minor changes to the cadence (5% of time for NEO-optimized observations)
%% - 90% completeness with major changes to the cadence (15% of time for NEO-optimized observations and 12 years)
The latest LSST simulation results, presented in \cite{JJI2016}, yielded a completeness of $\sim$72\% for
PHAs with $H<22$ using the current 10-year baseline survey. The minor difference compared to older
studies is attributable to the differences in simulated NEO populations and other modeling details. 


\cite{JPLstudy} described a new study, related to this work. Their preliminary results indicate a completeness
of $\sim$65\% for NEOs with $H<22$. The difference compared to \cite{JJI2016} result (72\%) is due 
to XXX (true?): PHA vs. NEO, possible slope of the size distribution. 


We're going to evaluate moving object detection capabilities with
LSST, comparing performance of our current prototype pipelines with 
the required performance during operations.  The two main goals are
to demonstrate that i) MOPS can cope with false detection in image differences,
and that ii) the NEO detection performance of the LSST baseline cadence can be
further boosted by adequate modifications 


From JPL paper -- which gives a concise summary of the main simulation aspects. 

The LSST baseline survey cadence relies primarily on single night pairs of detections, 
with roughly 30 minutes between the elements of a detection pair. These pairs form 
what are known in MOPS parlance as tracklets, and sets of tracklets are linked across 
multiple nights to form tracks, which can then be sent to the final step, which is orbit 
determination. The strategy of using pairs is an aggressive and potentially fragile
approach, but theoretically represents the most productive NEO search with the minimum 
impact on other LSST science drivers. An alternative to visit each field three times per 
night to form tracklets from triplets of detections may prove more robust, but likely 
with a penalty of reduced performance. One of our study objectives is to understand the
tradeoffs between these two approaches.

The two major questions to be addressed by our study can be informally stated as 
``Will MOPS work?'' and ``If MOPS works what fraction of NEOs will LSST discover?''. 

Main problems:
\begin{enumerate}
\item Linking large number of detections in the presence of false positives (false detections due to problems 
with image differencing software). 
\item Adequacy of data, including image depth, sky coverage and cadence, to reach the required 
completeness level. 
\end{enumerate} 

Therefore, the main analysis components to check are: 
\begin{enumerate}
\item The performance of image differencing, with emphasis on the rate and properties of 
   false detections 
\item Linking large number of detections in the presence of false positives 
\item Observing cadence simulations coupled with NEO population models to forecast 
        discovery rates 
\end{enumerate} 



\cite[hereafter GMS]{GMS2016} reported different NEO completeness levels than
published by the LSST team in 2007 and 2014 . There are three main 
reasons why the GMS results differ:
\begin{enumerate}
\item GMS used a different realization of the LSST baseline survey
\item GMS used different synthetic NEO populations to evaluate completeness
\item GMS {\it redefined} the completeness limit from the commonly
  used $H<22$ criterion to an albedo-dependent value of $H$ limit (which
  attempts to directly model the $D>140$ m size cut)
\end{enumerate}

Regarding the last point, GMS found that the completeness drops by 5\%
when $H<22$ criterion is replaced by $D>140$ m criterion. Regarding 
the first point, GMS results can be more meaningfully compared to an LSST
study by \cite{JJI2016}, who used the same simulated cadence. After accounting for 
the $H<22$ vs. $D>140$ m methodological difference of 5\%, GMS obtained a 
completeness of 67\% using 3 pairs in 12 nights (for simulated cadence {\it enigma\_1189}), while Jones et al. 
study obtained $\sim$73\% using 3 pairs in 15 nights (for simulated cadence {\it minion\_1016}, which is 
statistically very similar to {\it enigma\_1189}). This difference 
of $\sim$5\% is attributable to the differences in simulated NEO
populations and other modeling details. In summary, GMS find the NEO
completeness in the range $\sim$60\% to $\sim$70\% for the LSST
baseline cadence. The variation is due to different NEO populations,
different NEO detection criteria, and other specifics. When accounting 
for different choices of simulation parameters, their results are
consistent with the results published by the LSST team. 

But: how much higher can the completeness be pushed with cadence
modifications optimized for NEOs? 

In Observing Strategy white paper: fig. 3.4 gives 73.4\% for PHAs with 
$H<22$ and {\it minion\_1016}, using 3 pairs in 15 nights. 


\begin{enumerate} 
\item Control and quantify the rate of (false positive) detections
\item Software (and computational capacity) capable of inter-night linking of detections given the expected rates
\item Quantify the discovery yields (and their robustness) under those assumptions
\end{enumerate} 


From Mario's talk to NASA:

LSST will detect variability (motion and flux variability) by
differencing each incoming image against a deep template.
Sources will be detected at an S/N=5 threshold (see Appendix A). 

We expect on average about 1,000 per sq. deg. astrophysical, real,
detections, including up to about 500 asteroids per sq. deg on the 
Ecliptic (for scale, the LSST field of view is about 10 sq. deg., with 
about 20 4kx4k CCDs per sq. deg.)

We also expect a false-positive detections due to random
fluctuations in the background at a level of about 200 per
sq. deg. (all at the faint end).  XXX check Colin's report 

However, historically surveys have reported factors of 10 to 500 times
more (depending on the survey; see \citep{denneau13};
\citep{goldstein15} ). 
Those additional false positive
detections are due to systematic effects: 
\begin{itemize} 
\item Camera and telescope artifacts
\item Imperfect image subtractions
\item Cosmic rays
\end{itemize} 

For a ``menagerie'' of artifacts (with amusing names such as 
{\it chocolate chip cookies, frisbee, piano, arrowhead, UFO}), from
Pan-STARRS, see Fig.~17 in \cite{denneau13}. 


``Many of the false detections are easily explained as internal
reflections, ghosts, or other well-understood image artifacts,...''


Learning from PS1 Experience: PanSTARRS was a first generation
experiment. Over the past decade, subsequent surveys (including LSST) 
have learned tremendously from the PS1 experience. There are surveys 
running today which have largely solved the key problems that PS1 has encountered.
These are recent developments driven largely by extragalactic and
transient science cases. They are not yet well known beyond those
communities and reporting on those developments is additional
motivation for this paper. 
 
Major improvements to hardware include CCDs with significantly fewer 
artifacts (e.g. DECam, see below; LSST) and optical systems designed to
minimize ghosting and internal reflections (e.g. LSST). 

Improvements to the software include advanced image differencing
pipelines (e.g., PTFIDE for the Palomar Transient Factory and the
Zwicky Transient Facility) and various machine learning classifiers
for filtering false positives (see below). 

DECam: \cite{goldstein15} 

The Dark Energy Survey (DES) is an optical/near-infrared survey that
aims to probe the dynamics of the expansion of the universe and the
growth of large scale structure by imaging 5,000 sq. deg. of the
southern sky. It is technologically very similar to LSST with
\begin{itemize}
\item 520 Mpix camera, 62 mosaicked chips (deep depleted devices)
\item 3 sq.deg. field of view, same filter bands as LSST
\item Single-exposure depths comparable to LSST
\item Includes a supernova search program which employs image
differencing methods analogous to LSST’s  and detects objects at the 
same effective signal-to-noise ratio as LSST (S/N=5)
\end{itemize} 

The false positives in DECam data are morphologically much simpler
(compare Fig.~1 in \citep{goldstein15} to Fig.~17 in \citep{denneau13})
than those in Pan-STARRS, and thus are much more amenable to automated 
screening using machine learning methods. Using a Random Forest 
classifier, \cite{goldstein15} cleaned their sample from having a 
raw false detection rate of 13:1 to a filtered rate of 1:3. This performance
is already within the acceptable range for LSST performance goals. 


{\bf Need to refer to section by Colin.} 

LSST will use two methods to detect moving objects
\begin{enumerate}
\item Detecting trailed motion on the sky:  objects trailed by more
  than 2 PSF widths (corresponding to motion faster than about 1
  deg/day) will be easily detectable as trailed.  Two trailed
  detections within 30--60 minutes in a single night will be
  sufficient to identify an object as an NEO candidate,
\item Inter-night linking of pairs: this technique will recover
  objects moving too slow enough to be measurably elongated in 
  a single exposure. 
\end{enumerate} 

MOPS: Given the expected false-positive rates demonstrated by
\cite{goldstein15} and in Colin's  section, LSST MOPS linking will
be possible. This has already been shown by the PanSTARRS project 
with simulations performed for the PS4 system\footnote{PanSTARRS 
PS1 experience does not contradict this conclusion. In addition to 
hardware issues, PS1 was only 1/4 of the assumed system (see 
\citep{denneau13} for more details). }, which is in this
respect equivalent to LSST. The robustness to unexpected false
positives is further tested with simulations performed by LSST,
as described below. 
   






\section{LSST Strategy for Discovering Solar System Objects} 

Opening paragraph (lift text from overview and Lynne's IAU paper) 

Possible subsections here or below:

- Concepts for discovering moving objects

- Outline for simulations (see Chesley) 


\subsection{Brief Overview of LSST and LSST Survey Design} 
%\subsection{Brief Overview of LSST} 

LSST will be a large, wide-field ground-based optical telescope system
designed to obtain multiple images covering the sky that is visible
from Cerro Pach\'{o}n in Northern Chile. The current baseline design,
with an 8.4m (6.7m effective) primary mirror, a 9.6 deg$^2$ field of
view, and a 3.2 Gigapixel camera, will allow about 10,000 square
degrees of sky to be covered every night, with typical 5$\sigma$ depth 
for point sources of $r\sim24.5$ (AB). The system is designed to yield 
high image quality (the median delivered seeing in the $r$ band of 
about 0.8 arcsec) as well as superb astrometric  and photometric 
accuracy\footnote{For detailed specifications, please see the LSST
Science Requirements Document, http://ls.st/srd}. The total survey
area will include $\sim$30,000 deg$^2$ with $\delta<+34.5^\circ$, and 
will be imaged multiple times in six bands, $ugrizy$, covering the 
wavelength range 320--1050 nm. For a more detailed, but still concise,
summary of LSST, please see 
the LSST Overview paper\footnote{arXiv:0805.2366, http://ls.st/2m9}. 

The project is scheduled to  begin the regular survey operations at
the start of next decade. About 90\% of the observing time will be
devoted to a deep-wide-fast survey mode which will uniformly observe 
a 18,000 deg$^2$ region about 1000 times (summed over all six bands) 
during the anticipated 10 years of operations, and yield a coadded map 
to $r\sim27.5$. These data will result in catalogs including about
$40$ billion stars and galaxies, that will serve the majority of the
primary science programs. The remaining 10\% of the observing time
will be allocated to special projects such as a Very Deep and Fast
time domain survey\footnote{Informally known as ``Deep Drilling Fields".}.

The LSST will be operated in fully automated survey mode. The images
acquired by the LSST Camera will be processed by LSST Data Management
software \cite{juric15} to a) detect and characterize imaged
astrophysical sources and b) detect and characterize temporal changes
in the LSST-observed universe. The results of that processing will be
reduced images, catalogs of detected objects and the measurements of 
their properties, and prompt alerts to ``events'' -- changes in
astrophysical scenery discovered by differencing incoming images
against older, deeper, images of the sky in the same direction ({\em
emplates}, see \S \ref{sec:AppA} for more details). 
 


\subsection{LSST Observing Strategy} 

XXX get text from the overview paper and Science Book 

As deployed and funded (by the U.S National Science Foundation and
Department of Energy), LSST is primarily a science-driven mission. 
Existing cadence is optimized to maximize the overall science returns
(incl. Solar System science), rather than NEO/PHA discovery
completeness (though the two goals are highly interrelated).  As designed, the survey is not optimized for rapid
discovery and follow-up of all types of moving objects\footnote{
XXX What's the purpose of this footnote? Note that LSST will enable rapid identification and follow-up of
trailed objects (within 60 seconds of discovery). If deployed with a 
planetary-defense optimized cadence, the NEO yields could be
significantly improved, and approaching the 90\% completeness level
for $H<22$.} 
Early simulations indicate 90\% is achievable for NEO-optimized
cadence. However, other science goals would be affected (including
Solar System science!).  XXX Refer to Jones et al.  (2016) 

The current baseline cadence is optimized for science returns.
It is expected to yield approximately $\sim$70\% of the extant NEO population.


\subsection{Overview of LSST  Data Management and Image Processing} 

Refer to \cite{DM2016} and Appendix A. 



\subsection{Intro and Test Data}

False positives have historically dominated over true transients in previous
surveys. \citep{goldstein15, kessler15}

Major impediment to linking detections of moving objects

Develop a conservative estimate of the FP rate in the LSST stack, using similar
sensors and the current software.

Data from Allen et al., offers a useful cadence. Not the same depth or seeing,
but sufficient to test the software capabilities. Stack has not been optimized
to run at production-levels yet, so this is the starting point of that effort,
not an upper limit on LSST's performance.

\subsection{Processing}

Some reference to LSST software heritage and version, circa Jan 2016.

Differencing two images, rather than a template vs image. Running force
photometry. Ingest into DB.

\subsection{Correlated Noise}

Histogram of stack-reported diaSource SNR, alongside force photometry SNR.

Because we are diffing two images, we know that the photometry of the diaSource
must be the same as the difference of the force photometry on the direct images.
Gives us independent check on SNR reporting.

Strong ramp up of sources at SNR 5-6, but order of magnitude more than should be
present due to Gaussian statistics.

Check with force photometry shows that many of these cannot actually have the
high SNRs reported, more like 4-sigma events. FP values consistent with Gaussian
statistics.

Issue is tracking correlated noise after PSF-matching the science image. Known
issue, ref Paul \& Gene somehow (must be more references too).

For our purposes, we will estimate SNR using force phot. Several choices of long
term solutions; implementations are being studied now.

\subsection{False-Positive Results}

Using corrected SNR cut, ~1000 diaSources per square degree (only positive,
ignoring negative). ~500 on best fields.

Many of these are poorly-subtracted stars. To estimate rates of candidate moving
objects, exclude all diaSources that had detections in both direct images (at
the same position). These are still FPs, but they are someone else's FPs.
Resulting levels are ~350 per sq deg. Irreducable noise level is 33/sq deg.

Results from trying to make tracklets, hopefully?? On visual inspection, 25\%
look like junk, 25\% looked like good detections, and the rest were ambiguous
because low SNR. SNR power law exponent is $\sim -2.5$.

Very few detections around bright stars; mostly well-masked by the code. Few
large scale detected artifacts, but not a very big sample of images.

How detailed of a recipe are we providing here; should it be the same level of
detail as we sent to JPL?



\section{Analysis of Moving Object Processing System Performance \label{sec:mops}}


The linking of individual detections from difference images into plausible orbital tracks will be performed using
a special-purpose code referred to as the Moving Object Processing System (MOPS). There are several slightly modified
versions of MOPS in use by various projects; the original version was developed collaboratively by Pan-STARRS
and LSST, and is described in \cite{denneau13}. MOPS employs a two-step processing: first pairs of detections
from a given night are connected into {\it tracklets}, and then at least three tracklets are associated into a
candidate {\it track}. Realistic MOPS simulations show high linking efficiency ($>$99\%; \citealt{denneau13})
across all classes of Solar System objects. The core algorithmic components of MOPS are {\it findTracklets} and
{\it linkTracklets} kd-tree algorithms by \citet{kubica07}. {\it findTracklets} links \DIASources from a single
night to produce {\it tracklets}, and {\it linkTracklets} links tracklets from at least three nights to produce candidate
{\it tracks} (assuming quadratic motion in each coordinate; the LSST version also accounts for topocentric
corrections). Candidate tracks produced by MOPS are then filtered using initial orbital determination (IOD) step,
which is executed using a stand-alone code (e.g. OrbFit, \citealt{milani08}; OpenOrb, \citealt{OpenOrb2009}).

Given the empirically estimated false positive rates expected for LSST, discussed in the preceeding section,
in this section we show that MOPS performance is already adequate - MOPS requires significantly less
computing capacity than planned for other LSST data processing needs. In addition to reporting the results of
numerical experiments with MOPS, we also analyze them using analytic and semi-analytic results for the
rates of false tracklets and false tracks.




\subsection{A Summary of LSST tests of MOPS}

As a part of the Final Design Review preparations, the LSST team has developed an enhanced prototype
implementation of MOPS and analyzed its behavior. Here we summarize the main results of that work;
a detailed report is publicly available \citep{LDM-156}.

Simulated \DIASources were based on a Solar System model by \citet{Grav2011}.
The model includes about 11 million objects; about 9 million are main-belt asteroids. Observations span
30 days and were selected from a simulated baseline cadence (at that time, the baseline simulation was
OpSim3.61, which in this context is statistically the same as the current baseline cadence, {\it minion\_1016}).
The number of tracklets and tracks, the runtime, and the memory usage were studied as functions of
the false positive detection rate. The rate was varied from none to four times the asteroid detection rate
(100 deg$^{-2}$).  The highest rate corresponds to the expected false positive detection rate for LSST
($\rho_{FP} =  400$ deg$^{-2}$).

Tests were run with 16 threads on single 16 CPU node on Gordon cluster at San Diego Supercomputing
Center (in 2011). Due to computational constraints, a $v < 0.5$ deg/day velocity limit for pairing detections
into tracklets was imposed. For similar reasons, the filters that were imposed on track fitting were not
optimized, artificially reducing the yield. As we now understand the algorithmic scalings much better
(see Appendix~\ref{sec:appMOPS}), it is clear that these unoptimized filters have no major impact on the
simulation results and derived conclusions.

As expected, the addition of false detections increases the number of tracklets and tracks,
the runtime, and the memory usage. For the 4:1 false:true detection rate ratio, compared to case with
no false detections, the number of tracklets increases by about
a factor of 10, the number of tracks by about 50\%, and runtime increases by about a factor of 3.
For the 4:1 false:true detection rate ratio, the runtime with 16 CPUs is 33 hours, with maximum memory
usage of about 80 GB.




\begin{figure}[t!]
\centering
\vskip -0.3in
%\includegraphics[width=0.49\textwidth]{figures/tracklet}
%\includegraphics[width=0.49\textwidth]{figures/tracks}
\includegraphics[width=0.95\textwidth]{figures/track_stats}
\caption{A summary of MOPS tests for the dependence of the number of tracklets (left)
and tracks (right) on the false detection rate. As the rate of false detections
increases from none to four times the asteroid detection rate, the number of tracklets
increases by about an order of magnitude. At the same time, the number of candidate
tracks increases by only about 50\%.
\label{fig:MOPStests}}
\end{figure}




\subsection{Understanding MOPS Performance}

The rather slow increase of the number of tracks with false positive detection rate (a 50\% increase
although the number of tracklets increased by a factor of 10) is somewhat unexpected. We have
developed analytic and semi-analytic analysis to better understand the scaling of the number of
tracklets and tracks with false detection rate and other relevant parameters. Details of this
analysis are provided in Appendix~\ref{sec:appMOPS}. Here we briefly discuss the main results.

The increase of the number of tracklets with the false detection rate,
$\rho_{FP}$, shown in left panel in Figure~\ref{fig:MOPStests}, is well
described by eq.~\ref{eq:NttFalse}. In particular, the number of tracklets
approximately increases proportionally to $(C_1 + C_2\rho_{FP}^2)$, where $C_1$
and $C_2$ do not depend on $\rho_{FP}$. As both the full analytic result and the
simulations show, false tracklets quickly outnumber true tracklets even at low
false detection rates, resulting in the observed $\rho_{FP}^2$ behavior.

While the number of tracklets is dominated by the false detections, in the
baseline LSST cadence and the nominal noise assumptions under which the MOPS
simulations were run ($\rho_{FP} \leq 500\,\rm{deg}^{-2}$), the number of
tracks is not dominated by spurious detections---instead it is dominated by true
tracks and mislinkages between true objects. This is due to the fundamental
feature of MOPS: the 4-dimensional space of tracks (two coordinates and two
velocity vector components) is sparse at up to moderate levels of contamination,
and at the tested noise levels false tracklets are effectively filtered out. This
behavior accounts for the slow growth in tracks in the right panel of
Figure~\ref{fig:MOPStests}.

As we evaluate the impact of different survey parameters, we can assess the
number of tracks that would be generated (and thus require IOD processing) using
the analytic results developed in Appendix~\ref{sec:appMOPS}. For a given window
width and false detection density, the number of false tracks per search window
that would arise from false detections is given by
\begin{equation}
\label{eq:falsetracks2}
   N^{falsetracks} = 4.5 \times 10^6 \, \left( {N_w \over 30 \, {\rm day} } \right)^{8} \left( {\rho_{FP} \over 400 \, {\rm deg}^{-2} }\right)^{3.7}.
\end{equation}
This expression is valid around fiducial values and assumes $\rho_{ast}=100$ deg$^{-2}$.
The number of true tracks is of the order 10$^6$; therefore, with the baseline
window $N_W=15$ the contribution of false detections is small, while in the
enhanced NEO cadences with $N_W=30$ the contribution is only a factor of a few
times the number of true tracks.

\subsection{Required Computing Resources for MOPS and IOD Processing}

Given the modest computing resources used in MOPS tests described above, the runtime and memory
usage results bode well for LSST processing. Assuming a 1000-core machine dedicated to LSST moving
object processing (corresponding to about 1\% of the anticipated total LSST compute power at the
National Center for Supercomputing Applications), MOPS runtime for producing
candidate tracks should not exceed an hour, assuming sufficient parallelization can be achieved.

The IOD step can also be handled with anticipated resources and is trivially parallelizable. The number 
of available IOD computations for a compute system with $N_{core}$ cores and allocated runtime $T_{runtime}$ 
can be estimated as
\begin{equation}
  N_{IOD} = 3.6\times10^8 \left({ 0.1\,{\rm sec} \over T_{IOD}}\right) \,
                                         \left({ T_{runtime}  \over 10\,{\rm hr} }\right) \,
                                         \left({ N_{core}  \over 1000}\right).
\end{equation}
where $T_{IOD}$ is the time it takes to perform one IOD computation on a single core. Estimates
of the upper range of $T_{IOD}$ are of the order 50 ms (S. Chesley, priv. comm.), considerably below the fiducial
value of 100 ms adopted here. Tests of IOD with 
Find_Orb\footnote{\url{https://www.projectpluto.com/find_orb.htm}} using observations similar to
those from our short-arc MOPS tracks required about 0.3 ms to complete.
Given that the expected number of candidate tracks to filter using
IOD is well below $10^7$, it should be possible to accomplish the IOD step in well under an hour.
Alternatively, it is plausible that a 100-core machine might be sufficient for LSST moving object
processing (assuming no engineering safety margin).


\section{LSST Observing Cadence Optimization to Enhance PHA Completeness}

The effects of varying the LSST observing strategy on PHA completeness and other science can be evaluated in detail using a combination of the LSST Operations Simulator (OpSim) and the LSST Metrics Analysis Framework (MAF). 

The LSST Operations Simulation (OpSim) is a python software package that generates a realistic pointing history, with the time, filter, location, astronomical conditions, weather conditions, and predicted point-source 5-sigma limiting magnitude, for each LSST visit for ten years. This pointing history is generated using weather data (cloudiness and seeing) from the Cerro Pachon site and a high-fidelity model of the telescope itself (including slew and settle time and dome movement, for example), combined with a parameterized set of observing proposals that determine how the scheduling algorithm attempts to gather observations. By configuring OpSim with different parameters for the observing proposals, we can generate a series of simulated surveys which prioritize different science goals. 

The LSST Metrics Analysis Framework (MAF) is a user-oriented, python package for evaluating the pointing history from these simulated surveys in light of particular science goals or interests. The results of metrics coded in the MAF framework can be calculated for any given simulated survey and compared as proposal parameters are changed in OpSim. Metrics will be gathered from as wide a cross section of the astronomical community as possible, together with "figures of merit" that summarize and define 'success' for a given metric. This permits a thorough investigation of the trades between different observing strategies, in terms of the effect on science goals.

We can use MAF to evaluate the effect of various observing strategies on moving object completeness and characterization as well. Here we focus on discovery completeness of Potentially Hazardous Asteroids (PHAs) as both the observing strategy (related to changes in OpSim proposal parameters) and discovery criteria (related to LSST Data Management (DM) and the Moving Object Processing Software (MOPS) workloads) are varied. 

\subsection{Details of MAF analysis}

Using MAF to evaluate metrics for moving objects, such as PHAs, requires first defining the parameters of the input small body population by:

\begin{enumerate}
\item{Defining an orbital distribution for the input moving object population, specified by a set of orbital parameters for each object. Here we have chosen to use an orbital distribution defined by the large ($>1$ km) diameter known PHAs, as reported to the Minor Planet Center (MPC). This population is thought to be relatively complete and thus should be relatively unbiased. For comparison with other completeness estimates which use an Near Earth Object (NEO) population based on the \cite{Bottke2002} model instead of the subset of these objects which are classified as PHAs, we also evaluate completeness using a random sample of 2000 NEOs from the synthetic solar system model presented in \cite{Grav2011}.  A plot of the $a$, $e$, $i$ distribution for these PHAs and NEOs is shown in Figure~\ref{fig:PHA_orbits}.  }
\begin{figure}
\centering
\includegraphics[width=0.45\textwidth]{figures/pha20141031_orbits} 
\includegraphics[width=0.45\textwidth]{figures/neos_2k_orbits}
\caption{The eccentricity and inclination distributions, as a function of semi-major axis, of the PHAs (left) and NEOs (right) used in this analysis. The PHA population consists of the orbital distribution of $>1$~km PHAs recorded by the MPC as of 2014 (1511 objects). The NEO population is a random sampling of 2000 NEOs from the S3M \citep{Grav2011}, a synthetic solar system model based on the \cite{Bottke2002} NEO orbital distribution. \label{fig:PHA_orbits}}
\end{figure}

\item{Optionally, define a size for each object. For populations where the size distribution is strongly tied to the orbital distribution, this is necessary. However, most small body populations can be well described by independent orbital and size distributions; the PHA population larger than 140~m in diameter can be generally described in this manner. In these cases, a smaller set of orbits can be used to represent the overall larger population; during analysis, each object can be `cloned' from a fiducial $H$ magnitude associated with the orbit to a range of $H$ magnitudes covering the range interesting for analysis. This makes the metric analysis, and particularly the generation of the expected observations for each object, simpler and faster. As long as sufficient resolution of the orbital parameter space is maintained, the metric results over the range of $H$ magnitudes will be comparable to the results calculated with a larger population. Here we use the small population of $>1$~km diameter known PHAs and clone them to a range of $H$ magnitudes between $H$=11 and $H$=28. We have verified with a larger, simulated set of NEOs that reducing the population from 10,000 model NEOs to 2000 model NEOs does not change the calculated survey  completeness. }

\item{Optionally, define a spectrum or color for each object. This facilitates the conversion from $H$ magnitude (assumed to be in $V$ band) to the apparent magnitude in a given LSST observation, which may be in any of $u$, $g$, $r$, $i$, $z$, or $y$ filters. Here we have assumed that our entire PHA population consists of C-type asteroids, with resulting transformations to  LSST bandpasses as described in Table~\ref{tab:sed_colors}.  }
\begin{deluxetable}{ccccccc}
\centering
\tablecolumns{7}
\tablecaption{Color transformations from Harris $V$ band to LSST bandpasses, for C and S type asteroids. \label{tab:sed_colors}}
\tablewidth{0.7\textwidth}
\tablehead{ Type & $V-u$ & $V-g$ & $V-r$ & $V-i$ & $V-z$ & $V-y$  \\ }
\startdata
C  & -1.53 &  -0.28 &  0.18 &  0.29 &  0.30 & 0.30 \\
S & -1.82 &  -0.37 &  0.26 & 0.46 &  0.40 & 0.407  \\
\enddata
\end{deluxetable}

\end{enumerate}

Using the details of the input population, MAF then generates the expected observations of each object using the pointing history from a specific OpSim simulated survey. Ephemerides are generated using OpenOrb \citep{OpenOrb2009} for a closely spaced grid of times, and then interpolated to the exact times of each OpSim pointing. If the object is within the LSST field of view, its predicted position, velocity, and apparent $V$ magnitude (for the fiducial $H$ magnitude associated with the orbit) is recorded along with information about the simulated observation itself (such as the seeing, limiting magnitude, filter, and boresight RA/Dec). The full LSST camera footprint, including chipgaps, is used to determine if an object is within the field of view. The camera footprint is shown in Figure~\ref{fig:camera_footprint}. 

\begin{figure}
\centering
\includegraphics[width=0.65\textwidth]{figures/focalplane} 
\caption{Model of the LSST camera footprint, including chipgaps and CCD + raft layout. \label{fig:camera_footprint}}
\end{figure}

Trailing loss estimates are provided by MAF. Trailing losses occur whenever the motion of a moving object spreads its photons over a wider area than a simple stellar PSF. There are two aspects
of trailing loss to consider: simple SNR losses and detection losses.
The first is simply the degradation in SNR that occurs (relative to a
stationary PSF) because the trailed object includes a larger number of
background pixels in its footprint. This will affect photometry and
astrometry, but typically doesn't directly affect whether an object is
detected or not. The second effect (detection loss) is not related to
measurement errors but does typically affect whether an object passes
a detection threshhold. Detection losses occur because source
detection software is optimized for detecting point sources;
a stellar PSF-like filter is used when identifying sources that pass
above the defined threshhold, but this filter is non-optimal for
trailed objects. This can be mitigated with improved software ({\it                                                                               
e.g.} detecting to a lower SNR threshhold and attempting to detect
sources using a variety of trailed PSF filters). Both trailing losses can
be fit as:
\begin{eqnarray}
\Delta \, m & = &-1.25 \, log_{10} \left( 1 + \frac{a \, x^2} { 1 + b\,
    x} \right) \\
x & = & \frac{v \, T_{exp}} {24 \, \theta} 
\end{eqnarray}
where $v$ is the velocity (in degrees/day), $T_{exp}$ is the exposure
time (in seconds), and $\theta$ is the FWHM (in arcseconds). For
SNR trailing losses, we find $a = 0.67$ and $b = 1.16$; for
detection losses, we find $a=0.42$ and $b=0$. An illustration of the
magnitude of these trailing loss effects for 0.7'' seeing is given in
Figure~\ref{trailinglosses}. When considering whether a source would
be detected at a given SNR using typical source detection software,
the detection loss should be used.


and 'fading'

explain cloning and analysis (differential completeness), and how combined to generate cumulative completeness


\subsection{OpSim Simulated Surveys}

describe set of new opsim runs (minion1016, astrolsst011016, astrolsst011015, astrolsst011017)


- a basic description of the baseline cadence (you/I can lift that from 
  Ch. 2 in observing strategy white paper) 
- description of modifications for other 3 simulations 
- a set of basic metrics for all 4 sims that describe “non-NEO” programs: 
   fOArea, fONv, median coadded ugrizy depth for WFD, median airmass 
   for WFD (all bands together), parallax normed for WFD; use 10 yrs
   for baseline and 12 years for other 3 
   
 \subsection{Evaluation of PHA discovery completeness}
 
- existing cumulative completeness plots for each survey, with
       - 3 pairs in 30 days, SNR=4, no PSF detection loss
       - 3 pairs in 30 days, SNR=5, total trailing loss
       - 3 pairs in 15 days, SNR=5, total trailing loss
and summarize completeness changes 
\begin{enumerate}
\item baseline (minion1016) to more observations in NES
\item adding two years to astrolsst011016
\item extending MOPS window to 30 days, at SNR=5
\item adding longer exposures throughout the ecliptic (astrolsst011015/astrolsst1017)
\item making DM find objects with trailing loss, not detection losses
\item extending SNR limit to 4 instead of 5
\end{enumerate}

\subsection{Comparison with external completeness analysis}

The leading systematic effects in completeness estimates are: 
\begin{enumerate}
\item NEO vs. PHA difference (the completeness is about $\sim$5\% higher for PHAs than for NEOs) 
\item Different sample definitions: $H<22$ vs. $D>140$m (as shown by \citep{GMS2016}, completeness
           increases by $\sim$5\% when $H$-based criterion is used) 
\item Variations of ``discovery window'' (e.g., X visit pairs in N nights: changing N from 15 to 30 with X=3 increases
          completeness by 3\%; changing X from 3 to 4 with N=15 decreases completeness by 6\%). 
\end{enumerate}          
          

\subsection{Remaining uncertainties in completeness estimates}

Common to all, really.          

\begin{enumerate}
\item Sensitivity to details in sky coverage and cadence (e.g. nightly pairs of visits vs. quads of visits;
          requiring quads instead of pairs of visits decreases completeness by 30\% using baseline cadence; 
          about half of that loss can be recovered using cadence simulations that request four visits per night) 
\item Uncertainties when predicting effective image depth (system throughput, variation of the detection efficiency
          with the signal-to-noise ratio, treatment of trailing losses); for a survey that has a completeness above 60\%, 
          each additional one magnitude of depth for a given survey cadence increases the completeness by another 10\%.
\item Uncertainties when predicting apparent flux (albedo distribution, phase effects, photometric variability 
          due to non-spherical shapes, color distributions); assuming an uncertainty of 0.2 mag in the effective 
          limiting magnitude, the corresponding  systematic uncertainty in completeness is about 2\%.)
\item The slope of the asteroid size distribution (current measurement uncertainty of this parameter 
          corresponds to a systematic uncertainty in completeness of about 2\%.)
\item The impact of known objects (assuming that 43\% objects would be discovered by the start of
          LSST survey, \citep{GMS2016} boosted the final PHA completeness for LSST baseline survey by 11\%). 
\end{enumerate} 

Given these systematic effects, a comparison of different simulation results (both for the same system,
and those of different systems, especially systems operating at different wavelengths) has to be undertaken
with due care. It is unlikely that a meaningful quantitative comparison can be pushed beyond a level
of a few percent (and perhaps as much as 10\%). In practice, the completeness of a given operating survey
is best estimated using the object re-discovery rate. 





\section{Discussion and Conclusions\label{sec:discussion}}

We have examined and quantified here the ability of LSST to contribute to the census of NEO and PHA populations and the fulfillment of the Congressional George E. Brown, Jr. mandate, as well as examined options for further improvement of LSST's NEO and PHA yields relative to the baseline cadence.

We expect that the LSST strategy for discovering moving Solar System objects will be successful because the following three conditions are likely to be met:
\begin{enumerate}
	\item The LSST system hardware and image differencing software performance will result in false detection
	rates not significantly exceeding $\rho_{FP} =  450$ deg$^{-2}$, conservatively estimated here using real DECam data
	processed using prototype LSST software.
	\item Given an anticipated 1000-core machine, LSST MOPS will be able to easily process as many as
	10$^8$ tracklets per search window, and daily computations to produce up to about 10$^7$
	candidate tracks will be completed in about an hour.
	\item With the IOD computational budget of 0.1 sec per track -- comfortably above the benchmarked value of $<26$ms -- the final track filtering step can
	be easily accomplished in about an hour.
\end{enumerate}

Our determination of the expected false detection rate is based on processing data acquired by DECam using prototype LSST pipelines. With a conservative extrapolation of these data to expected LSST depth, we find expected rates of false detections of $450$~deg$^{-2}$. This estimate includes no provision for real-bogus type classifiers, which have been successfully applied by existing surveys to reduce the false detection rates by an order of magnitude or more \citep[e.g.][]{goldstein15}. It is therefore best to think of it as a {\it conservative upper limit}, possibly overestimating the true rate by a factor of few.

Assuming this rate, numerical tests with LSST MOPS prototypes and representative IOD routines demonstrate that the planned compute system will be (more than) adequate to
process LSST data. Even if the realized false detection rate is twice as high as
the (already conservative) estimate reported here, it can still be handled without a change of baseline cadence, linking criteria, or
increase in computing resources. Quantitatively, the false detection rates of up to about
1000 deg$^{-2}$ can be readily handled with an approximately 1000-core cluster dedicated to moving object processing (equating, in 2022, to $\sim 36$ CPUs, or $\sim 18$ nodes).


%Significant further compute margin exists, both because the overall LSST compute needs are driven by other more
%demanding processing\footnote{At the beginning of the survey, the LSST compute cluster is expected to count $\sim 22,000$ cores.}, and because there are various mitigation strategies that can decrease the
%compute load. For example, minor modifications of the cadence, such as a simple shortening
%of the nightly revisit time by about a factor of 3, could mitigate about a factor of two increase in
%the false detection rate. At the same time, a several times larger false detection rates for LSST
%than measured using DECam images and prototype LSST software are rather implausible. If the
%LSST camera, or any other system component, would somehow cause such high false detection rates,
%the whole LSST mission would be jeopardized.

Having established that the LSST will successfully identify moving objects, we've examined the expected discovery yields for the PHA and NEO populations. Our results show that the current LSST baseline survey strategy would yield a completeness for PHAs
with $H<22$ of about 66\% (without including objects discovered by prior and contemporaneous surveys). We compared our results presented here to an analogous recent study by \citet[]{GMS2016} and \citet{VeresChesley2017neo}, and find them consistent within the modeling uncertainties. The original
PHA completeness forecasts published in \cite{LSSToverview} are somewhat higher than reported here, primarily due to using different PHA populations. 

Finally, we examine and quantify the efficiency of a number of possible extensions to the LSST baseline survey that could raise the
completeness for PHAs with $H<22$ beyond 80\%. We find that the ``extra spur ecliptic visits over 12 years'' survey
(astro\_lsst\_01\_1016) with a 30-day MOPS linking window could boost the overall PHA completeness to 86\% (including objects discovered by contemporaneous surveys). With this cadence, 
the main LSST survey (``deep-wide-fast'') receives as many visits after 12 years as it would receive after 10 years of the
baseline cadence (minion\_1016). This makes it particularly attractive as, assuming resources were identified to extend the survey by two years, the number of remaining undiscovered PHAs could be essentially {\em halved} relative to the existing baseline, without seriously disadvantaging other LSST science cases.


\acknowledgements
This material is based upon work supported in part by the National Science
Foundation through Cooperative Agreement 1258333 managed by the Association of
Universities for Research in Astronomy (AURA), and the Department of Energy
under Contract No. DE-AC02-76SF00515 with the SLAC National Accelerator
Laboratory. Additional LSST funding comes from private donations, grants to
universities, and in-kind support from LSSTC Institutional Members.

\appendix
\section{LSST Image Processing Steps and Data Products Relevant for Asteroids} \label{sec:AppA}

The data products produced by the LSST Data Management system are described in
LSST Document LSE-163\footnote{See \url{http://ls.st/LSE-163}} (LSST Data Products
Definition Document). Here we briefly summarize parts of that
document\footnote{To ensure the continued scientific adequacy of LSST data
products, their designs and plans are periodically reviewed and updated and
thus LSE-163 is a living document -- please always consult the latest version.}
that are most relevant for discovering moving Solar System objects.

The LSST Data Management system will perform nightly analysis of difference images\footnote{A difference
image is an image produced by subtracting a science image from an appropriate
``average'' of the previously collected similar images of the same sky area, and using the
same filter.}, with the goal of detecting and characterizing astrophysical phenomena
revealed by their time-dependent nature. The detection of supernovae superimposed
on bright extended galaxies is an example of this analysis, and of course moving Solar
System objects are another example. The processing will be done on nightly/daily
basis and will result in the so-called Level 1 data products. Level 1 products will include
difference images, catalogs of sources detected in difference images (the so-called
\DIASources), static astrophysical objects\footnote{The LSST has adopted the nomenclature by
which single-epoch detections of astrophysical {\em objects} are called {\em sources}.
The reader is cautioned that this nomenclature is not universal: some surveys call
{\em detections} what LSST calls {\em sources}, and use the term {\em sources} for what
LSST calls {\em objects}.} these \DIASources are positionally associated to (the so-called \DIAObjects),
and moving Solar System objects (\SSObjects\footnote{\SSObjects used to be called
``Moving Objects'' in previous versions of the LSST Data Products baseline. The name is
potentially confusing as high-proper motion stars are moving objects as well. A more
accurate distinction is the one between objects {\em inside} and {\em outside} of the Solar
System.}). The catalogs will be entered into the Level 1 database and made available in near
real time. Notifications (``alerts'') about new \DIASources will be issued using
community-accepted standards\footnote{For example, VOEvent, see \url{http://ls.st/4tt}} within
60 seconds of observation.

The Moving Object Processing Software (\code{MOPS}) pipeline combines all unassociated \DIASources into
plausible \SSObjects and estimates their orbital parameters. The three main pipeline stages
include associating new \DIASources with known \SSObjects, discovering new \SSObjects,
and orbit refinement and management. This conceptual MOPS design is illustrated in
Figure~\ref{fig:Pipe8}. Further details about the MOPS pipeline design and implementation are available
from the LSST Science Pipelines Design Document\footnote{See \url{http://ls.st/LDM-151}}, LDM-151.
The next section briefly describes the main processing steps in nightly/daily Level 1 data processing.

\begin{figure}[!t]
    \centering
    %\vskip -2.3in
%    \hskip -0.2in
    \includegraphics[scale=0.60, angle=270]{MOPS-Level0}
    \vskip -0.1in
    \caption{Illustration of the conceptual algorithm design for the Moving Object Processing Software.
   \DIASources are data structures that describe detections of sources in difference images and
   \SSObjects are data structures that describe discovered Solar System objects (see Table~\ref{tab:SSObj}).
\label{fig:Pipe8}}
\end{figure}


\subsection{LSST Level 1 Data Processing}

Level 1 data products are a result of difference image analysis (DIA).
\DIASources are sources detected on difference images with the signal-to-noise ratio $S/N>transSNR$,
with $transSNR$=5.
They represent changes in flux with respect to a deep template. Physically, a \DIASource may be an observation of new astrophysical object that was not present at that position in the template image (for example, an asteroid), or an observation of flux change in an existing source (for example, a variable star). Their flux can be negative (eg., if a source present in the template image reduced its brightness, or moved away). Their shape can be complex (eg., trailed, for a source with proper motion approaching $\sim {\rm deg}/{\rm day}$, or ``dipole-like", if an object's observed position exhibits an offset -- true or apparent -- compared to its position on the template).
Some \DIASources will be caused by background fluctuations; with $transSNR = 5$,
the expected false positive rate is about three per CCD ($\sim 60$ per sq. deg.) for the median seeing,
or of the order 500,000 per typical night.
The expected number of false positives due to background fluctuations is a very strong function
of adopted $transSNR$: a change of $transSNR$ by 0.5
results in a variation of an order of magnitude, and a change of $transSNR$ by unity changes the number of false
positives by about two orders of magnitude.

Clusters of \DIASources detected on visits taken at different times are associated with either a \DIAObject or an \SSObject, to represent the underlying astrophysical phenomenon. The association can be made in two different ways: by assuming the underlying phenomenon is an object within the Solar System moving on an orbit around the Sun\footnote{LSST pipelines will not fit for motion around other Solar System bodies; eg., identifying new satellites of Jupiter is left to the community.}, or by assuming it to be distant enough to only exhibit small parallactic and proper motion\footnote{Where ``small'' is small enough to unambiguously positionally associate together individual apparitions of the object.}. The latter type of association is performed during difference image analysis right after the image has been acquired. The former is done at daytime by \code{MOPS}, unless the \DIASource is an apparition of an already known \SSObject. In that case, it will be flagged as such during difference image analysis. At the end of the difference image analysis of each visit, LSST will issue time domain event alerts for all
newly detected \DIASources\footnote{For observations on the Ecliptic near the opposition Solar System objects will dominate the \DIASource counts and (until they're recognized as such) overwhelm the explosive transient signal. It will therefore be advantageous to quickly identify the majority of Solar System objects early in the survey.}.

\subsubsection{Nightly Difference Image Processing \label{sec:ssProcessing}}

The following is a high-level description of steps which will occur during regular {\em nightly}
difference image analysis:
\begin{enumerate}
\item A visit is acquired and reduced to a single {\em visit image} (cosmic ray rejection, instrumental signature removal\footnote{Eg., subtraction of bias and dark frames, flat fielding, bad pixel/column interpolation, etc.}, etc.).
\item The visit image is differenced against the appropriate template and \DIASources are detected and
their properties measured.
\item The flux and shape\footnote{The ``shape'' in this context consists of weighted 2$^{\rm nd}$ moments
of the intensity distribution, as well as fits to a trailed source model and a dipole model.} of the DIASource are measured on the difference image. PSF photometry is performed on the visit image at the position of the \DIASource to obtain a measure of the total flux.
\item The Level 1 database is searched for a \DIAObject or an \SSObject with which to positionally associate the newly discovered \DIASource\footnote{The association algorithm will guarantee that a \DIASource is associated with not more than one existing \DIAObject or \SSObject. The algorithm will take into account the parallax and proper (or Keplerian) motions, as well as the errors in estimated positions of \DIAObject, \SSObject, and \DIASource, to find the maximally likely match. Multiple \DIASources in the same visit will not be matched to the same \DIAObject.}. If no match is found, a new \DIAObject is created and the observed \DIASource is associated to it.
\item If the \DIASource has been associated with an \SSObject (a known Solar System object), it will be flagged as such and an alert will be issued. Further processing will occur in daytime (see \S\ref{sec:ssProcessing} below).
\item Otherwise, the associated \DIAObject measurements will be updated with new data
collected during previous 12 months. All affected columns will be recomputed, including proper motions, centroids, light curves, etc.
\item The \DR\footnote{\DR is a database resulting from annual data release processing.} is searched for one or more \Objects positionally close to the \DIAObject, out to some maximum radius\footnote{Eg., a few arcseconds.}. The IDs of these nearest-neighbor \Objects are recorded in the \DIAObject record and provided in the issued
event alert.
\item An alert is issued that includes the \DIASource ID, the \SSObject ID or \DIAObject ID, and the associated science content (centroid, fluxes, low-order lightcurve moments, periods, etc.), including the full light curves.
\item For all \DIAObjects overlapping the field of view, including those that have an associated
new \DIASource from this visit, forced photometry will be performed on difference image (point source photometry only).
No alerts will be issued for these \DIASources.
\item Within 24 hours of discovery, {\em precovery} PSF forced photometry will be performed on any difference image overlapping the position of new \DIAObjects taken within the past 30 days, and added to the database. Alerts will not be issued with precovery photometry information.
\end{enumerate}

In addition to the processing described above, a smaller sample of sources detected on difference images {\em below} the nominal $transSNR = \transSNR$ threshold will be measured and stored, in order to enable monitoring of difference image analysis quality.

Also, the system will have the ability to measure and alert on a limited\footnote{It will be sized for no less than $\sim 10\%$ of average \DIASource per visit rate.} number of sources detected below the nominal threshold for which additional criteria are satisfied. For example, a $transSNR = 3$ source detection near a gravitational keyhole\footnote{
A gravitational keyhole is a region of space where Earth's gravity would modify the orbit of a passing asteroid
such that the asteroid would collide with the Earth in the future.}
may be highly significant in assessing the danger posed by a potentially hazardous asteroid.
The initial set of criteria will be defined by the start of LSST operations.

\subsubsection{Solar System Object Processing \label{sec:ssProcessing}}

The following will occur during regular Solar System object processing in daytime\footnote{Note that there {\em is no strict bound on when daytime Solar System processing must finish}, just that, averaged over some reasonable timescale (eg., a month), a night's worth of observing is processed within 24 hours. Nights rich in moving objects may take longer to process, while nights with less will finish more quickly. In other words, the system requirement is on {\em throughput}, not latency.}, after a night of observing (see Figure~\ref{fig:Pipe8}):
\begin{enumerate}
\item The orbits and physical properties of all \SSObjects re-observed on the previous night are recomputed. External orbit catalogs (or observations) are also used to improve orbit estimates. Updated data are entered to the \SSObjects table.
\item All \DIASources detected on the previous night, that have {\em not} been matched at a high confidence level to a known \Object,
\DIAObject, \SSObject, or an artifact, are analyzed for potential pairs, forming {\em tracklets}.
\item The collection of tracklets collected over the past 30 days is searched for subsets forming {\em tracks} consistent with being on the same Keplerian orbit around the Sun.
\item For those that are, an orbit is fitted and a new \SSObject table entry created. \DIASource records are updated to point to the new \SSObject record. \DIAObjects ``orphaned'' by this unlinking are deleted.\footnote{Some \DIAObjects may only be left with forced photometry measurements at their location (since all \DIAObjects are force-photometered on previous and subsequent visits);  these will be kept but flagged as such.}.
\item Precovery linking is attempted for all \SSObjects whose orbits were updated in this process. Where successful, \SSObjects (orbits) are recomputed as needed.
\end{enumerate}


\subsubsection{Level 1 Catalogs}
\label{sec:level1db}

The described alert processing design relies on the ``living'' \DB that contains the objects and sources detected on difference images. At the very least\footnote{It will also contain exposure and visit metadata, MOPS-specific tables, etc.}, this database will have tables of \DIASources, \DIAObjects, and \SSObjects, populated in the course of nightly and daily difference image and Solar System object processing\footnote{The latter is also colloquially known as {\em DayMOPS}.}. As these get updated and added to, their updated contents becomes visible (query-able) immediately\footnote{No later than the moment of issuance of any event alert that may refer to it.}.

Table~\ref{tab:SSObj}  presents the {\em conceptual schema} for the \SSObject table (it conveys {\em what} data
will be recorded in each table, rather than the details of {\em how}).
Columns whose type is an array will likely be expanded to one table column per element of the array
once this schema is translated to SQL\footnote{The SQL realization of this schema can be browsed at \url{http://ls.st/8g4}}. In addition, the table presented here is normalized (i.e., it contains no redundant
information with other tables in Level 1 database). For example, since the band of observation can be found
by joining a \DIASource table to the table with exposure metadata, there's no column named {\tt band} in the \DIASource table. In the as-built database, the views presented to the users will be appropriately denormalized
for ease of use.

\subsubsection{\SSObject Table}

\begin{center}
\label{tab:SSObj}
\begin{longtable}{p{3cm}p{2cm}p{2cm}p{5cm}}
%\caption[\SSObject Table]{\SSObject Table} \\

\hline \multicolumn{1}{c}{\bf Name} & \multicolumn{1}{c}{\bf Type} & \multicolumn{1}{c}{\bf Unit} & \multicolumn{1}{c}{\bf Description} \\ \hline
\endhead

\hline \multicolumn{4}{r}{{\em Continued on next page}} \\
\endfoot

\hline\hline
\endlastfoot

ssObjectId & uint64 & ~ & Unique identifier. \\

oe & double[7] & various & Osculating orbital elements at epoch ($q$, $e$, $i$, $\Omega$, $\omega$, $M_0$, epoch). \\

oeCov & double[21] & various & Covariance matrix for \texttt{oe}. \\

arc & float & days & Arc of observation. \\

orbFitLnL & float & ~ & Natural log of the likelihood of the orbital elements fit. \\

orbFitChi2 & float & ~ & $\chi^2$ statistic of the orbital elements fit. \\

orbFitNdata & int & ~ & The number of data points (observations) used to fit the orbital elements. \\

MOID & float[2] & AU & Minimum orbit intersection distances\footnote{\url{http://www2.lowell.edu/users/elgb/moid.html}} \\

moidLon & double[2] & degrees & MOID longitudes. \\

H & float[6] & mag & Mean absolute magnitude, per band (Muinonen et al. 2010 magnitude-phase system). \\

${\rm G_1}$ & float[6] & mag & $G_1$ slope parameter, per band (Muinonen et al. 2010 magnitude-phase system). \\

${\rm G_2}$ & float[6] & mag & $G_2$ slope parameter, per band (Muinonen et al. 2010 magnitude-phase system). \\

hErr & float[6] & mag & Uncertainty of H estimate.\\

g1Err & float[6] & mag & Uncertainty of $G_1$ estimate. \\

g2Err & float[6] & mag & Uncertainty of $G_2$ estimate. \\

flags & bit[64] & bit & Various useful flags. \\ \hline

\end{longtable}
\end{center}

The $G_1$ and $G_2$ parameters for the large majority of asteroids will not be well constrained until later in the survey. LSST may decide not to fit for it at all over the first few DRs and add it later in Operations, or provide two-parameter $G_{12}$ fits. Alternatively, they may be fitted using strong priors on slopes poorly constrained by the data. The design of the data management system is insensitive to this decision, making it possible to postpone it to Commissioning to ensure it follows the standard community practice at that time.
The LSST database will provide functions to compute the phase (Sun-Asteroid-Earth) angle $\alpha$ for every observation, as well as the reduced, $H(\alpha)$, and absolute, $H$, asteroid magnitudes in LSST bands.

\section{The Impact of False Detections on MOPS Performance \label{sec:appMOPS}}


We seek to develop an analytic understanding for the behavior of the MOPS results.
In particular, we want to be able to predict the numbers of tracklets and
candidate tracks for a given input number of true and false detections. In addition, we seek
to understand how these numbers scale with the search window width,
velocity cutoff when forming tracklets, the temporal separation of two
detections in a tracklet, and the density of false detections. For example, available
MOPS experiments indicate that the number of tracklets increases with
the square of the false detection density, but other scalings are unclear,
especially the behavior of false candidate tracks.

We first derive the simpler false tracklet rates, and then use these results to
discuss false candidate track rates. 


\subsection{Expected False Tracklet Rates \label{sec:tracklets} }

Given a detection in the first difference image, another difference image, obtained at a different epoch,
is searched for a matching detection to form a tracklet,  {\it e.g.} \citet{denneau13, kubica07}. For orientation,
the sky density of asteroids down to LSST $5\sigma$ faint flux limit ($r \sim 24.5$) is of the order
$\rho_{ast} \sim 100$ deg$^{-2}$. The predicted highest asteroid sky density for $r<24.5$,
on the Ecliptic, is up to about five times larger (with an uncertainty of about a factor of 2,
depending on model assumptions), and the density decreases rapidly with the ecliptic latitude.
A typical LSST observing night includes about 1000 visits, with two visits per night over
the active sky area. The nominal LSST field-of-view area is $A_{FOV}=9.6$ deg$^2$, with a
fill factor of 0.9, giving an effective field-of-view area of $A_{FOV}^{eff}=8.64$ deg$^2$. Hence,
the number of detected asteroids per night is of the order 500,000 (with implied two detections
per asteroid), although it can be significantly lower when the Ecliptic is not well covered (and
it could be a few times higher if the majority of visits were obtained along the Ecliptic).

The number of false detections due to (Gaussian) background fluctuations is
about $\rho_{bkgd} = 60$ deg$^{-2}$, assuming typical LSST seeing (0.8 arcsec)
and SNR$>$5. For a given seeing and SNR threshold, the rate of false detections can never be
lower than this estimate. This false detection rate decreases with the square of the seeing, and
strongly depends on SNR: the rate increases/decreases by as much as a factor of about ten
when SNR threshold is changed to 4.5 and 5.5, respectively (see \S\ref{sec:imDiff}).

Analysis of DECam images reduced using prototype LSST software, described in \S\ref{sec:imDiff},
shows a higher rate of detections in difference images, and a fraction of those detections
cannot be readily associated with true moving objects. This analysis implies a conservative
upper limit for the false detection rate of about $\rho_{FP} =  400$ deg$^{-2}$. This value
is conservative because analyzed DECam fields are close to the Ecliptic, with a significant but
not well known contribution from real asteroids (due to very faint flux levels, $r \sim 24$),
and it also includes true astrophysical transients that are not associated with static objects
(stars and galaxies). It is quite possible that the false detection rate might be several
times lower, though we will proceed with the most conservative estimate above.

The sky density of detections in difference images, $\rho_{det}$, is given by
the sum of contributions from true asteroids and false detections, $\rho_{det} = \rho_{ast} + \rho_{FP}
= 500$ deg$^{-2}$. When searching for a matching detection in another difference image, there are
two distinct types of behavior. Correct matches of detections of the same asteroid into tracklets follow the behavior
expected for correlated samples: as long as the object's angular displacement between the two epochs
is sufficiently larger than the seeing disk, while at the same time smaller than the search radius, the
number of matches (that is, the number of true tracklets produced per LSST pointing, assuming
two visits of the same area per night) is simply
\begin{equation}
                  N_{tracklet}^{true} = \rho_{ast}  \, A_{FOV}^{eff},
\end{equation}
With  $\rho_{ast} = 100$ deg$^{-2}$, $N_{tracklet}^{true} \sim 1,000$ per a pair of visits, and with
500 visit pairs per typical observing night, $N_{tracklet}^{true} \sim 500,000$ per night (same as
the number of detected asteroids in the active sky area, of course). Again,
this number can be much lower for fields far away from the Ecliptic, and a few times larger
for exceptionally good coverage of the Ecliptic. We emphasize that this number of true tracklets
does not directly depend on the search radius, nor the time elapsed between the two visits, as long
as they have their plausible values (about an arcminute, and a few tens of minutes, as discussed
further below).


\begin{figure}[t!]
\centering
\vskip -2.0in
\includegraphics[width=0.95\textwidth]{figures/TrackSlide2}
\vskip -2.2in
\caption{An illustration of positional matching of detections to form tracklets.
Detections come in two flavors: asteroids (A, circles) and false detections (FP, triangles).
The figure shows the search for a matching detection in epoch 2 for each detection
in epoch 1, with a maximum search radius $\delta_{max}$. Note that there are six
possibilities: matches A-A, A-FP, FP-A, FP-FP, and orphaned A and FP.
\label{fig:TrackSlide2}}
\end{figure}

There are three other types of tracklets that follow the behavior for uncorrelated (random)
samples: associations of different asteroids, associations of asteroids and false detections,
and tracklets made of two false detections. Assuming the same $\rho_{det}$ in both
difference images, for each of $N_{det} = \rho_{det} \, A_{FOV}^{eff}$ detections in one image,
we search for a matching detection in another image (see Figure~\ref{fig:TrackSlide2}).
The search radius is given by
\begin{equation}
                     \delta_{max} = v_{max} \, \Delta t .
\end{equation}
Here $v_{max}$ is the  cutoff velocity and $\Delta t$
is the time elapsed between the two images. For LSST baseline cadence, $\Delta t$ is in
the range 20-60 minutes. The search area, $A_S = \pi \delta_{max}^2$, is then
\begin{equation}
\label{eq:AS}
      A_S = 0.0055 \left( v_{max}  \over {\rm deg \, day}^{-1} \right)^2 \, \left(\Delta t \over {\rm hour} \right)^2 {\rm deg}^2.
\end{equation}
To guide setting the cutoff velocity, simulations imply that 95\% of NEO detections have $v<1$ deg day$^{-1}$; with this threshold,
the completeness for main-belt asteroids is essentially 100\%. Objects moving faster than 1 deg day$^{-1}$ will
be easily resolved in LSST images and can be treated separately using specialized algorithms.
Adopting $v_{max} = 1$ deg day$^{-1}$,  and $\Delta t = 30$
minutes (which together  imply a search radius of $\delta_{max} = 1.3$ arcmin), gives a search area of
$A_S = 0.0014$ deg$^2$.


The expectation value for the number of matching detections within the search area $A_S$ (that is, the expected
number of tracklets per matching trial) is
\begin{equation}
                      p_{tracklet}^{false} =   \rho_{det}  \, A_S,
\end{equation}
and the total expected number of {\it false} tracklets for $N_{det}$ trials is thus
\begin{equation}
\label{eq:NttFalse}
           N_{tracklet}^{false} = N_{det} \, p_{tracklet}^{false} =  N_{visit} \, \rho_{det}^2 \, A_S \, A_{FOV}^{eff} = N_{visit} \, \rho^2_{FP}  \, A_S \, A_{FOV}^{eff} \,  \left(1 + 2 \eta + \eta^2\right),
\end{equation}
where $\eta = \rho_{ast}  / \rho_{FP} \sim 0.25$ (recall that $\rho_{det} = \rho_{ast} + \rho_{FP}$).
With $\rho_{ast} = 100$ deg$^{-2}$ and  $\rho_{FP} = 400$ deg$^{-2}$,
$N_{tracklet}^{false} \sim 3,000$ per pair of visits, and $N_{tracklet}^{false} \sim 1.5$ million per observing night with
$N_{visit}=500$ visit pairs. We note that the density of false tracklets ($\rho_{tracklet}^{false}=350$ deg$^{-2}$) is similar to
$\rho_{FP}$; this similarity is a consequence of choosing $\delta_{max}$ such that $\rho_{FP} A_S \sim 1$.

The first term in eq.~\ref{eq:NttFalse} is the largest and corresponds to tracklets made of two false
detections ($\sim1.0$ million), the second term corresponds to associations of asteroids and false detections,
and the third and the smallest term ($<0.1$ million) is due to incorrect associations of different asteroids.
For the chosen parameter values, the total number of tracklets is about 2 million per observing night, Given that
these choices are rather conservative, this estimate is essentially an upper limit; approximately,
{\it we expect of the order a million tracklets per observing night}.

To the first order ($\eta \approx 0$), the total number of tracklets per night is
\begin{equation}
    N_{tracklet} =  N_{tracklet}^{true} + N_{tracklet}^{false} =
       N_{visit} \, A_{FOV}^{eff} \, \left(\rho_{ast}  + \rho^2_{FP}  \, A_S \right).
\end{equation}
In addition to $N_{tracklet}^{false}$ scaling with the square of $\rho_{FP}$, as demonstrated using MOPS,
$N_{tracklet}^{false}$ scales with the square of
both $v_{max}$ and  $\Delta t$ (via the dependence on $A_S$). Therefore, if $\Delta t$ would be made
as small as 10 minutes by modifying the observing strategy, the resulting $N_{tracklet}^{false}$ would be about an
order of magnitude smaller (and $N_{tracklet}$ about three times smaller).  Hence, the shortening of $\Delta t$ is
a good mitigation strategy against high false detection rates in difference images\footnote{An
extreme example of this mitigation strategy would be to obtain two consecutive 30-second visits -- their
mid-exposure times would be separated by 34 seconds (additional 2 seconds due to shutter motion and another
2 seconds due to readout), which is sufficient to detect motion faster than about 0.1 deg day$^{-1}$.}.


\subsubsection{False tracklet velocity distribution \label{sec:falsev}}

False tracklets have randomly distributed velocities (motion vectors) with a cutoff given by $v_{max}$
(recall that $v_{max} = 1$ deg day$^{-1}$ was adopted above). The implied tracklet velocity is given by
\begin{equation}
                       v =  \delta / \Delta t,
\end{equation}
where $\delta$ is the angular separation of two detections. Since the number of tracklets
with separation $\delta$ increases linearly with $\delta$ (because the area of a circular
annulus is $2\pi r dr$), the false tracklet velocity distribution will increase linearly with
$v$ for $v<v_{max}$, and the vector orientation will be random. We show below that candidate
tracks can be efficiently pruned using this result.



\subsection{Expected False Track Rates \label{sec:tracks} }

In this section, we present an approximate estimate of the expected number of false candidate tracks.
Our goal is to derive the scaling of this number with the relevant input parameters, such as the true and false
tracklet rates per night ($N_{tracklet}^{true}=5\times10^5$ and $N_{tracklet}^{false}=1.5\times10^6$, respectively).
For a fiducial case, we assume that the search window is $N_w= 30$ days wide;  therefore, with $N_{tracklet} = 2\times10^6$
per night, there are $6\times10^7$ tracklets in the fiducial dataset. With about 4,300 deg$^2$ (500 pairs of visits)
of sky observed each night, the average density of (all) tracklets is $\rho_{tracklet} = 450$ deg$^{-2}$. Assuming
that on average the same field is revisited every $T_{revisit}=3$ days, the active area includes about 13,000 deg$^2$
of sky.

As discussed below in more detail, there are of the order 1000 different ways to chose a triplet of nights
from the search window. Given 10$^6$ tracklets per night, there are of the order
10$^{21}$ different combinations of tracklet triplets that could form a candidate track.
While this number of candidate tracks is obviously prohibitively large to test for consistency
with heliocentric Keplerian motion, it can be sufficiently reduced (to about the same number
as the number of true tracks along the Ecliptic) using pre-filtering steps based on tracklet motion
vectors, summarized below and following similar methods as described in \citet{denneau13, kubica07}. 

In the first step, the motion vector of a tracklet from the first night is linearly extrapolated
to the second night and tracklets from the second night are searched for within a radius set
by the orbital curvature (which dominates over LSST's expected astrometric errors). With appropriate use of
kd-trees and similar algorithms for fast searches, only a small fraction (of the order a percent)
of tracklets from the second night need to be examined in detail. The cutoff radius varies
from $\sim$1 arcmin for the case of two consecutive nights to $\sim$1 deg. for a 15-day separation
(as discussed in detail further below). In addition, the velocity of second tracklet is required
to be consistent with the velocity implied by the positions of the two tracklets. After
this step, there are about 10$^{10}$ tracklet pairs for further processing (for $N_w=30$
days).

In the second step, parameters of a parabola (for each coordinate) are constrained using the
positions and velocities of the two tracklets, and this parabolic motion is extended to a third
night to search for matching third tracklet. This step results in up to 10$^{11}$ candidate
tracks.

Using the positions of the three tracklets, parabolic motion (for each coordinate) is fit
in the third step. The velocities implied by this motion are compared to the velocities for
the first and third tracklet. This filtering step reduces the number of candidate tracks
by a factor of about 10$^{5}$ and brings the number of false candidate tracks to
the same range as the number of true tracks close to the Ecliptic. These three matching
and pre-filtering steps bring the number of candidate tracks to a level that
can be easily handled by the IOD filtering step.

We now proceed with a more detailed description of three pre-filtering steps
for candidate tracks.

\subsubsection{The Number of 3-night Combinations in the Search Window}

We can form a candidate triplet of tracklets by first choosing the middle (second) tracklet.
For simplicity, we will measure time of observation in integer days. Given $N_w$ nights
in the search window, the middle tracklet comes from night indexed $k$, with
$2 \le k \le (N_w-1)$. The night that contributes the first tracklet is indexed by $j$,
with $1 \le j \le (k-1)$, and the night that contributes the third tracklet is indexed by $l$,
with $(k+1) \le l \le N_w$. The number of 3-night combinations can be expressed in a closed
form
\begin{equation}
\label{eq:N3}
  N_{3nights} = \sum_{k=2}^{N_w-1} \, (k-1)\, (N_w-k) =\frac{1}{6}N_w^3 - \frac{1}{2}N_w^2 + \frac{1}{3}N_w,
\end{equation}
giving $N_{3nights} = 455$ for $N_w=15$ and $N_{3nights} = 4,060$ for $N_w=30$.  Note
that for large $N_w$, $N_{3night}$ is proportional to $N_w^3$ -- the number of 3-nights
combinations increases by about an order of magnitude when $N_w$ is doubled from
15 days to 30 days.

It is important to point out that in steady-state processing a single night is added to the
window from the previous night, and the first night is dropped. Therefore, only the {\it new}
3-night combinations, where the third night is the last night in the search window, need
be considered in steady-state processing (and the ramp up is easy because of the gradually
increasing search window size). It is straightforward to show that the number of such
3-night combinations is
\begin{equation}
\label{eq:N3n}
  N_{3nights}^{new} = \sum_{k=2}^{N_w-1} \, (k-1) =\frac{1}{2}N_w^2 - \frac{3}{2}N_w + 1,
\end{equation}
yielding $N_{3nights}^{new} = 91$ for $N_w=15$ and $N_{3nights}^{new} = 406$ for $N_w=30$.
Note that $N_{3nights}^{new} \sim N_{3nights} / 10)$ for $N_w=30$, which represents
a significant reduction.


\subsubsection{The Tracklet Motion Vector Accuracy \label{sec:astromerrors}}

In addition to its mean position at the mean epoch, each tracklet constrains the motion vector.
Typical astrometric errors for LSST detections will range from about 50 mas at SNR=100 (dominated
by systematics) to 150 mas at SNR=5 (dominated by random errors). For simplicity, we will assume 
hereafter that the astrometric errors are
$\sigma_a=150$ mas for all detections, or $\sim 100$ mas per coordinate. With a temporal
separation of two detections in a tracklet of $\Delta t$, the motion vector is measured with an
accuracy per coordinate of
\begin{equation}
\label{eq:sigv}
          \sigma_v = 3.6 \, \left({\rm hour} \over \Delta t\right) \,\,\, {\rm arcsec} \, {\rm day}^{-1}.
\end{equation}
With a typical $\Delta t = 30$ min, and assuming a linear motion in each ecliptic coordinate (longitude
$\lambda$ and latitude $\beta$), each coordinate can be predicted at time $t$ with an accuracy of
\begin{equation}
            \sigma_x = 7.2 \, \Delta k \,\,\, {\rm arcsec},
\end{equation}
where $\Delta k$, in days, is the elapsed time between the mean tracklet epoch and time $t$
(for example, the number of nights between the first and the second tracklet in a candidate track).
For illustration, when $\Delta k = 7$ days, $\sigma_x = 50$ arcsec, which is roughly the same
as the typical detection separation in a tracklet, and comparable to typical distance between
two tracklets.  However, it turns out that the positional discrepancies due to a simple linear extrapolation of
motion for NEOs are an order of magnitude larger than the astrometric measurement errors
even in case of two consecutive nights ($\sim$1 arcmin vs. 7 arcsec, respectively). We proceed with
a quantitative analysis of the required matching radius using simulated orbits for main-belt asteroids
and NEOs.



\subsubsection{Initial Linking of Tracklets into Candidate Tracks}

Given a combination of 3 different nights from the search window, for each tracklet
from the first night we can linearly extrapolate its motion vector and require that the
measured  position of a tracklet from the second night is consistent with the predicted
position (the night ordering can be reversed from 1-2-3 to 3-2-1). Given a tracklet from
the first night, it is not necessary to search through all tracklets from the second night.
Search methods such as kd-trees can be used to rapidly reject tracklets that have no
chance of being matched. As an example of a ``poor man's'' rapid search, consider the
fact that tracklets from each night are already ``self-organized'' into about 500 visits,
which correspond to a field of view with a diameter of 3.5 deg. It is easy to show that with
an upper limit on possible motion of 5 deg, only 19 visits from the second night need to be
searched for matching tracklets. This significant reduction of a factor of $\sim$25 in
the number of candidate matching tracklets can be further boosted by applying more
sophisticated tree algorithms.

Using ecliptic longitude $\lambda$ for illustrative purposes, the predicted search position for the
second tracklet is
\begin{equation}
\label{eq:lambdaPred}
               \lambda_2^\ast = \lambda_1 + v_1^\lambda \, \Delta T_{21},
\end{equation}
where $\Delta T_{21}$ is the elapsed time between the epochs of the first and second tracklet,
and $v_1^\lambda$ is the longitudinal component of $v_1$, the motion vector for the first
tracklet, {\it divided by} cos($\beta$). The expectation value for the number of matches in
an ellipse (see the left panel in Figure~\ref{fig:TrackSlide1})
centered on predicted position ($\lambda_2^\ast, \beta_2^\ast$), and within limits $r_\lambda^{max}$
and  $r_\beta^{max}$ along the Ecliptic longitude and latitude, is given by
\begin{equation}
\label{eq:Nmdk}
     N_{match}(\Delta k) = \pi \, r_\lambda^{max} \, r_\beta^{max}  \, \rho_{tracklet} \left({1 \, {\rm day} \over T_{revisit}}\right)
\end{equation}
where the division by $T_{revisit}$ reflects the fact that each field is revisited on average only
every $T_{revisit}$ days (statistically speaking; the number of matches is zero for all but one
night out of $T_{revisit}$ nights).

The extrapolation given by eq.~\ref{eq:lambdaPred} implies that orbits can be approximated by
linear motion (in each coordinate) over time $\Delta T$. This is an incorrect assumption
due to orbital curvature and we analyze this effect using orbital simulations of MBA and NEO
samples described in \S\ref{sec:MAFdetails}.

Analysis of simulated samples shows that an adequate acceleration limit\footnote{See also
Figure 16 in \citet{LDM-156}.} is $a^{max}=0.02$ deg day$^{-2}$: essentially
all main-belt asteroids and more than 95\% of NEOs satisfy this criterion. If this acceleration
were constant during an interval of $\Delta k$ days, the maximum positional discrepancy
would be proportional to $\Delta k^2$. Numerical analysis of the simulated orbital motions
suggests that an approximately constant selection completeness (as a function of $\Delta k$)
is attained for
\begin{equation}
\label{eq:matching1}
                r_\beta^{max} = A \, \Delta k^{1.5}.
\end{equation}
with $A=1.0$ arcmin, and $r_\lambda^{max} = 5 \, r_\beta^{max}$. The achieved completeness for
a fiducial $\Delta k$=7 days is 0.99 for MBAs and 0.95 for NEOs, with very little dependence
on $\Delta k$ for 1 day $\le \Delta k \le$ 21 days (per single search window -- note that most
objects will have multiple discovery chances).  With this linear motion model, the number
of matched tracklets per single trial tracklet is
\begin{equation}
\label{eq:Nmdk2}
   N^{L}_{match}(\Delta k) = 1.96 \, \left({1 \, {\rm day} \over T_{revisit}}\right) \,
                    \left( \rho_{tracklet}  \over 450 \, {\rm deg}^{-2} \right) \, (\Delta k)^3.
\end{equation}
For example,  the expected number of matches for $\Delta k$=7 days is $\sim$224
(a 19 arcmin by 93 arcmin matching ellipse), and rises to $\sim$6,000 for
$\Delta k$=21 days.

Given the two matched tracklets, we can then approximate the motion as a parabola
\begin{equation}
\label{eq:parabola}
          \lambda(t) = \frac{1}{2}a^\lambda \, t^2 + v^\lambda \, t + \lambda_1,
\end{equation}
where $t = mjd - mjd_1$ (and analogously for latitude $\beta$).  Using the tracklet
positions and the motion vector of the first tracklet, the acceleration can be directly
estimated as
\begin{equation}
 \label{eq:accPred}
             a^\lambda = 2 \, {\lambda_2- \lambda_1 - v_1^\lambda \Delta T_{21} \over \Delta T^2_{21}},
\end{equation}
and the predicted velocity for the second tracklet can be estimated from
\begin{equation}
\label{eq:v2cut}
        (v^\lambda_2)^\ast =  a^\lambda \,\Delta T_{21}  + v_1^\lambda =
      {2(\lambda_2- \lambda_1) \over \Delta T_{21}}  - v_1^\lambda.
\end{equation}

We find that a comparison of $v_2^\lambda$ and $(v^\lambda_2)^\ast$ can further decrease
the number of false tracks (recall \S\ref{sec:falsev}); with tolerances of $\Delta v^\lambda < 0.3$ deg/day and
$\Delta v^\beta < 0.07$ deg/day (applied simultaneously for both coordinates as an
elliptical condition), the reduction is about a factor of 50 (for $v_{max}$ = 1 deg day$^{-1}$),
with only a minimal impact on the sample completeness. Therefore, depending on $\Delta k$, the number of tracklet
pairs per trial tracklet  to continue processing ranges from $\sim$4 for $\Delta k$=7 days to
$\sim$120 for $\Delta k$=21 days. When added over all possible pairs of nights (with
$T_{revisit}=3$ days), the total number of candidate tracklet pairs normalized by
the number of tracklets per night ranges from 350 for $N_w=15$ days to 13,400 for $N_w=30$ days.
Therefore, the following, more involved, selection steps need to be executed for
no more than about $10^{10}$ tracklet pairs (for $N_w=30$ days; and only for $3\times10^{8}$
pairs when $N_w=15$ days). These numbers are significantly lower than the naive estimate of
10$^{15}$ ($10^3\times10^6\times10^6$).

We note that in steady-state processing, the new candidate tracklet pairs need to
be evaluated only for pairs of nights where the second night is the penultimate
night in the search window (all other combinations will have been already computed
on previous days). Because the caching of results from previous night is not
yet implemented in MOPS, we don't account for this reduction (of about a factor of
3 to 6) in the analysis presented here.

Given the acceleration estimate from eq.~\ref{eq:accPred}, the position of the third
tracklet can be predicted from
\begin{equation}
\label{eq:lambdaPred3}
  \lambda_3^\ast = \frac{1}{2} a^\lambda \, \Delta T_{32}^2 + v_2^\lambda \, \Delta T_{32} + \lambda_2.
\end{equation}
Similarly to eq~\ref{eq:matching1}, an approximately constant selection completeness
can be achieved using
\begin{equation}
\label{eq:matching2}
                r_\beta^{max} = B \, \Delta k^{1.5}.
\end{equation}
with $B=0.2$ arcmin, and $r_\lambda^{max} = 5 \, r_\beta^{max}$. Note that the search
area is now 25 times smaller than in the first case, thanks to parabolic rather
than linear extrapolation. Therefore, the number of matched tracklets per single
trial tracklet pair is
\begin{equation}
\label{eq:Nmdk3}
     N^P_{match}(\Delta k) = 0.078 \, \left({1 \, {\rm day} \over T_{revisit}}\right) \,
                    \left( \rho_{tracklet}  \over 450 \, {\rm deg}^{-2} \right) \, (\Delta k)^3,
\end{equation}
and the expected number of matches ranges from 9 for $\Delta k$=7 days to
$\sim$240 for $\Delta k$=21 days.

%####
The total number of candidate tracks per single trial tracklet, for all possible 3-night
combinations (where the third night is the last night in the search window) is
\begin{equation}
\label{eq:Ntt}
   N_{tracklet}^{tracks} = 3.2\times10^{-3} \, \left( \rho_{tracklet}  \over 450 \, {\rm deg}^{-2} \right)^2 \, \left({1 \,{\rm day}\over T_{revisit}}\right)^2 \, \sum_{k=2}^{N_w-1} \sum_{j=1}^{k-1} \, (k-j)^3 \, (N_w-k)^3.
\end{equation}
The normalization constant is equal to $1.96\times0.078\times(\Delta v^\lambda
\Delta v^\beta/v_{max}^2)$, where the term in parenthesis is $\sim0.02$. This
normalization gives the number of candidate tracks per search window normalized
by the number of tracklets per night (which is assumed constant for all nights).
The two terms in the sum reflect the multiplication of the number of matches found
in the first selection step (linear extrapolation from the first to the second night,
eq.~\ref{eq:Nmdk2}) and the number of matches found in the third selection step
(parabolic extrapolation from the second to the third night, eq.~\ref{eq:Nmdk3}).

The sums in eq.~\ref{eq:Ntt} can be evaluated analytically, but the result is cumbersome.
Using numerical evaluation (with $T_{revisit}=3$ days), we find that the number of candidate
tracks per tracklet ranges from $\sim600$ for $N_w=15$ days to $\sim174,000$ for
$N_w=30$ days ($N_{tracklet}^{tracks}$ scales with $N_w^8$ when the third night must be
the last night from the search window).
Therefore, the matching of the candidate third tracklet brings the number of candidate
tracks per search window to the range 10$^{9}$ - 10$^{11}$. The ratio of false candidate tracks
to true tracks is in the range 10$^{3}$ - 10$^{5}$, depending on $N_w$. Despite the reduction by
a factor of  about $10^{10}$ to $10^{12}$ from the combinatorial number of tracklet triplets,
another significant reduction is required before the IOD step can be attempted.


\begin{figure}[th!]
\centering
\vskip -2.6in
\includegraphics[width=0.95\textwidth]{figures/TrackSlide1}
\vskip -2.7in
\caption{The left panel shows a hypothetical asteroid trajectory as the curved blue line (with
the curvature greatly exaggerated). Three tracklets are shown by the black dots; the
two individual detections per tracklet are not shown, but are implied by the three measured
motion vectors ($v_1$, $v_2$ and $v_3$). The third tracklet illustrates a false tracklet.
The motion vector of the first tracklet is linearly extrapolated to the time of the second
tracklet and matched within the red ellipse. The first two tracklets are then used to
constrain parabolic extrapolation, shown by the green dotted line, which is then matched
within the green ellipse. Given three candidate tracklets, a parabola is fit to their positions
and predicted motion vectors are computed for each tracklet (the blue vectors in the middle
panel). This comparison is illustrated in the right panel, where the circle signifies the cutoff
velocity for forming tracklets. Note that the third tracklet has a measured velocity ($v_3$)
that is inconsistent with the predicted velocity ($v_3^\ast$). The consistency radii are discussed
in the text.
\label{fig:TrackSlide1}}
\end{figure}

%%%%   add references to the figure (left panel)  in above sections

\subsubsection{Using Tracklet Motion Vectors to Prune Candidate Tracks}

The positional matching described above didn't use strong constraints on the tracklet velocities
for the first and third tracklets. Since false tracklets (anything other than a true 
asteroid-asteroid pair of detections) have random velocities, 
velocity filtering can further reduce the number of false tracks.
With three candidate tracklets, a parabolic motion (see eq.~\ref{eq:parabola}) can be fit {\it without}
using tracklet velocities. This fit predicts the velocity of each tracklet from the first derivative of
the fit, which can then be compared to each measured velocity. Figure~\ref{fig:TrackSlide1}
illustrates a situation where, e.g., $v_3$ is inconsistent with the velocity predicted using such
parabolic fit.

The consistency tolerances are driven by the orbital curvature and acceleration, rather than
by the velocity measurement errors (velocities are measured with a precision of about 0.001
deg day$^{-1}$, see eq.~\ref{eq:sigv}). Analysis of the simulated samples described in
\S\ref{sec:MAFdetails} shows that velocity tolerances of $\delta v_\lambda^{max}$=0.12 deg day$^{-1}$
for the longitudinal component and $\delta v_\beta^{max}$=0.03 deg day$^{-1}$ for the latitudinal component
reject most false tracklets with only a few percent effect (per single discovery attempt) on overall
sample completeness.

The probability that a random false tracklet velocity will be consistent with a given expected
velocity is approximately (assuming a uniform distribution of false tracklet velocities)
\begin{equation}
        p_v =  { \delta v_\beta^{max} \, \delta v_\lambda^{max} \over  v_{max}^2 }
\end{equation}
With $v_{max} = 1$ deg day$^{-1}$, $p_v = 0.0036$. In reality, this probability is a bit smaller because
the false tracklet velocity distribution is not uniform (it is biased towards the velocity cutoff).
Finally, the probability that all three tracklets have velocities consistent with those
implied by their positions is $p_v^2 \sim 10^{-5}$ (not $p_v^3$ because $v_2$ was already
subjected to a fairly stringent cut, see eq.~\ref{eq:v2cut}; a more stringent cut here
would provide a reduction by about a factor of five, which we ignore)

This significant reduction in the number of candidate false tracks, due to filtering velocities
of the first and third tracklets, brings it to the range 10$^{4}$ - 10$^{6}$, which is smaller or at
most about the same as the number of true candidate tracks (on the Ecliptic). 

False tracks can also arise from incorrect matches of true (asteroid-asteroid) tracklets 
from different asteroids and these do not have a random distribution
of velocities. We discuss the impact of this below, when evaluating the results 
of numerical (instead of analytical) scaling evaluations, however the increase in
number of false tracks is small, as the density of false detections is much larger than
the density of asteroids.

With this final
reduction, the IOD step can be attempted with no more than about 10$^6$ candidate tracks per
search window.



\subsubsection{The scaling of the number of false tracks with the density of false detections}

The final number of false tracks can be computed using eq.~\ref{eq:Ntt}, after multiplying
the normalization constant by $p_v^2$ to account for velocity filtering. Numerical evaluation
shows that the expected number of false tracks per tracklet can be described as
\begin{equation}
\label{eq:NttFinalFit}
   N_{tracklet}^{tracks} = 2.4 \, \left({N_w\over 30\, {\rm day}}\right)^8 \, \left( \rho_{tracklet}  \over 450
        \, {\rm deg}^{-2} \right)^2 \, \left( {3\,{\rm day} \over T_{revisit}} \right)^2 \,
         \left( {1 \, {\rm deg} \, {\rm day}^{-1}  \over  v_{max} }\right)^6.
\end{equation}
Note the very steep dependence on $N_w$: the large power-law index (8) is a result of the two
powers of 3 under sum in eq.~\ref{eq:Ntt}, and the scaling of the number of three-night
combinations with $N_w^2$ from eq.~\ref{eq:N3n}. The scaling of $N^{falsetracks} = N_{tracklet}^{tracks} \, N_{tracklet}$
with the density of false positive detections is very steep, too. Since $N_{tracklet}$ and $\rho_{tracklet}$
are approximately proportional (in the limit $\rho_{ast}=0$) to $\rho_{FP}^2$ (see eq.~\ref{eq:NttFalse}),
the number of false candidate tracks approximately scales with $\rho_{FP}^6$. 
$N_{tracklet}^{tracks}$ scales with $v_{max}^{-6}$ because of velocity filtering for three tracklets,
with each of the three steps giving a $v_{max}^{-2}$ contribution). 
Since $N^{falsetracks}$ scales with $\rho_{tracklet}^3$, and $\rho_{tracklet}$ scales with $v_{max}^2$
(due to dependence of $A_S$ on $v_{max}$, see eq.~\ref{eq:AS}), $N^{falsetracks}$ is approximately 
independent of $v_{max}$ (but note that the number of intermediate filtering operations does 
depend on $v_{max}$). 

Without the limiting $\rho_{ast}=0$  approximation, eq.~\ref{eq:NttFalse} implies a shallower scaling
of the number of false candidate tracks, $N^{falsetracks}$ with $\rho_{FP}$.
We have determined numerically that the scaling of the number
of false candidate tracks per search window with the density of false detections in difference
images, as well as other relevant parameters, is well described by
\begin{equation}
\label{eq:falsetracks}
   N^{falsetracks} = 4.5 \times 10^6 \, \left( {N_w \over 30 \, {\rm day} } \right)^{8} \left( {\rho_{FP} \over 400 \, {\rm deg}^{-2} }\right)^{3.7}
    \left( {\Delta t  \over 30 \, {\rm min}} \right)^{2.7}
     \left( { 1 \, {\rm deg} \, {\rm day}^{-1} \over v_{max} }  \right)^{1.3}.
\end{equation}
This numerical approximation is only valid around above {\it fiducial values} of various 
parameters\footnote{For example, eq.~\ref{eq:falsetracks} cannot be used to conclude that
an infinitely large $v_{max}$ cutoff would result in no false tracks.}, 
and assumes $\rho_{ast}=100$ deg$^{-2}$ and $T_{revisit}=3$ days. With fiducial parameters, and 
when $\rho_{ast}=0$, the number of false tracklets per night is $\sim10^6$, and the number of false 
tracks per search window with $N_w=30$ days is about 550,000. For $N_w=15$ days,  the number 
of false tracks drops to $\sim2,000$. 

We note that the relevant quantity that determines the number of false candidate tracks is {\it not}
the ratio of false to real (asteroid) detections in difference images, but rather the overall number 
(and density) of false detections.

The scaling result given by eq.~\ref{eq:falsetracks} may prove useful when optimizing cadence
and search strategy, as well as for sizing the required computational resources. For example,
for the rate of 8,200 deg$^{-2}$ false detections from Pan-STARRS1, one would expect a factor
of $7\times10^4$ more false candidate tracks that discussed above (that is, about $10^{11}$).
Even with $N_w$=15 days, the predicted number of false candidate tracks remains of the
order 10$^9$.

Although MOPS algorithms operate in a different way, these analytic probabilistic considerations
explain why the number of candidate tracks produced in MOPS experiments stays approximately
the same (to within a factor of two) even when the number of input tracklets per night is increased
by about an order of magnitude. With $N_{tracklet}^{tracks} \sim2$, the number of  candidate
tracks (both true and false) per search window is about $N^{tracks} \sim 5\times10^6$ for $N_w= 30$ days,
that is, not overwhelmingly larger than the number of true tracks (500,000). In other words, the ratio of
false to true detections of 4:1 generates a ratio of false to true tracklets of 3:1 and a ratio of false to true
candidate tracks of 10:1 (for $N_w=15$ days and $\rho_{ast}=100$ deg$^{-2}$, the ratio of false to true
candidate tracks drops to below 5\%).

This similarity in the number of true and false candidate tracks is in good
agreement with the results of MOPS simulations\footnote{See the top left panel
in Figure 21 in \citet{LDM-156}} (though note that those simulations used more aggressive filtering
based on ``parabolic motion plus topocentric correction'' model, and thus obtained a factor of a few
lower counts of candidate tracks).

% this might be interesting, but it's not completed
% \section{Fallback survey strategies}\label{sec:AppFallback}

To investigate the effects of requiring triplets or quads of visits within a single night as part of the discovery criteria, instead of simply pairs, two additional simulated surveys were run where the WFD (but not the NES) requested visits in sequences of three or four, instead of just pairs. It is important to note that due to the behavior of the OpSim software, it is frequently the case that more than the requested minimum number of visits are received in a particular field on a given night. That is, even in the baseline minion\_1016 run, which requested pairs of visits in each night for the WFD and NES, it is often the case that three or four visits were obtained despite the smaller request. Thus, these additional runs provide an indication of the trade-offs in coverage and completeness that result from requesting more visits per field, per night, but the effects would be more pronounced if this behavior of the software did not exist. The simulated surveys enigma\_1281 and enigma\_1282 are the result of requesting triples and quads in the WFD proposal, with 2,052,029 and 2,033,431 WFD visits in these runs, respectively. 



Effects of pairs vs triples vs quads - minion\_1016 vs enigma\_1281 vs enigma\_1282.



\bibliography{neo_capabilities}
\end{document}



To Do:
1) authors
- ask Lori Allen to be coauthor (and if there are others from her team)
- ask OpSim crew to be coauthors?
- anyone else, e.g. Scott Daniel, Yusra, Peter, etc
