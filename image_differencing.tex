
\subsection{Intro and Test Data}

False positives have historically dominated over true transients in previous
surveys. \citep{goldstein15, kessler15}

Major impediment to linking detections of moving objects

Develop a conservative estimate of the FP rate in the LSST stack, using similar
sensors and the current software.

Data from Allen et al., offers a useful cadence. Not the same depth or seeing,
but sufficient to test the software capabilities. Stack has not been optimized
to run at production-levels yet, so this is the starting point of that effort,
not an upper limit on LSST's performance.

\subsection{Processing}

Some reference to LSST software heritage and version, circa Jan 2016.

Differencing two images, rather than a template vs image. Running force
photometry. Ingest into DB.

\subsection{Correlated Noise}

Histogram of stack-reported diaSource SNR, alongside force photometry SNR.

Because we are diffing two images, we know that the photometry of the diaSource
must be the same as the difference of the force photometry on the direct images.
Gives us independent check on SNR reporting.

Strong ramp up of sources at SNR 5-6, but order of magnitude more than should be
present due to Gaussian statistics.

Check with force photometry shows that many of these cannot actually have the
high SNRs reported, more like 4-sigma events. FP values consistent with Gaussian
statistics.

Issue is tracking correlated noise after PSF-matching the science image. Known
issue, ref Paul \& Gene somehow (must be more references too).

For our purposes, we will estimate SNR using force phot. Several choices of long
term solutions; implementations are being studied now.

\subsection{False-Positive Results}

Using corrected SNR cut, ~1000 diaSources per square degree (only positive,
ignoring negative). ~500 on best fields.

Many of these are poorly-subtracted stars. To estimate rates of candidate moving
objects, exclude all diaSources that had detections in both direct images (at
the same position). These are still FPs, but they are someone else's FPs.
Resulting levels are ~350 per sq deg. Irreducable noise level is 33/sq deg.

Results from trying to make tracklets, hopefully?? On visual inspection, 25\%
look like junk, 25\% looked like good detections, and the rest were ambiguous
because low SNR. SNR power law exponent is $\sim -2.5$.

Very few detections around bright stars; mostly well-masked by the code. Few
large scale detected artifacts, but not a very big sample of images.

How detailed of a recipe are we providing here; should it be the same level of
detail as we sent to JPL?
