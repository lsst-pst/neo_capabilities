\section{LSST Strategy for Discovering Solar System Objects}

We briefly describe the LSST system design and observing strategy, and discuss in more
detail image processing and moving object detection.

\subsection{A Brief Overview of LSST  Design}

LSST will be a large, wide-field ground-based optical telescope system
designed to obtain multiple images covering the sky that is visible
from Cerro Pach\'{o}n in Northern Chile. The current baseline design,
with an 8.4m (6.7m effective) primary mirror, a 9.6 deg$^2$ field of
view, and a 3.2 Gigapixel camera, will allow about 10,000 square
degrees of sky to be covered every night, with typical 5$\sigma$ depth
for point sources of $r\sim24.5$ mag (AB). The system is designed to yield
high image quality (with a median delivered seeing in the $r$ band of
about 0.8 arcsec) as well as superb astrometric  and photometric
accuracy\footnote{For detailed specifications, please see the LSST
Science Requirements Document, \url{http://ls.st/srd}}. The total survey
area will include $\sim$30,000 deg$^2$ with $\delta<+34.5^\circ$, and
will be imaged multiple times in six bands, $ugrizy$, covering the
wavelength range 320--1050 nm. The project is scheduled to  begin the
regular survey operations at the start of next decade.

LSST will be operated in a fully automated survey mode. About 90\% of the
observing time will be devoted to a deep-wide-fast survey mode which will
uniformly observe a 18,000 deg$^2$ region about 1000 times (summed over
all six bands) during the anticipated 10 years of operations, and yield a coadded map
to a depth of $r\sim27.5$. These data will result in catalogs including about
$40$ billion stars and galaxies, that will serve the majority of the
primary science programs. The remaining 10\% of the observing time
will be allocated to special projects such as a Very Deep and Fast
time domain survey\footnote{Informally known as ``Deep Drilling Fields".}.



\subsection{LSST Observing Strategy}

As designed and funded (by the U.S National Science Foundation and
the Department of Energy), LSST is primarily a science-driven mission.
The LSST is designed to achieve goals set by four main science themes:
\begin{enumerate}
\item Probing Dark Energy and Dark Matter;
\item Taking an Inventory of the Solar System;
\item Exploring the Transient Optical Sky;
\item Mapping the Milky Way.
\end{enumerate}
Each of these four themes itself encompasses a variety of analyses, with
varying sensitivity to instrumental and system parameters. These themes
fully exercise the technical capabilities of the system, such as photometric
and astrometric accuracy and image quality.

The current baseline survey strategy is designed to maximize the overall science returns, including
Solar System science, rather just than the completeness of NEO/PHAs brighter than $H=22$ (though the
two goals are highly interrelated). Discovering and linking objects in the Solar System
moving with a wide range of apparent velocities (from several degrees per day for
NEOs to a few arc seconds per day for the most distant TNOs) places strong
constraints on the cadence of observations. The baseline strategy requires closely
spaced pairs of observations, two or preferably three times per lunation. The visit
exposure time is set to 30 seconds to minimize the effects of trailing for the majority of
moving objects. The images are well sampled to enable accurate astrometry, with
anticipated absolute accuracy of at least 0.1 arcsec (and possibly an order of magnitude
better when calibrated with upcoming Gaia's dataset, \citealt{Gaia}).

LSST observations can be simulated using the LSST Operations Simulator tool
\citep[OpSim,][]{delgado14}. OpSim runs a survey simulation with given
science-driven desirables,  a software model of the telescope and its control
system, and models of weather and other  environmental variables. The output of
such a simulation is an ``observation history'', which  is a record of times,
pointings, used filter, and associated environmental data and telescope
activities throughout the simulated survey.  This history can be examined using
the LSST  Metrics Analysis Framework tool \citep[MAF,][]{jones14} to assess the
efficacy of the simulated survey for any particular science goal or
interest\footnote{For examples of such analysis, see \url{http://ls.st/xpr}}.


\subsection{LSST Baseline Survey Strategy}

As the system understanding improves, the baseline survey strategy and the telescope model
are updated, generally on a yearly schedule. The current
reference baseline survey is known as {\it minion\_1016}. It includes 2.4
million visits collected over 10 years, with 85\% of the observing time spent on the
main survey and the rest on various specialized programs. The median number of visits
{\it per night} is 816, with 3,026 observing nights. The median airmass is 1.23 (the
minimum attainable altitude for the LSST telescope is 20 deg.). In the $r$ band, the median
seeing (FWHM) is 0.81 arcsec, and the median $5\sigma$ depth for point sources is 24.16
(using the best current estimate of the fiducial depth at airmass of one, $m_5(r)$=24.39).

There are a few known problems with this simulation, including twilight sky brightness
estimates that are too bright, the moon avoidance that is not as aggressive as it could be,
and observations that are biased towards west, away from the meridian. The implied impact
of these shortcomings on NEO completeness estimates is a few percent (the performance
of this simulated cadence in NEO context is discussed in detail in \S5). An improved simulation,
that will presumably rectify these problems, will become available by the end of 2017.


\subsection{Overview of LSST  Data Management and Image Processing}

The images acquired by the LSST Camera will be processed by LSST Data Management
software \citep{juric15} to a) detect and characterize imaged
astrophysical sources and b) detect and characterize temporal changes
in the LSST-observed universe. The results of that processing will be
reduced images, catalogs of detected objects and their measured properties, and
prompt alerts to ``events'' -- changes in astrophysical scenery discovered by differencing
incoming images against older, deeper, images of the sky in the same direction ({\em
templates}). More details about the main algorithms and pipeline design are available
in Appendix~\ref{sec:AppA}.

LSST will use two methods to detect moving objects in difference images:
\begin{enumerate}
\item Detecting trailed motion on the sky: objects trailed by more
  than 2 PSF widths (corresponding to motion faster than about 1
  deg/day) will be easily detectable as trailed.  Two trailed
  detections within 20--60 minutes in a single night will be
  sufficient to identify an object as an NEO candidate,
\item Inter-night linking of pairs of detections from the same night: this technique will
  recover objects moving too slow to be measurably elongated in a single exposure.
\end{enumerate}

We note that sources detected in difference images (\DIASources in LSST parlance, see Appendix~\ref{sec:AppA})
will also include false detections, colloquially known as {\it false positives}.
In addition to false positives due to instrumental artifacts and software glitches,
in this context they will also include detections of true astrophysical transients
(e.g. gamma-ray burst afterglow) that will not be associated with static sources
(e.g. stars and galaxies). Estimates of expected false positive rates are discussed
in \S\ref{sec:imDiff}.



\subsection{The Basic Strategy for Linking Detections into Orbits}

The LSST strategy for linking detections into orbits assumes the following main steps:
\begin{enumerate}
\item Detections in difference images (obtained during the same night), that do not
         have a nearby static object (e.g. variable stars) within a small exclusion radius
         (a fraction of an arcsecond, but possibly larger for brighter stars), are linked into tracklets. There will be of the order
         a million tracklets per observing night (see \S\ref{sec:tracklets}).
\item At least three tracklets obtained in a 15-30 day wide window are linked into
         candidate tracks, using kd-trees and pre-filtering steps based on tracklets' positions
         and motion vectors (see \S\ref{sec:tracks}). These pre-filtering steps result in
         about the same number of false tracks as true tracks on the Ecliptic (of the order
         a million), with the completeness depending on population (e.g. main-belts
         asteroids vs. NEOs) and chosen tunable pre-filtering parameters (generally well above 90\%).
\item Candidate tracks are then filtered further (pruned from false tracks) using the initial orbit
         determination (IOD) and measurement uncertainties for the positions of all six
         detections. Given typical astrometric errors (0.15 arcsec for the faintest objects,
         see \S\ref{sec:astromerrors})
         and a fiducial tracklet density of 450 deg$^{-2}$, the probability of a chance alignment
         of a {\it single} false detection within $5\sigma$ astrometric tolerance from a fiducial
         10 degree long trajectory is only about 10$^{-4}$. For a trajectory constrained by 3 detections, the
         probability that 3 more detections will line up by chance is thus about 10$^{-12}$ (compared to $\sim 10^7$ detections per night). Hence,
         the number of false tracks which could pass IOD-based filtering is negligible, as well as
         the incompleteness induced by this step. Indeed, the only reason for track pre-filtering
         in the second step is to decrease the number of candidate tracks that are processed
         by the more computationally expensive IOD.
\end{enumerate}

False positives, be it false detections, false tracklets or false tracks, are assumed not be
an issue because IOD will efficiently and reliably filter out false tracks due to high-accuracy
astrometry and well-understood simple Keplerian model predictions. Therefore, the
essential question is whether the resulting number of false tracks, and the corresponding
IOD step, can be handled with available computing resources. As discussed in detail in \S\ref{sec:mops}
and in Appendix~\ref{sec:appMOPS}, even with computing resources quite modest compared to the
sizing of LSST data processing system, the handling of the three steps above is not an issue.
