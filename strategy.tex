\section{LSST Strategy for Discovering Solar System Objects} 

We briefly describe the LSST system design and observing strategy, and discuss in more
detail image processing and moving object detection. 

\subsection{A Brief Overview of LSST  Design} 

LSST will be a large, wide-field ground-based optical telescope system
designed to obtain multiple images covering the sky that is visible
from Cerro Pach\'{o}n in Northern Chile. The current baseline design,
with an 8.4m (6.7m effective) primary mirror, a 9.6 deg$^2$ field of
view, and a 3.2 Gigapixel camera, will allow about 10,000 square
degrees of sky to be covered every night, with typical 5$\sigma$ depth 
for point sources of $r\sim24.5$ (AB). The system is designed to yield 
high image quality (the median delivered seeing in the $r$ band of 
about 0.8 arcsec) as well as superb astrometric  and photometric 
accuracy\footnote{For detailed specifications, please see the LSST
Science Requirements Document, http://ls.st/srd}. The total survey
area will include $\sim$30,000 deg$^2$ with $\delta<+34.5^\circ$, and 
will be imaged multiple times in six bands, $ugrizy$, covering the 
wavelength range 320--1050 nm. The project is scheduled to  begin the 
regular survey operations at the start of next decade. 

LSST will be operated in fully automated survey mode. About 90\% of the 
observing time will be devoted to a deep-wide-fast survey mode which will 
uniformly observe a 18,000 deg$^2$ region about 1000 times (summed over 
all six bands) during the anticipated 10 years of operations, and yield a coadded map 
to $r\sim27.5$. These data will result in catalogs including about
$40$ billion stars and galaxies, that will serve the majority of the
primary science programs. The remaining 10\% of the observing time
will be allocated to special projects such as a Very Deep and Fast
time domain survey\footnote{Informally known as ``Deep Drilling Fields".}.



\subsection{LSST Observing Strategy} 

As designed and funded (by the U.S National Science Foundation and
Department of Energy), LSST is primarily a science-driven mission. 
The LSST is designed to achieve goals set by four main science themes:
\begin{enumerate}
\item Probing Dark Energy and Dark Matter;
\item Taking an Inventory of the Solar System;
\item Exploring the Transient Optical Sky;
\item Mapping the Milky Way.
\end{enumerate}
Each of these four themes itself encompasses a variety of analyses, with 
varying sensitivity to instrumental and system parameters. These themes 
fully exercise the technical capabilities of the system, such as photometric 
and astrometric accuracy and image quality. 

The current baseline survey strategy is planned to maximize the overall science returns, including 
Solar System science, rather than NEO/PHA discovery completeness (though the 
two goals are highly interrelated). Discovering and linking objects in the Solar System 
moving with a wide range of apparent velocities (from several degrees per day for 
NEOs to a few arc seconds per day for the most distant TNOs) places strong 
constraints on the cadence of observations. The baseline strategy requires closely 
spaced pairs of observations, two or preferably three times per lunation. The visit
exposure time is set to 30 seconds to minimize the effects of trailing for the majority of 
moving objects. The images are well sampled to enable accurate astrometry, with 
anticipated absolute accuracy of at least 0.1 arcsec.

LSST observations can be simulated using the LSST Operations Simulator tool (OpSim, XXX give
reference here). OpSim runs a survey simulation with given science-driven desirables, 
a software model of the telescope and its control system, and models of weather and other 
environmental variables. The output of such a simulation is an ``observation history'', which 
is a record of times, pointings and a record of associated environmental data and telescope  
activities throughout the simulated survey.  This history can be examined to assess  
the efficacy of the simulated survey for any particular science goal or 
interest\footnote{For examples of such analysis, see http://ops2.lsst.org:8080 XXX shorten this URL!}. 


\subsection{LSST Baseline Survey Strategy}

As the system understanding improves, the baseline survey strategy and the telescope model gets updated, generally on a yearly schedule. The current reference baseline survey is known as {\it minion\_1016}. It includes 2.4
million visits collected over 10 years, with 85\% of the observing time spent on the 
main survey and the rest on various specialized programs. The median number of visits
{\it per night} is 816, with 3,026 observing nights. The median airmass is 1.23 (the
minimum altitude for LSST telescope if 20 deg.). In the $r$ band, the median seeing 
(FWHM) is 0.81 arcsec, and the median $5\sigma$ depth for point sources is 24.16 
(using the best current estimate of the fiducial depth at airmass of one of 24.39). 

There are a few known problems with this simulation, including twilight sky brightness
estimates that are too bright, the moon avoidance is not as aggressive as it could be,
and observations are biased towards west, away from the meridian. An improved 
simulation will become available by the end of 2017. 

The performance of this simulated cadence in NEO context is discussed in detail in \S5. 


\subsection{Overview of LSST  Data Management and Image Processing} 

The images acquired by the LSST Camera will be processed by LSST Data Management
software \cite{juric15} to a) detect and characterize imaged
astrophysical sources and b) detect and characterize temporal changes
in the LSST-observed universe. The results of that processing will be
reduced images, catalogs of detected objects and their measured properties, and 
prompt alerts to ``events'' -- changes in astrophysical scenery discovered by differencing 
incoming images against older, deeper, images of the sky in the same direction ({\em
templates}). More details about the main algorithms and pipeline design are available
in Appendix~\ref{sec:AppA}. 

LSST will use two methods to detect moving objects in difference images: 
\begin{enumerate}
\item Detecting trailed motion on the sky: objects trailed by more
  than 2 PSF widths (corresponding to motion faster than about 1
  deg/day) will be easily detectable as trailed.  Two trailed
  detections within 30--60 minutes in a single night will be
  sufficient to identify an object as an NEO candidate,
\item Inter-night linking of pairs: this technique will recover
  objects moving too slow enough to be measurably elongated in 
  a single exposure. 
\end{enumerate} 


Sources detected in difference images (DIASources in LSST parlance, see Appendix~\ref{sec:AppA})
will include false detections, colloquially known as {\it false positives}. 

