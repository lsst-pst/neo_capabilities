
\section{Introduction}

The small-body populations in the Solar System, such as asteroids, trans-Neptunian objects (TNOs)
and comets, are remnants of its early assembly. Collisions in the main asteroid belt between Mars and
Jupiter still occur, and occasionally eject objects on orbits that may place them on a collision course
with Earth. About 20\% of this near-Earth Object (NEO) population pass sufficiently close to Earth orbit such that
orbital perturbations with time scales of a century can lead to intersections and the possibility of collision. These objects that pass within 0.05 AU of Earth's orbit are termed potentially hazardous asteroids (PHAs).
In order to improve the quantitative understanding of this hazard, in December 2005 the U.S. Congress
directed\footnote{National Aeronautics and Space Administration Authorization Act of 2005 (Public Law 109-155), January 4, 2005, Section 321, George E. Brown, Jr. Near-Earth Object Survey Act} NASA to implement a NEO survey that
would catalog 90\% of NEOs with diameters larger than 140 meters by 2020 (known as the George
E. Brown, Jr.\ mandate). It is estimated that there are approximately 30,000 such objects 
\citep{2012ApJ...752..110M, harris15, 2016Natur.530..303G, 2017Icar..284..114S}, with just over 7,500 currently known.
For a compendium of additional information about NEOs and PHAs and an up-to-date summary of
discovery progress, see NASA's CNEO webpage\footnote{\url{https://cneos.jpl.nasa.gov/}}.

The completeness level set by the Congressional mandate could be accomplished with a 10-meter-class
ground-based optical telescope, equipped with a multi-gigapixel camera and a sophisticated and robust data
processing system \citep[see NASA-commissioned reports by ][]{stokes03,shapiro10}. 
The Large Synoptic Survey Telescope\footnote{http://lsst.org} (LSST), currently being
constructed, approaches such a system. A concise LSST system description, discussion of science drivers, and other
information, are available in \cite{LSSToverview}.

The LSST baseline strategy for discovering Solar System
objects is predicated on two observations of the same field per night, spaced by a few tens of minutes, and
a revisit of the same field with another pair of observations within a few days. The main reason for two
observations per night is to help association of observations of the same object from different nights,
as follows. The typical distance between two nearby asteroids on the Ecliptic, at the faint fluxes probed by
LSST, is a few arcminutes (object counts are dominated by main-belt asteroids). Typical asteroid motion
during several days is larger (of the order a degree or more) and thus, without additional information,
detections of individual objects are ``scrambled''. However, with two detections per night, the motion
vector can be estimated. The motion vector makes the linking problem much easier because
positions from one night can be approximately extrapolated to future (or past) nights. The predicted
position's uncertainty is typically of the order of several arcminutes, rather than a degree, which effectively
``de-scrambles'' detections from different nights (for a detailed discussion of this algorithm, see \citealt{kubica07} as well as Appendix~\ref{sec:appMOPS} for a theoretical derivation of expected scalings).

Early simulations of LSST performance presented by \cite{IvezicNEO2007} showed that the 10-year baseline
cadence would result in 75\% completeness for PHAs greater than 140 m (more  precisely, for PHAs with
$H<22$). They also suggested that with additional optimizations of the
observing cadence, LSST could achieve 90\% completeness. An example of such an optimization was discussed
by \cite{LSSToverview} who reported that, to reach 90\% completeness, about 15\% of observing time would
have to be dedicated to NEOs, and the survey would have to run for 12 years. More recently, estimates of
LSST yields have been revisited by \citet{GMS2016} (predicted PHA completeness of 62\% for LSST alone) and \citet{VeresChesley2017neo} (predicted PHA completeness of 65\% for LSST alone).
%% From the overview paper:
%% - the LSST baseline cadence provides orbits for 82% of PHAs larger than 140 meters after 10 years of operations
%% - 84% completeness with minor changes to the cadence (5% of time for NEO-optimized observations)
%% - 90% completeness with major changes to the cadence (15% of time for NEO-optimized observations and 12 years)
The latest LSST simulation results, presented in Section~\ref{sec:opsim}, yielded a completeness of $\sim$66\% for
PHAs with $H<22$, using the current 10-year baseline survey. The differences in reported completeness
between these studies are attributable to differences in the simulated NEO populations and other modeling
details (such as an improved understanding of the hardware and updated cadence and sky brightness models).
This is discussed further in \S\ref{sec:other}.

These completeness estimates are based on an implicit assumption that 3 pairs of observations
obtained within a 15-30 day wide window are sufficient to recognize that these observations belong
to the same object, and to estimate its orbital parameters (the same criterion has been used in NASA
studies\footnote{See \url{http://neo.jpl.nasa.gov/neo/report2007.html}}).
This so-called linking of individual detections into plausible orbital tracks
will be performed using a special-purpose code referred to as the Moving Object
Processing System (MOPS).

The capability and effectiveness of LSST for discovering moving objects has been questioned (e.g., \citealt{GMS2016}) on
two grounds:
\begin{itemize}
\item A large number of false detections due to problems with image differencing software may
make linking problem prohibitively hard for MOPS. In particular, this objection is motivated by the experience
from extant surveys, such as Pan-STARRS1 \citep{denneau13} or the Catalina Sky Survey \citep{2003DPS....35.3604L}.
\item Modifications of LSST baseline cadence, including image depth, sky coverage and cadence,
required to reach 90\% completeness level, have not yet been explicitly demonstrated using detailed
operations simulations, and made available to the community.
\end{itemize}
We aim to address these critiques here: the two major questions addressed by our study can be informally
stated as ``Will MOPS work?'' and ``If MOPS works, what fraction of  NEOs will LSST discover?''.

We use a combination of sophisticated simulations and real datasets to address these questions.
The main analysis components presented here include:
\begin{enumerate}
\item Analysis of the performance of prototype LSST image differencing software, with emphasis on the rate and
    properties of false detections (so-called ``false positives''), using DECam imaging data.
\item Analysis of the linking of asteroid detections in the presence of a large number of false positives, using MOPS
         and simulated observations.
\item Analysis of a large number of modified observing cadence simulations, coupled with NEO population
          models, to forecast discovery rates.
\end{enumerate}

In \S2 we provide a brief overview of LSST and its strategy for discovering moving Solar
System objects. We measure the performance of prototype LSST image differencing pipeline
in \S3, and MOPS performance in \S4. Modifications of the baseline cadence designed to
boost NEO/PHA completeness are examined in \S5.

We conclude that i) LSST implementation of MOPS can cope with the anticipated false detection rates
in LSST difference images, and that ii) the NEO discovery performance of the LSST baseline cadence
can be appreciably boosted by adequate modifications of the observing strategy. These findings are summarized and discussed in \S6.
