
\section{Introduction}

XXX This is so totally work in progress. ZI push-ed it just as a backup... XXX

Main parts:
\begin{itemize}
\item NEO impacts as a concern; the Brown mandate, NASA panels, introduce LSST 
\item LSST defaults and claimed performance in older publications
\item informal concerns by the community, GMS paper
\item study by the JPL team, technical support from LSST (assuming baseline cadence)
\item exploration of baseline cadence modifications designed to boost NEO completeness 
\item Outline of this paper
\end{itemize} 



The small-body populations in the Solar System, such as asteroids, trans-Neptunian objects (TNOs) 
and comets, are remnants of its early assembly. Collisions in the main asteroid belt between Mars and 
Jupiter still occur, and occasionally eject objects on orbits that may place them on a collision course 
with Earth. About 20\% of this near-Earth Object (NEO) population, the so-called potentially hazardous 
asteroids (PHAs), are in orbits that pass sufficiently close to Earth's orbit, to within 0.05 AU, that 
perturbations with time scales of a century can lead to intersections and the possibility of collision. 
In December 2005, the U.S. Congress directed\footnote{For details see http://neo.jpl.nasa.gov/neo/report2007.html} 
NASA to implement a NEO survey that would catalog 90\% of NEOs with diameters larger than 140 meters 
by 2020 (the George E. Brown, Jr. mandate). For a compendium of information about NEOs and PHAs 
and an up-to-date summary of discovery progress, see NASA's NEO webpage\footnote{http://neo.jpl.nasa.gov/neo/}. 

The completeness level set by Congressional mandate can be fulfilled with a 10-meter-class ground-based
telescope equipped with a multi-gigapixel camera, and a sophisticated and robust data processing system. 
The Large Synoptic Survey Telescope (LSST), currently being constructed, is such a system (for a concise
system description, science drivers and other information, see \citep{LSSToverview}). Early simulations of 
LSST performance presented by \cite{IvezicNEO2007} showed that the 10-year baseline cadence would 
result in 75\% completeness for PHAs greater than 140 m (more precisely, for PHAs with $H<22$; see 
\S~XX for further discussion). They also suggested that with additional optimization of the observing cadence, 
LSST could achieve 90\% completeness. Such optimization was discussed by \cite{LSSToverview} who
reported that, to reach 90\% completeness, about 15\% of observing time would have to be dedicated to NEOs
and the survey would have to run for 12 years.  
%% From the overview paper: 
%% - the LSST baseline cadence provides orbits for 82% of PHAs larger than 140 meters after 10 years of operations
%% - 84% completeness with minor changes to the cadence (5% of time for NEO-optimized observations)
%% - 90% completeness with major changes to the cadence (15% of time for NEO-optimized observations and 12 years)
The latest LSST simulation results, presented in \cite{JJI2016}, yielded a completeness of $\sim$72\% for
PHAs with $H<22$ using the current 10-year baseline survey. The minor difference compared to older
studies is attributable to the differences in simulated NEO populations and other modeling details. 


\cite{JPLstudy} described a new study, related to this work. Their preliminary results indicate a completeness
of $\sim$65\% for NEOs with $H<22$. The difference compared to \cite{JJI2016} result (72\%) is due 
to XXX (true?): PHA vs. NEO, possible slope of the size distribution. 


We're going to evaluate moving object detection capabilities with
LSST, comparing performance of our current prototype pipelines with 
the required performance during operations.  The two main goals are
to demonstrate that i) MOPS can cope with false detection in image differences,
and that ii) the NEO detection performance of the LSST baseline cadence can be
further boosted by adequate modifications 


From JPL paper -- which gives a concise summary of the main simulation aspects. 

The LSST baseline survey cadence relies primarily on single night pairs of detections, 
with roughly 30 minutes between the elements of a detection pair. These pairs form 
what are known in MOPS parlance as tracklets, and sets of tracklets are linked across 
multiple nights to form tracks, which can then be sent to the final step, which is orbit 
determination. The strategy of using pairs is an aggressive and potentially fragile
approach, but theoretically represents the most productive NEO search with the minimum 
impact on other LSST science drivers. An alternative to visit each field three times per 
night to form tracklets from triplets of detections may prove more robust, but likely 
with a penalty of reduced performance. One of our study objectives is to understand the
tradeoffs between these two approaches.

The two major questions to be addressed by our study can be informally stated as 
``Will MOPS work?'' and ``If MOPS works what fraction of NEOs will LSST discover?''. 

Main problems:
\begin{enumerate}
\item Linking large number of detections in the presence of false positives (false detections due to problems 
with image differencing software). 
\item Adequacy of data, including image depth, sky coverage and cadence, to reach the required 
completeness level. 
\end{enumerate} 

Therefore, the main analysis components to check are: 
\begin{enumerate}
\item The performance of image differencing, with emphasis on the rate and properties of 
   false detections 
\item Linking large number of detections in the presence of false positives 
\item Observing cadence simulations coupled with NEO population models to forecast 
        discovery rates 
\end{enumerate} 



\cite[hereafter GMS]{GMS2016} reported different NEO completeness levels than
published by the LSST team in 2007 and 2014 . There are three main 
reasons why the GMS results differ:
\begin{enumerate}
\item GMS used a different realization of the LSST baseline survey
\item GMS used different synthetic NEO populations to evaluate completeness
\item GMS {\it redefined} the completeness limit from the commonly
  used $H<22$ criterion to an albedo-dependent value of $H$ limit (which
  attempts to directly model the $D>140$ m size cut)
\end{enumerate}

Regarding the last point, GMS found that the completeness drops by 5\%
when $H<22$ criterion is replaced by $D>140$ m criterion. Regarding 
the first point, GMS results can be more meaningfully compared to an LSST
study by \cite{JJI2016}, who used the same simulated cadence. After accounting for 
the $H<22$ vs. $D>140$ m methodological difference of 5\%, GMS obtained a 
completeness of 67\% using 3 pairs in 12 nights (for simulated cadence {\it enigma\_1189}), while Jones et al. 
study obtained $\sim$73\% using 3 pairs in 15 nights (for simulated cadence {\it minion\_1016}, which is 
statistically very similar to {\it enigma\_1189}). This difference 
of $\sim$5\% is attributable to the differences in simulated NEO
populations and other modeling details. In summary, GMS find the NEO
completeness in the range $\sim$60\% to $\sim$70\% for the LSST
baseline cadence. The variation is due to different NEO populations,
different NEO detection criteria, and other specifics. When accounting 
for different choices of simulation parameters, their results are
consistent with the results published by the LSST team. 

But: how much higher can the completeness be pushed with cadence
modifications optimized for NEOs? 

In Observing Strategy white paper: fig. 3.4 gives 73.4\% for PHAs with 
$H<22$ and {\it minion\_1016}, using 3 pairs in 15 nights. 


\begin{enumerate} 
\item Control and quantify the rate of (false positive) detections
\item Software (and computational capacity) capable of inter-night linking of detections given the expected rates
\item Quantify the discovery yields (and their robustness) under those assumptions
\end{enumerate} 


From Mario's talk to NASA:

LSST will detect variability (motion and flux variability) by
differencing each incoming image against a deep template.
Sources will be detected at an S/N=5 threshold (see Appendix A). 

We expect on average about 1,000 per sq. deg. astrophysical, real,
detections, including up to about 500 asteroids per sq. deg on the 
Ecliptic (for scale, the LSST field of view is about 10 sq. deg., with 
about 20 4kx4k CCDs per sq. deg.)

We also expect a false-positive detections due to random
fluctuations in the background at a level of about 200 per
sq. deg. (all at the faint end).  XXX check Colin's report 

However, historically surveys have reported factors of 10 to 500 times
more (depending on the survey; see \citep{denneau13};
\citep{goldstein15} ). 
Those additional false positive
detections are due to systematic effects: 
\begin{itemize} 
\item Camera and telescope artifacts
\item Imperfect image subtractions
\item Cosmic rays
\end{itemize} 

For a ``menagerie'' of artifacts (with amusing names such as 
{\it chocolate chip cookies, frisbee, piano, arrowhead, UFO}), from
Pan-STARRS, see Fig.~17 in \cite{denneau13}. 


``Many of the false detections are easily explained as internal
reflections, ghosts, or other well-understood image artifacts,...''


Learning from PS1 Experience: PanSTARRS was a first generation
experiment. Over the past decade, subsequent surveys (including LSST) 
have learned tremendously from the PS1 experience. There are surveys 
running today which have largely solved the key problems that PS1 has encountered.
These are recent developments driven largely by extragalactic and
transient science cases. They are not yet well known beyond those
communities and reporting on those developments is additional
motivation for this paper. 
 
Major improvements to hardware include CCDs with significantly fewer 
artifacts (e.g. DECam, see below; LSST) and optical systems designed to
minimize ghosting and internal reflections (e.g. LSST). 

Improvements to the software include advanced image differencing
pipelines (e.g., PTFIDE for the Palomar Transient Factory and the
Zwicky Transient Facility) and various machine learning classifiers
for filtering false positives (see below). 

DECam: \cite{goldstein15} 

The Dark Energy Survey (DES) is an optical/near-infrared survey that
aims to probe the dynamics of the expansion of the universe and the
growth of large scale structure by imaging 5,000 sq. deg. of the
southern sky. It is technologically very similar to LSST with
\begin{itemize}
\item 520 Mpix camera, 62 mosaicked chips (deep depleted devices)
\item 3 sq.deg. field of view, same filter bands as LSST
\item Single-exposure depths comparable to LSST
\item Includes a supernova search program which employs image
differencing methods analogous to LSST’s  and detects objects at the 
same effective signal-to-noise ratio as LSST (S/N=5)
\end{itemize} 

The false positives in DECam data are morphologically much simpler
(compare Fig.~1 in \citep{goldstein15} to Fig.~17 in \citep{denneau13})
than those in Pan-STARRS, and thus are much more amenable to automated 
screening using machine learning methods. Using a Random Forest 
classifier, \cite{goldstein15} cleaned their sample from having a 
raw false detection rate of 13:1 to a filtered rate of 1:3. This performance
is already within the acceptable range for LSST performance goals. 


{\bf Need to refer to section by Colin.} 

LSST will use two methods to detect moving objects
\begin{enumerate}
\item Detecting trailed motion on the sky:  objects trailed by more
  than 2 PSF widths (corresponding to motion faster than about 1
  deg/day) will be easily detectable as trailed.  Two trailed
  detections within 30--60 minutes in a single night will be
  sufficient to identify an object as an NEO candidate,
\item Inter-night linking of pairs: this technique will recover
  objects moving too slow enough to be measurably elongated in 
  a single exposure. 
\end{enumerate} 

MOPS: Given the expected false-positive rates demonstrated by
\cite{goldstein15} and in Colin's  section, LSST MOPS linking will
be possible. This has already been shown by the PanSTARRS project 
with simulations performed for the PS4 system\footnote{PanSTARRS 
PS1 experience does not contradict this conclusion. In addition to 
hardware issues, PS1 was only 1/4 of the assumed system (see 
\citep{denneau13} for more details). }, which is in this
respect equivalent to LSST. The robustness to unexpected false
positives is further tested with simulations performed by LSST,
as described below. 
   



