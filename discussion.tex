

\section{Discussion and Conclusions}

We have discussed here the ability of LSST to contribute to Near-Earth Objects (NEO) discoveries and
the Congressional George E. Brown, Jr. mandate to NASA. We have quantitatively addressed
the robustness of the LSST strategy for discovering NEOs using nightly pairs of observations, and
the expected cumulative completeness for PHAs with $H<22$.

We argued that the observing and data processing strategies chosen by LSST are robust. The
robust determination of the false detection rate in difference images presented here is based
on data obtained with DECam and processed using prototype LSST software. Assuming the same rate,
numerical tests with MOPS demonstrate that even a modest compute system will be adequate to
process LSST data. Even if the false detection rate in difference images is twice as high as
reported here, it can still be handled without a change of baseline cadence, linking criteria, or
increase of computing resources. Quantitatively, the false detection rates of up to about
1000 deg$^{-2}$ could be readily handled with an approximately 1000-core machine dedicated for
moving object processing.

There is a significant compute margin, both because LSST compute needs are driven by other more
demanding processing, and because there are various mitigation strategies that can decrease the
compute load. For example, minor modifications of the cadence, such as a simple shortening
of the nightly revisit time by about a factor of 3, could mitigate about a factor of two increase in
the false detection rate. At the same time, a several times larger false detection rates for LSST
than measured using DECam images and prototype LSST software are rather implausible. If the
LSST camera, or any other system component, would somehow cause such high false detection rates,
the whole LSST mission would be jeopardized.

In summary, the LSST strategy for discovering moving Solar System objects will be successful
because the following three conditions will be met:
\begin{enumerate}
\item The LSST system hardware and image differencing software performance will result in false detection
          rates not significantly exceeding $\rho_{FP} =  400$ deg$^{-2}$, estimated here using DECam data
          and prototype LSST software.
\item Given an anticipated 1000-core machine, MOPS will be able to easily process as many as
         10$^8$ tracklets per search window, and daily computations to produce up to about 10$^7$
         candidate tracks will be completed in about an hour.
\item Given that IOD can be executed within 0.1 sec per track, this final filtering step can
         be easily accomplished in about an hour.
\end{enumerate}


Our results show that the current LSST baseline survey strategy would yield a completeness for PHAs
with $H<22$ of about 68\% (without including known objects). 
We discuss a number of systematic effects that must be taken into
account when comparing different simulations of the same survey, as well as simulations of different
surveys and observing systems. The largest effect, at the level of 5\%, is due to orbital differences
between NEO and PHA populations. We compared our results presented here, as well as our previously
published and consistent results, to an analogous recent study by \citet[]{GMS2016} and \citet{VeresChesley2017neo}. 
We found that purported discrepancies can be fully understood as due to the various systematic effects discussed in
\S\ref{sec:syseff}. All these recent results are consistent within the modeling uncertainties
with even the earliest forecasts of LSST performance (e.g., the 10-year LSST baseline cadence would 
result in 75\% completeness for PHAs with $H<22$, according to \citealt{IvezicNEO2007}). The 
PHA completeness forecasts published in \cite{LSSToverview} are somewhat higher than reported
here, primarily due to using different PHA populations. 

We describe a number of modifications of the LSST baseline survey which potentially could raise the
completeness for PHAs with $H<22$ beyond 80\%. In particular, the ``extra ecliptic visits'' survey
(astro\_lsst\_01\_1016) can boost the completeness to 85\% with a 12-year survey and known objects
accounted for. In this cadence, the main LSST survey (``deep-wide-fast'')
receives as many visits after 12 years as it would receive after 10 years of the
baseline cadence (minion\_1016). Therefore, most of other science programs would
retain their performance if the ``extra ecliptic visits'' survey was adopted for
LSST, and resources identified to prolong the survey from 10 years to 12 years. With substantial additional
investment in data processing to enable detecting to $SNR=4$ level and improved trailed source detections,
the completeness could be further pushed to about 88\%.
