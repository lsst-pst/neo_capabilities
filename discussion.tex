

\section{Discussion and Conclusions\label{sec:discussion}}

We have tested and quantified the ability of LSST to contribute to the census of NEO and PHA populations, as well as options for further improvement of LSST's NEO and PHA yields relative to the baseline cadence.

We expect that the LSST strategy for discovering moving Solar System objects will be successful because the following three conditions are likely to be met:
\begin{enumerate}
	\item The LSST system hardware and image differencing software performance will result in false detection
	rates not significantly exceeding $\rho_{FP} =  450$ deg$^{-2}$, conservatively estimated here using real DECam data
	processed using prototype LSST software.
	\item Given an anticipated 1000-core machine, LSST MOPS will be able to easily process as many as
	10$^8$ tracklets per search window, and daily computations to produce up to about 10$^7$
	candidate tracks will be completed in about an hour. We demonstrate this by running an LSST MOPS prototype on a representative simulated dataset. It has been independently confirmed by \cite{VeresChesley2017mops}, using PanSTARRS-PS1 MOPS.
	\item With the IOD computational budget of 0.1 sec per track -- comfortably above the value of $<26$ms we measure here -- the final track filtering step can
	be easily accomplished in about an hour.
\end{enumerate}

Our determination of the expected false detection rate is based on processing data acquired by DECam using prototype LSST pipelines. With a conservative extrapolation of these data to expected LSST depth, we find expected rates of false detections of $450$~deg$^{-2}$. This estimate includes no provision for real-bogus type classifiers, which have been successfully applied by existing surveys to reduce the false detection rates by an order of magnitude or more \citep[e.g.][]{goldstein15}. It is therefore best to think of it as a {\it conservative upper limit}, possibly overestimating the true rate by a factor of few.

Assuming this rate, numerical tests with LSST MOPS prototypes and representative IOD routines demonstrate that the planned compute system will be (more than) adequate to
process LSST data. Even if the realized false detection rate is twice as high as
the (already conservative) estimate reported here, it can still be handled without a change of baseline cadence, linking criteria, or
increase in computing resources. Quantitatively, the false detection rates of up to about
1000 deg$^{-2}$ can be readily handled with an approximately 1000-core cluster dedicated to moving object processing\footnote{Based on LSST's technology estimates and the data center sizing model, this equates to just $36$ CPUs, or $18$ compute nodes, in late 2022 at the beginning of the survey.}.


%Significant further compute margin exists, both because the overall LSST compute needs are driven by other more
%demanding processing\footnote{At the beginning of the survey, the LSST compute cluster is expected to count $\sim 22,000$ cores.}, and because there are various mitigation strategies that can decrease the
%compute load. For example, minor modifications of the cadence, such as a simple shortening
%of the nightly revisit time by about a factor of 3, could mitigate about a factor of two increase in
%the false detection rate. At the same time, a several times larger false detection rates for LSST
%than measured using DECam images and prototype LSST software are rather implausible. If the
%LSST camera, or any other system component, would somehow cause such high false detection rates,
%the whole LSST mission would be jeopardized.

Having established that the LSST will successfully identify moving objects, we've examined the expected discovery yields for the PHA and NEO populations. Our results show that the current LSST baseline survey strategy would yield a completeness for PHAs with $H<22$ of about 66\%, and NEOs with $H<22$ of about 61\% (without including objects discovered by prior and contemporaneous surveys). We compared our results presented here to two analogous recent studies by \citet[]{GMS2016} and \citet{VeresChesley2017neo}, and find them consistent within the modeling uncertainties. The original
PHA completeness forecasts published in \cite{IvezicNEO2007} are higher than reported here (75\% vs 66\%), primarily due to using different PHA populations and a 30-day search window.

Finally, we examine and quantify the efficiency of a number of possible extensions to the LSST baseline survey that could raise the
completeness for PHAs with $H<22$ beyond 80\%. We find that the ``extra ecliptic visits'' strategy (astro\_lsst\_01\_1016), executed over 10 years and with a 30-day MOPS linking window could boost the overall PHA completeness to 84\% (including objects discovered by contemporaneous surveys). The downside to this option is that the remaining LSST science cases would receive $\sim 20$\% less observing time, likely rendering it unacceptable to the broader LSST community.

However, if the survey is extended to 12 years and utilizes this strategy the main LSST proposal (``deep-wide-fast'') receives as many visits after 12 years as it would receive after 10 years of the
baseline cadence (minion\_1016). Additionally, the PHA completeness rises further, to 86\%. This makes this strategy particularly attractive as, assuming resources were identified to extend the survey by two years, the fraction of discovered PHAs would approach $\sim 90$\% without seriously disadvantaging other LSST science cases.
