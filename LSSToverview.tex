\subsection{Brief Overview of LSST} 

LSST will be a large, wide-field ground-based optical telescope system
designed to obtain multiple images covering the sky that is visible
from Cerro Pach\'{o}n in Northern Chile. The current baseline design,
with an 8.4m (6.7m effective) primary mirror, a 9.6 deg$^2$ field of
view, and a 3.2 Gigapixel camera, will allow about 10,000 square
degrees of sky to be covered every night, with typical 5$\sigma$ depth 
for point sources of $r\sim24.5$ (AB). The system is designed to yield 
high image quality (the median delivered seeing in the $r$ band of 
about 0.8 arcsec) as well as superb astrometric  and photometric 
accuracy\footnote{For detailed specifications, please see the LSST
Science Requirements Document, http://ls.st/srd}. The total survey
area will include $\sim$30,000 deg$^2$ with $\delta<+34.5^\circ$, and 
will be imaged multiple times in six bands, $ugrizy$, covering the 
wavelength range 320--1050 nm. For a more detailed, but still concise,
summary of LSST, please see 
the LSST Overview paper\footnote{arXiv:0805.2366, http://ls.st/2m9}. 

The project is scheduled to  begin the regular survey operations at
the start of next decade. About 90\% of the observing time will be
devoted to a deep-wide-fast survey mode which will uniformly observe 
a 18,000 deg$^2$ region about 1000 times (summed over all six bands) 
during the anticipated 10 years of operations, and yield a coadded map 
to $r\sim27.5$. These data will result in catalogs including about
$40$ billion stars and galaxies, that will serve the majority of the
primary science programs. The remaining 10\% of the observing time
will be allocated to special projects such as a Very Deep and Fast
time domain survey\footnote{Informally known as ``Deep Drilling Fields".}.

The LSST will be operated in fully automated survey mode. The images
acquired by the LSST Camera will be processed by LSST Data Management
software \cite{juric15} to a) detect and characterize imaged
astrophysical sources and b) detect and characterize temporal changes
in the LSST-observed universe. The results of that processing will be
reduced images, catalogs of detected objects and the measurements of 
their properties, and prompt alerts to ``events'' -- changes in
astrophysical scenery discovered by differencing incoming images
against older, deeper, images of the sky in the same direction ({\em
emplates}, see \S \ref{sec:AppA} for more details). 
