\section{LSST Observing Cadence Optimization to Enhance PHA Completeness \label{sec:opsim}}

By observing in pairs of visits -- a strategy validated in the previous sections -- rather than triplets or quads of visits in each night, we can increase sky coverage and improve the survey efficiency. However, even within this constraint, there are multiple approaches to surveying the sky. In this section we evaluate the effects of varying the LSST observing strategy and the resulting PHA completeness.

This evaluation is carried out using a combination of the LSST Operations Simulator (OpSim) and the LSST Metrics
Analysis Framework (MAF).
The LSST Metrics Analysis Framework (MAF) is a user-oriented, python package for evaluating the pointing history
from these simulated surveys in light of particular science goals or interests. The various metrics coded in the
MAF framework can be calculated for any given simulated survey and compared as simulation parameters are changed
in OpSim. This permits a thorough investigation of the trades between different observing strategies, in terms of the
effect on multiple science goals, including the PHA completeness.  We first describe the basic steps in our simulations,
then describe the baseline and modified LSST simulated surveys, and then discuss our results.


\subsection{Simulations of LSST Asteroid Discoveries}

The basic components of our end-to-end simulation of asteroid discovery, described in detail below, include
\begin{enumerate}
\item {\it NEO Population Modeling.} Orbital parameters are used to generate asteroid positions during the
simulated survey duration for a simulated or properly debiased extant NEO population. The population needs
to adequately sample color, size and other properties. A database of such positions evaluated with an adequate
time step  is available as an input to MAF.
\item {\it Survey Cadence Modeling.} A series of LSST pointings with instrumental metadata and observing conditions
is generated by OpSim. In addition to boresight positions, the camera orientation and selected filter, available
metadata enable the computation of instrumental sensitivity (limiting magnitudes).
\item {\it Asteroid Optical Flux Modeling.}  Optical flux from an arbitrary asteroid needs to be computed
as a function of the positions of the Sun, the asteroid and Earth, and asteroid physical properties (e.g., size
and color). This model is implemented in MAF.
\item {\it Source Detection Modeling.} Given the instrument model, observing conditions and asteroid flux,
the signal-to-noise ratio is estimated and used to compute detection probability. This model is implemented
in MAF.
\item {\it Detection Linking Modeling.}  Instead of running MOPS, a model that emulates MOPS
performance is used to significantly speed up the computations. This is equivalent to assuming that an object is discovered 
if a given pattern of observations for an object is achieved. This model is implemented
in MAF.
\item {\it Completeness Estimation.} Given a list of ``discovered objects'', and the input population,
the completeness is estimated as a function of asteroid properties (e.g. size) and various other parameters
(e.g. observing strategy). This model is implemented in MAF.
\end{enumerate}

We proceed to describe these models in more detail, and then discuss the baseline and several modified LSST
surveys, and the corresponding PHA completeness estimates.

\subsubsection{NEO Population Modeling \label{sec:MAFdetails}}

We use random samples from the synthetic solar system model (S3M) presented in \cite{Grav2011} in order to model completeness for NEOs and PHAs. We have chosen a sample of 2000 NEOs from the \cite{Grav2011} NEO population, which is based on the \cite{Bottke2002} model. We chose a separate sample of 2000 PHAs from the same model, by choosing NEOs with a MOID $\le 0.05$~AU. The PHA population is useful for evaluating PHA completeness directly; the NEO population is useful for comparison to other survey evaluations. A plot of the $a$, $e$, $i$ distributions for these PHAs and NEOs is shown in Figure~\ref{fig:PHA_orbits}.

With this small set of orbits, we then assume that the $H$ magnitude distribution is independent of the orbital distribution. For most small body populations, including the PHA population larger than 140~m in diameter, this is approximately true. Assuming an independent distribution, each orbit can be ``cloned'' from the fiducial $H$ magnitude to a range of values covering the interesting sizes for analysis; this allows the analysis to use a large number of objects at each $H$ value, without requiring extensive resources to generate ephemerides for a much larger set of orbits. We use the small population of 2000 NEOs or PHAs and clone them to a range of $H$ magnitudes between $H$=11 and $H$=28 using $dN/dH = 10^{\alpha\, H}$, with $\alpha=0.3$ \citep{2017Icar..284..114S}. We have verified with a larger simulated set of NEOs that reducing the population from 10,000 to 2000 objects does not change the calculated survey completeness significantly.

Using the details of the input population, MAF generates the expected observations of each object using the pointing history
from a specific OpSim simulated survey. Ephemerides are generated using OpenOrb \citep{OpenOrb2009} for a closely spaced grid
of times (typically every 2 hours), and then interpolated to the exact times of each OpSim pointing.


\begin{figure}[t!]
\centering
\includegraphics[width=0.49\textwidth]{figures/phas_2k_orbits}
\includegraphics[width=0.49\textwidth]{figures/neos_2k_orbits}
\vskip -0.2in
\caption{The eccentricity and inclination distributions, as a function of semi-major axis, of the PHAs (left) and NEOs (right) used in this analysis. Both populations were randomly sampled from the S3M model \citep{Grav2011}, a synthetic solar system model based on the \cite{Bottke2002} NEO orbital distribution. NEOs are defined as objects with $q<1.3$~AU; PHAs are defined as having a Minimum Orbit Intersection Distance (MOID) with Earth of less than 0.05~AU (implying $q\le1.05$~AU) and having $H\le22$.  \label{fig:PHA_orbits}}
\end{figure}

\subsubsection{Survey Cadence Modeling}

The LSST Operations Simulation \citep[OpSim,][]{delgado14} is a python software package that generates a realistic pointing history, with the time, filter, location, astronomical conditions, weather conditions, and predicted point-source $5\sigma$ limiting magnitude, for each LSST visit
for, typically, ten years. This pointing history is generated using weather data (cloudiness and seeing) from the Cerro Pach\'{o}n site and a high-fidelity model of the telescope itself (including slew and settle time and dome movement, for example), combined with a parameterized set of observing proposals that determine how the scheduling algorithm attempts to gather observations. By configuring OpSim with different parameters for the observing proposals, we can generate a series of simulated surveys which prioritize different science goals. The LSST baseline survey and its modifications designed to enhance the PHA completeness are described in detail
in \S\ref{sec:surveys} below.


\subsubsection{Asteroid Optical Flux Modeling}

Given $H$ magnitude for an object, its apparent magnitude in Johnson's $V$ band can be easily computed
given the positions of the object, the Sun and the observer (e.g. \citealt{juric02}).
Magnitudes, or fluxes, in any other optical and near-IR band (in case of LSST, $u$, $g$, $r$, $i$, $z$, and $y$)
can be computed from $V$ magnitude by specifying a spectrum for each object. We have
assumed that our entire NEO population has the same spectral energy distribution as C-type main-belt asteroids.
The computed color transformations for LSST bandpasses are listed in Table~\ref{tab:sed_colors}. Choosing the
spectral energy distribution of  S-type main-belt asteroids instead results in $<$1\% changes in completeness.
These simulation-based colors were verified using SDSS observations \citep{2001AJ....122.2749I} and analogous
computations with SDSS bandpasses.

\begin{deluxetable}{ccccccc}
\centering
\tablecolumns{7}
\tablecaption{Color transformations from Johnson's $V$ band to LSST bandpasses, for C and S type asteroids. \label{tab:sed_colors}}
\tablewidth{0.7\textwidth}
\tablehead{ Type & $V-u$ & $V-g$ & $V-r$ & $V-i$ & $V-z$ & $V-y$  \\ }
\startdata
C  & -1.53 &  -0.28 &  0.18 &  0.29 &  0.30 & 0.30 \\
S & -1.82 &  -0.37 &  0.26 & 0.46 &  0.40 & 0.41  \\
\enddata
\end{deluxetable}


\subsubsection{Source Detection Modeling}


\begin{figure}[t!]
\centering
\includegraphics[width=0.65\textwidth]{figures/focalplane}
\caption{Model of the LSST camera footprint, including chipgaps and CCD + raft layout. \label{fig:camera_footprint}}
\end{figure}

If the object is within the LSST field of view, its predicted position, velocity, and apparent $V$ magnitude (calculated from the fiducial $H$ magnitude associated with the orbit) is recorded along with information about the simulated observation itself (such as the seeing, limiting magnitude, filter, and boresight RA/Dec). The full LSST camera footprint (see Figure~\ref{fig:camera_footprint}), including chip gaps, is
used to determine whether an object is within the field of view.

MAF also calculates signal-to-noise (SNR) loss due to trailing for each observation, which is required when evaluating whether
a particular object is detectable in a given observation. Trailing losses occur whenever the movement of an object spreads its photons over a wider area than a simple stellar point spread function (PSF). There are two aspects of trailing loss to consider: SNR losses and detection algorithm losses. The first is the
irreversible degradation in SNR that occurs because the trailed object includes a larger number of background pixels in its footprint, compared to a stationary PSF. The second effect, detection loss, occurs because source detection software is optimized for detecting point sources; a stellar PSF-like matched filter is used when identifying sources that pass above the defined threshold. This filter is non-optimal for trailed objects but losses can be mitigated with improved software ({\it e.g.} detecting to a lower PSF-based SNR threshold and then using a variety of trailed PSF filters to detect sources). When considering whether a source would be detected at a given SNR using typical source detection software, the sum of SNR trailing and detection losses should be used. With an improved
algorithm optimized for trailed sources (implying additional scope for LSST data management), the smaller SNR losses should be
used instead.

\begin{figure}[t!]
\centering
\includegraphics[width=0.85\textwidth]{figures/trailing_losses}
\caption{Trailing losses for 30 second LSST visits, assuming seeing of
  0.7''. The dotted line shows SNR trailing losses, the solid line
  indicates cumulative losses that also account for non-optimal detection
  algorithm. With software improvements the latter detection losses can be
  mitigated. At the fiducial $v=1$ deg/day, the SNR loss is $\sim$0.3 mag,
  and non-optimal detection algorithm contributes an additional loss of
  $\sim$0.16 mag.
\label{fig:trailinglosses}}
\end{figure}

Our simulations of these effects show that both types of trailing losses can be fit well with the
same functional form:
\begin{eqnarray}
\Delta \, m & = &-1.25 \, log_{10} \left( 1 + \frac{a \, x^2} { 1 + b\,
    x} \right) \\
x & = & \frac{v \, T_{exp}} {24 \, \theta}
\end{eqnarray}
where $v$ is the velocity (in deg/day), $T_{exp}$ is the exposure time (in seconds), and $\theta$ is the FWHM (in arcseconds). For trailing SNR losses, we find $a = 0.67$ and $b = 1.16$; for the cumulative loss, that includes both SNR and detection losses,
we find $a=0.42$ and $b=0$. An illustration of the magnitude of these trailing loss effects for 0.7 arcsec seeing is given in Figure~\ref{fig:trailinglosses}.

We calculate the probability of detecting a particular source given its magnitude $m$
and the $5\sigma$ limiting magnitude $m_5$ (after accounting for trailing losses) using a logistic function
\begin{eqnarray}
     P & = & \left[ 1 +  {\rm exp}\left(\frac {m -  m_5}{\sigma}\right) \right]^{-1}.
\end{eqnarray}
where $\sigma$=0.12 describes the width of the completeness falloff \citep{2014ApJ...794..120A}. A source is randomly classified
as detected using the probability $P$. We also evaluate more optimistic discovery criteria using only SNR trailing losses
(i.e. without taking detection losses into account), as well as detections to SNR=4 instead of SNR=5.  This detection model assumes a flat $m_5$
value across the focal plane; no vignetting or background variation (or masking) due to bright stars is included. These effects however are small.


\subsubsection{Detection Linking Modeling}

Once we have computed the set of all visits in which a given object was within the field of view and detected, we locate subsets of these visits that match our target discovery criteria. These criteria generally consist of a given number of visits within a specified
time span within a single night, followed by a given number of additional nights (each with the same required number
of visits in the same time span) falling within a specified time window. The basic criteria is a pair of visits in each
night occurring within 60 minutes, repeated for 3 nights within a 15 day time window. However, we also evaluate
the effect of varying the discovery criteria to require triplets or quads of visits within a single night, and increase
the length of the search window from 15 to 30 days. An independent study  assessed the linking performance of these 
criteria and concluded that they result in complete and accurate linking, including rejection of false links using orbit determination 
(Vere\v{s}, P. \& Chesley, S. 2017, submitted to Astronomical Journal).


\subsubsection{Completeness Estimation}

With each unique set of discovery criteria, we have a record of what objects would be ``discovered'' at each $H$ value.
With this we calculate the differential discovery completeness, the fraction of objects discovered at a given $H$ magnitude.
To turn this into a cumulative discovery completeness, we simply integrate over $H$, assuming a given $H$ distribution
for the population (recall that we use $dN/dH = 10^{\alpha\, H}$, with $\alpha$ = 0.3).


\subsection{OpSim Simulated Surveys \label{sec:surveys}}

\subsubsection{The LSST Baseline Survey}

The current baseline observing strategy for LSST is represented by our reference run, minion\_1016. This simulated survey
contains observations balanced between several different observing proposals:
\begin{enumerate}
\item The Wide, Fast, Deep (WFD) proposal (also known as the Universal proposal) is the primary LSST survey, expected to receive about 90\% of the observing time and to cover 18,000 deg$^2$ of sky. In the baseline observing strategy, this proposal is configured to obtain visits in pairs spaced about 30 minutes apart, and will typically return to each field about every 3-4 days, balancing the six $ugrizy$ filters. The footprint for the WFD proposal covers approximately $+5^\circ$ to $-60^\circ$ in declination, with a full range of RA values except for a region around the Galactic plane. This declination range corresponds to an airmass limit of about 1.3 when the fields are at an Hour Angle of $\pm$2 hours. In minion\_1016, the WFD proposal receives 85\% (2,083,758) of the total number of visits.
\item The North Ecliptic Spur (NES) proposal is an extension to the WFD to reach the northern limits of the Ecliptic plane ($+$10 degrees), and allows higher airmass observations. The visit timing is similar to the WFD, although the $u$ and $y$ filter are not requested in this region. In the baseline observing strategy, minion\_1016, each NES field requests about 40\% of the total number of WFD visits per field when considering $griz$ filters only (304 visits per field in $griz$ vs 795 visits per field in $griz$ in WFD), and receives 6\% (158,912) of the total number of visits.
\item The Deep Drilling Fields (DD) proposal includes a set of single pointings that are requested in extended sequences; currently these sequences are $grizy$ visits, with additional coverage in $u$ band. Each sequence requires about an hour of observing time, and is repeated every few days. In minion\_1016, there are 5 DD fields, 4 of which correspond to fields which have been officially selected
by the Project and announced to the community; these five fields receive 5\% of the total visits.
\item The Galactic Plane (GP) proposal covers the region with high stellar density around the Galactic plane not covered by the WFD. This proposal requests a small number of visits in each of the six $ugrizy$ filters, with no timing constraints. In minion\_1016, this proposal receives 2\% of the total visits.
\item The South Celestial Pole (SCP) proposal is an extension of the WFD footprint to cover the region south of $-60^\circ$ declination. Like the GP, this proposal requests a small number of visits in each of the six $ugrizy$ filters, with no timing constraints. In minion\_1016, this proposal receives 2\% of the total visits.
\end{enumerate}

The footprint of these various proposals in the baseline minion\_1016 reference run is shown in Figure~\ref{fig:minion_footprints}. In each proposal, the individual visits are 30 seconds long, consisting of two back-to-back coadded 15 second exposures.

\begin{figure}[t!]
\centering
\includegraphics[width=0.85\textwidth]{figures/minion_1016_proposal_footprint}
\vskip -1.0in
\caption{The footprints of the various proposals included in the baseline observing strategy, represented by reference run minion\_1016.
\label{fig:minion_footprints}}
\end{figure}


\subsubsection{Modified Surveys}

A series of additional OpSim simulated surveys were created with parameters intended to improve the efficiency of discovering PHAs and increase the cumulative PHA completeness. They span the range from minor modifications to extreme changes that would
jeopardize other LSST science goals. We consider the latter in order to assess what would be ultimate performance of an
LSST-like system fully dedicated to NEO surveying.

\textbf{Extra ecliptic spur visits:} The first cadence change is simply to increase the number of visits requested for the NES proposal in $griz$, to increase the likelihood of discovering objects near the northern portion of the Ecliptic plane, and to extend the survey from 10 years to 15 years, increasing the discovery rate of PHAs with long synodic periods. Other proposals remain the same as in minion\_1016. Reprioritizing the NES relative to the WFD results in the WFD receiving 69\% (2,561,334 visits over 15 years) of the total visits and the NES receiving 24\%, compared to 85\% and 6\% respectively in the baseline strategy. The resulting simulated survey is astro\_lsst\_01\_1016.  We consider astro\_lsst\_01\_1016 as our 'baseline PHA' run, as it makes minimal changes to the overall survey strategy while attempting to be more PHA friendly.

\textbf{Longer ecliptic visits:} This simulation introduce an Ecliptic Band (EB) proposal, requesting observations with visit timing similar to the WFD in the $griz$ filters, but with field locations surrounding the Ecliptic Plane $\pm15^\circ$ and extending down to the WFD where the Ecliptic reaches its northernmost  range. This proposal replaces the NES proposal, and requests longer 60 second visits in order to reach deeper limiting magnitudes. Other proposals remain the same as in astro\_lsst\_01\_1016. With this reprioritization, the WFD receives 44\% (1,159,319) of the total visits while the EB receives 53\%; the simulated survey is astro\_lsst\_01\_1015.

\textbf{NEO-focused survey:} An attempt at an aggressively NEO-optimized survey was also created, where modified versions of the EB and WFD proposals are used. The visit timing in each proposal is the same as the standard WFD visit timing (a pair of visits separated by about 30 minutes), however the WFD footprint is changed to simply cover the entire area between 0$^\circ$ and $-60^\circ$ declination except for the EB footprint. The EB and WFD proposals request observations in only $gri$ filters, with 30 second visits in the WFD and 60 second visits in the EB. No other proposals are included. The resulting simulated survey is astro\_lsst\_01\_1017.

\begin{figure}[t!]
\centering
\includegraphics[width=0.49\textwidth]{figures/astro_lsst_01_1015_proposal_footprint}
\includegraphics[width=0.49\textwidth]{figures/astro_lsst_01_1017_proposal_footprint}
\vskip -0.5in
\caption{The footprints of the proposals, including the Ecliptic Band proposal, used in the NEO-optimized simulated surveys astro\_lsst\_01\_1015 (the ``longer ecliptic visits'' survey, left) and astro\_lsst\_01\_1017 (the ``NEO-focused'' survey, right). The astro\_lsst\_01\_1017 survey only includes two proposals.
\label{fig:neo_footprints}}
\end{figure}


\subsection{Completeness analysis results}

In the baseline reference run, minion\_1016, with the baseline discovery
criteria (pairs of visits occurring within 60 minutes and
repeated for 3 nights within a 15 day time window), we find a cumulative
completeness at $H\le22$ of 65.6\% for our PHA input population (see
Figure~\ref{fig:minionC1}). This should be considered our initial baseline PHA
completeness, as it uses the reference run and the baseline MOPS and data
management requirements.

From this baseline, we can evaluate the effects of changing both the survey design (reallocating telescope resources) and the discovery criteria, which effectively sets the computational resources.
There is an interplay between discovery criteria and survey design -- as an obvious example, discovery criteria requiring triplets of visits per night instead of pairs will result in much different completeness results if the survey was designed to only request two visits per night rather than three. Likewise, some changes in survey design work best with changes to the discovery criteria; for example, lengthening the visit time increases the detection losses and pushing source detection to the ``trailing loss'' limit is required for a significant improvement in completeness. While we compare discovery criteria within a single simulated survey as much as possible, there are some changes in discovery criteria which must be compared between different surveys using different observing strategies.

All the completeness results presented below assume that no objects are known prior to LSST survey,
and thus are biased low. For example, by assuming that 43\% PHAs would be discovered by the start of
LSST survey, \cite{GMS2016} showed that the final PHA completeness for LSST baseline survey would
be boosted by 11\%. We discuss our own independent estimates of this effect in \S\ref{sec:known}.

\begin{figure}[t!]
\centering
\includegraphics[width=0.99\textwidth]{figures/minion_1016_CumulativeCompleteness_NEO_and_PHA_Cumulative_Completeness}
\vskip -0.2in
\caption{The cumulative completeness for PHAs and NEOs, as a function of absolute magnitude $H$, for the baseline
cadence minion\_1016. The completeness is below 100\% at the bright end (large size limit) because some objects have
synodic periods longer than the survey duration (and thus effectively ``hide'' behind the Sun), and some are visible but
 do not receive the required number of observations due to telescope scheduling. \label{fig:minionC1}}
\end{figure}

\begin{deluxetable}{lccccccc}
\tablecaption{The cumulative completeness for PHAs with $H\le22$ for various
survey strategies (rows) and discovery criteria (columns). In addition to
changing the overall duration of the survey (12 years instead of 10), the
completeness is shown for different track linking windows ($N_w=15$ or $30$ days),
enhanced detection algorithms to reduce trailing losses (``Trail Det''), and
pushing the individual detection threshold from SNR$=5$ to SNR$=4$. These are
primarily computational changes, while the various rows show different survey
cadences. These range from the current baseline, to adding additional visits or
longer visits in the ecliptic region, to focusing the majority of the time on
performing a NEO-focused survey. \label{tab:completeness}}
\tablehead{
& \multicolumn{3}{c}{10 year survey}  &  \multicolumn{4}{c}{12 year survey}  \\
\cmidrule(r){2-4} \cmidrule(r){5-8}
Simulation  & $N_w$=15 & $N_w$=30 & $N_w$=30 & $N_w$=15 & $N_w$=30 & $N_w$=30 & $N_w$=30 \\
            &      &      & Trail Det&  & & Trail Det &  SNR=4
%            &      &      &      &      & &           & SNR=4
}
\startdata
LSST baseline & 65.6 & 68.4 & 69.1 &  --- & --- & --- & --- \\
Extra ecliptic visits & 66.1 & 69.8 & 70.5 & 70.5 & 73.9 & 74.8 & 77.1 \\
Longer ecliptic visits & 63.2 & 67.5 & 70.5 & 67.3 & 71.7 & 74.5 & 75.7 \\
NEO-focused cadence & 66.5 & 70.3 & 72.3 & 70.2 & 73.8 & 75.8 & 77.2 \\
\enddata
\end{deluxetable}


\subsubsection{Modified Discovery Criteria \& Computational Strategies}

Using only the baseline minion\_1016 and not changing the survey strategy, we can explore the impact on PHA completeness of changing the discovery criteria. These correspond primarily to the different columns of Table~\ref{tab:completeness}, and include:

\begin{itemize}
\item extending the MOPS window for linking pairs of detections from the nominal 15 day window to a 30 day window: this increases completeness by about 3\%, although with an estimated increase in the compute requirements by about an order of magnitude (see Appendix \ref{sec:appMOPS}).
\item using sources detected down to SNR=4 instead of SNR=5: this increases completeness by about 3\%, although with an estimated increase in the compute requirements by about two orders of magnitude (see \S\ref{sec:kaiser}).
\item enhancing source detection algorithms to mitigate detection losses to the trailing loss level: with the 30 second visits in the baseline minion\_1016, this only increases completeness by a very small amount, about 0.5\%.
\end{itemize}

Increasing the MOPS linking window from 15 to 30 days achieves a substantial gain in completeness for the baseline survey
(or any survey with a maximum visit time of 30 seconds) without a significant computational cost. The increased window allows more opportunities to capture the PHAs in a set of observations which meet the discovery criteria. It is worthwhile to note that the current OpSim behavior does not prioritize capturing large chunks of contiguous sky, and often leaves gaps in coverage from night to night. This behavior is likely related to the increase in completeness going from 15 day windows to 30 day windows; with the large LSST field of view, after 30 days the areal coverage will be much more evenly distributed than after 15 days. Changes to the scheduling algorithm to favor covering contiguous blocks of sky\footnote{A similar modification of
the baseline cadence, the so-called ``rolling cadence'', is also favored by the supernovae science programs. A release of a series of simulated surveys implementing this idea is anticipated for late 2017.} are likely to improve the completeness even further. Completeness could potentially be increased by over 10\% by better scheduling; this is illustrated in Figure~\ref{fig:more_completeness}, where only requiring a single night of pairs or requiring 6 observations in any sequence over 60 nights increases completeness over 10\%.  Pushing to SNR=4 requires substantial
compute resources and is not cost effective in comparison.


\begin{figure}[t!]
\centering
\includegraphics[width=0.99\textwidth]{figures/minion_1016_neo_More_Cumulative_Completeness}
\vskip -0.2in
\caption{The cumulative completeness for NEOs, as a function of absolute magnitude $H$, for the baseline
cadence minion\_1016, considering a variety of detection requirements. The completeness is below 100\% at the bright end (large size limit) because some objects have
synodic periods longer than the survey duration (and thus effectively ``hide'' behind the Sun), and also because some objects do not receive the required number of observations within the `window'. It is not due to the limited sensitivity of the LSST system, as can be seen by the ``3 pairs in 15 nights, $\infty$ sensitivity'' line, which shows the completeness expected when the discovery requirement is 3 nights with pairs of observations within a 15 night window but assuming an infinitely sensitive survey. The potential gains with better optimized scheduling (observing larger contiguous chunks of sky, for example), can be seen in the difference between the single pair of detections or 6 separate detections within 60 nights, vs. 3 pairs of detections in 15 or 30 nights (indicating potential gains of over 10\% in completeness for $H\le22$). \label{fig:more_completeness}}
\end{figure}


When an NEO population is used instead of a PHA input population, the cumulative completeness is about 5\% lower
(see Figure~\ref{fig:minionC1}). This is primarily due to differences in their orbital distributions, as illustrated in Figure~\ref{fig:PHA_orbits}. The definition of PHAs includes a Minimum Orbit Intersection Distance (MOID) with Earth of 0.05~AU, requiring PHAs to more closely approach Earth than NEOs (which are defined as simply having $q<1.3$~AU), and thus the PHAs achieve brighter peak V magnitudes than the NEOs. To quantify this effect, we calculated the apparent $V$ magnitude for both the NEO and PHA input populations every night for ten years, while accounting for trailing losses and assuming a constant $H=22$ magnitude for every member of the population. The resulting distributions of the
brightest 10-year $V$ magnitude values are shifted by about 0.3 magnitudes (the mean brightest magnitude values are 22.6 for NEOs and 22.3 for PHAs).


\subsubsection{The Performance of Modified Surveys}

The potential improvement in PHA discovery rates for modified survey cadences is
summarized in the rows of Table~\ref{tab:completeness} and described below.

\begin{itemize}
\item \textbf{Extra ecliptic spur visits:} By adding these extra visits over the course of a 10 year survey, the increase in completeness over the LSST baseline is only about 1\%. This improvement comes at a cost to other science cases, as the main survey footprint (the WFD proposal) only receives 1,715,354 visits (82\%) of the number of visits in the reference run; the outcome of many science programs is roughly proportional to the number of visits.
\item \textbf{Extending the survey by two years:} Since minion\_1016 is a reference run, it only simulates 10 years.
However, we can evaluate the ``extra ecliptic visits'' run at the 12 year mark, at which point the WFD proposal has received approximately the same number of visits as it would receive in the baseline 10 year survey. The additional two years of operations boost the completeness
by about 4\%.
\item \textbf{Longer visits in the ecliptic:} This strategy reaches fainter limiting magnitudes, but the effect of longer exposures alone is minimized by the fact that trailing losses are also increased. It is also hard to disentangle the effects of increasing the visit time near the Ecliptic and the resulting lower frequency of observations (and thus fewer opportunities for sets of observations matching the basic discovery criteria). The small modifications made by this survey strategy actually show a decrease in completeness, {\it until} detection losses are partially compensated for by modifying source detection algorithms to the trailing loss level; then this run provides a similar completeness level as the ``extra ecliptic visits'' survey at twelve years. Presumably this is where modifying the observing strategy to favor large contiguous chunks of sky would make a significant difference, consolidating the fewer ecliptic visits into a shorter amount of time suitable for object discovery.
\item \textbf{Aggressively NEO-focused survey:} This survey uses a limited filter set, discards other proposals, and uses longer exposures along the ecliptic. This survey shows a modest increase in completeness (about 1.5\%) relative to the ``extra ecliptic visits'' survey, after using longer MOPS windows and pushing source detection to the trailing loss level.
However, many science programs would be jeopardized with this observing strategy because observations in the $uzy$ filters,
and observations of the DD and SCP fields, would not be obtained.
\end{itemize}

To summarize, when altering the survey strategy the largest individual gain ($\sim$4\%)
comes from simply extending the survey lifetime from 10 to 12 years. For the case of PHAs and 30-day wide MOPS window,
the completeness can be boosted from 65.6\% after 10 years with minion\_1016 survey to 73.9\% after 12 years with
``extra ecliptic visits'' survey (recall that this completeness estimates do not account for the contribution of known objects).


\subsection{LSST in context of the wider NEO Discovery System\label{sec:known}}

The completeness results presented above assumed that no objects are known prior to LSST survey.
By including known objects, the completeness is boosted to higher values; the current (2016)
completeness for NEOs with $H<22$ is estimated to be about 25\% \citep{GMS2016}. According to the JPL NEO discovery page\footnote{See http://neo.jpl.nasa.gov/stats/}, one can conservatively estimate that
discovery of 140m NEOs started in earnest in 2000.

We use a simplified model to estimate the contribution of known objects discovered both prior
and during LSST survey. Since we do not know the
survey pointing history for all the previous surveys, we simulate them by adopting a simple solar elongation cut and a two-step Johnson $V$ magnitude threshold: $V_{max1}$ before 2015 and $V_{max2}$ after 2015, when several more sensitive surveys such as PanSTARRS started reporting NEO discoveries. We set the solar elongation cut to be $60^\circ$, roughly matching most telescope pointing capabilities, and assume $V_{max2}$ (the limiting magnitude after 2015) to be 22.0, corresponding to the limiting magnitude expected from current state of the art surveys such as Pan-STARRS1. We integrate orbits for our synthetic NEO model
population from 2000 to 2032 and consider an object discovered if its peak $V$ magnitude is brighter than $V_{max1}$ (before 2015) or $V_{max2}$ (after 2015) while its solar elongation is greater than $60^\circ$.  To determine $V_{max1}$, we vary this threshold until the completeness
for NEOs with $H<22$ in 2016 is $\sim$25\%, obtaining $V_{max1}$=20.0. This also
produces a completeness of 95\% for NEOs with $H<18$ in 2015, i.e. for canonical objects larger than 1 km.
With these conditions, we expect a 44\% completeness for NEOs
and 55\% completeness for PHAs with $H<22$ in 2022 at the start of the LSST survey. To determine the total impact of other surveys on LSST completeness, we continue discovering objects with $V_{max}=22.0$ throughout
the LSST survey; all objects discovered by these other surveys are added to the post-LSST sample, providing
a boost to the final completeness of 8-11\% for NEOs and 11-15\% for PHAs.

Our results are summarized in Figure~\ref{fig:knownObj} and Table~\ref{tab:completeness2}.
For a 12-year survey, we predict a completeness of 85\% for PHAs, when known objects are taken into account.


\begin{deluxetable}{lcccc}
\tablecaption{The cumulative completeness (in \%) for NEOs and PHAs with $H\le22$ for
LSST baseline survey strategy extended to a 12-year survey (with some extra visits along the
Ecliptic, corresponding to the second row in Table~\ref{tab:completeness}). The length of track
linking window ($N_w$) is set to 30 days, and the detection threshold is set to SNR$=5$.
\label{tab:completeness2}}
\tablehead{
& \multicolumn{2}{c}{10 year survey}  &  \multicolumn{2}{c}{12 year survey}  \\
\cmidrule(r){2-3} \cmidrule(r){4-5}
Population  & only LSST & w/ known &  only LSST & w/ known
}
\startdata
    NEO    & 63.5 & 73.9 & 68.1 & 75.9  \\
    PHA    & 68.4 & 82.8 & 73.9 & 85.4  \\
\enddata
\end{deluxetable}


\begin{figure}[t!]
\centering
\includegraphics[width=0.99\textwidth]{figures/astro_lsst_01_1016_completeness.pdf}
\vskip -0.2in
\caption{The cumulative completeness for NEOs (left) and PHAs (right) with $H\le22$, as a function of
time, when known objects (gray solid lines) are taken (black solid lines) and not taken (dotted lines)
into  account (see also Table~\ref{tab:completeness2}).
\label{fig:knownObj}}
\end{figure}



\subsection{Systematic Effects due to Varying Modeling Assumptions \label{sec:syseff}}

As indicated by the above discussion, a number of systematic effects must be taken into account when
comparing different simulations of the same survey, as well as simulations of different surveys and observing
systems. It is unlikely that a meaningful quantitative comparison can be pushed beyond a level of a few percent
in completeness (in practice, the completeness of a given operating survey is best estimated using the object
re-discovery rate). Based on our analysis, the leading systematic effects in simulated completeness estimates are:
\begin{enumerate}
\item NEO vs. PHA difference. The completeness is about $\sim$5\% higher for PHAs than for NEOs; for example,
the cumulative completeness for minion\_1016 is 65.6\% for PHAs and 60.7\% for NEOs with 15 day MOPS windows.
\item Orbital parameter distribution for the simulated asteroid population (e.g. the Bottke model
             vs. the Granvik model); varying populations contribute completeness uncertainty of about a few percent).
\item Different sample definitions: $H<22$ vs. $D>140$m (as shown by \citealt{GMS2016}, completeness
           increases by $\sim$5\% when an $H$-based criterion is used).
\item Variations of the ``discovery window'' (e.g., X visit pairs in $N_w$ nights: changing $N_w$ from 15 to 30 with X=3 increases
          completeness by about 3\%, while changing $N_w$ from 15 to 12 decreases completeness by about 1\%).
          %changing X from 3 to 4 with $N_w=15$ decreases completeness by 6\%).
\item Uncertainties when predicting effective image depth (system throughput, variation of the detection efficiency
          with the signal-to-noise ratio, treatment of trailing losses); for a survey that has a completeness above 60\%,
          each additional {\it 0.1 magnitude of depth for a given survey cadence increases the completeness by another 1\%}.
\item Uncertainties when predicting asteroid's apparent flux (albedo distribution, phase effects, photometric variability
          due to non-spherical shapes, color distributions); assuming an uncertainty of 0.2 mag in the effective
          limiting magnitude, the corresponding  systematic uncertainty in completeness is about 2\%.
\item Variations of the nominal detection threshold. If the detection threshold is changed from the
          signal-to-noise ratio of 5 or greater to 4 or greater, the completeness is boosted by $\sim$3\%;
          the difference between the optimal detection using trailed profile and point-spread-function
          detection, which is negligible for LSST baseline exposure time of 30 seconds, would be worth $\sim$2\%
          in completeness for visits with a doubled exposure time.
\item Sensitivity to details in sky coverage and cadence (e.g. nightly pairs of visits vs. quads of visits).
          Requiring quads instead of pairs of visits decreases completeness by 30\% using the baseline cadence;
          about half of that loss can be recovered using cadence simulations that request four visits per night.
\item The slope of the asteroid size distribution. Current measurement uncertainty of this parameter
          corresponds to a systematic uncertainty in completeness of about 2\%.
\item The impact of known objects. We estimated that 55\% of PHAs with $H<22$ would be discovered
          by current survey assets by the start of LSST survey in 2022 (currently $\sim$36\%), and they would
          boost the final PHA completeness for 10-year LSST baseline survey by 14\% (to 82.8\%).
\end{enumerate}

We proceed with an example of a comparison of different simulations.

\subsection{A Comparison with the Grav, Mainzer \& Spahr (2016) Study \label{sec:GMS}}

\citet*[][hereafter GMS]{GMS2016}
reported somewhat different NEO completeness levels than published by the LSST team
in 2007 and 2014. Given the above discussion of various systematic effects, it is easy to understand
the reported differences. There are three main reasons why the GMS results differ:
\begin{enumerate}
\item GMS used a different definition of the the completeness limit: instead of 
          the commonly used $H<22$ criterion, they used an albedo-dependent value 
          of $H$ limit which attempts to directly model the $D>140$ m size limit.
\item GMS used a different realization of the LSST baseline survey.
\item GMS used a different realization of the PHA population.
\end{enumerate}

The first of these effects is the dominant one.  Redefining the completeness limit from $H<22$ to $D>140$~m leads to a drop in completeness of about 5\% according to GMS ({\it i.e.} GMS would calculate a completeness of about 5\% less for NEOs and PHAs).
Different versions of the simulated survey include advances in our understanding of the system throughput and delivered seeing. For example, between {\it enigma\_1189} (the baseline simulated survey used in GMS) and the new baseline {\it minion\_1016} used here, the limiting magnitudes are on average a few tenths of a magnitude fainter in {\it enigma\_1189} due to changes in the system throughput and delivered seeing; as a result, the completeness values are about 2\% larger for {\it enigma\_1189} than for {\it minion\_1016}.
These two surveys are statistically the same otherwise.
For the NEO population, both GMS and this work used samples from the \cite{Grav2011} model and calculate similar results for completeness (after adjusting for the other two effects).

In summary, the GMS calculation of 62\% completeness for PHAs (12-day window and $D>140$m) corresponds
well to our 65.6\% completeness (see Table 2) for PHAs with $H<22$ (15-day window).  Therefore, after accounting
for different choices of simulation parameters, we conclude the GMS results are fully consistent (within 1-2 \%)
with the results previously published by the LSST team, as well as with the results discussed here.
